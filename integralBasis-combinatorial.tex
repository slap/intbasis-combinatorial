%\usepackage[draft]{hyperref}
%\newcommand{\PE}{{\mathcal P}(x)}
%\newcommand{\PEP}{{\mathcal P}[x]}
%\usepackage{epsfig}
%\usepackage{pstricks}
%\usepackage{pst-node}
%\usepackage{pst-tree}
%\usepackage{pst-grad}
%\usepackage{pst-plot}
%\DeclareMathOperator{\IntegralElementSimple}{IntegralElementSimple}
%\DeclareMathOperator{\LocalIntegralBasisElement}{LocalIntegralBasisElement}

\documentclass[a4paper,11pt]{amsart}%
\usepackage{multicol}
\usepackage{bm}
\usepackage[utf8x]{inputenc}
\usepackage{amssymb}
\usepackage{amsmath}
\usepackage{synttree}
\usepackage{amsthm}
\usepackage{algorithm}
%\usepackage{algpseudocode}
\usepackage[noend]{algorithmic}
\usepackage{todonotes}
\usepackage{amsfonts}
\usepackage{graphicx}
\usepackage{comment}
\usepackage{hyperref}%
\usepackage{enumitem}
\setcounter{MaxMatrixCols}{30}
%TCIDATA{OutputFilter=latex2.dll}
%TCIDATA{Version=5.00.0.2570}
%TCIDATA{LastRevised=Wednesday, February 26, 2014 20:49:07}
%TCIDATA{<META NAME="GraphicsSave" CONTENT="32">}
%TCIDATA{<META NAME="SaveForMode" CONTENT="1">}
\allowdisplaybreaks
\renewcommand{\algorithmicrequire}{\textbf{Input:}}
\renewcommand{\algorithmicensure}{\textbf{Output:}}
\theoremstyle{definition}
\theoremstyle{plain}
\newtheorem{defn}{Definition}[section]
\newtheorem{theorem}[defn]{Theorem}
\newtheorem{proposition}[defn]{Proposition}
\newtheorem{lemma}[defn]{Lemma}
\theoremstyle{remark}
\newtheorem{remark}[defn]{Remark}
\newtheorem{example}[defn]{Example}
\newtheorem{notation}[defn]{Notation}
\DeclareMathOperator{\LocalIntegralBasis}{LocalIntegralBasis}
\DeclareMathOperator{\IntegralElement}{IntegralElement}
\DeclareMathOperator{\IntegralBasisIterative}{\mathtt{IntegralBasisIterative}}
\DeclareMathOperator{\TruncatedFactor}{\mathtt{IntegralBasisElement}}
\DeclareMathOperator{\Splitting}{Splitting}
\DeclareMathOperator{\BlockSplitting}{BlockSplitting}
\DeclareMathOperator{\SegmentSplitting}{SegmentSplitting}
\DeclareMathOperator{\Spec}{Spec}
\DeclareMathOperator{\Sing}{Sing}
\DeclareMathOperator{\Ann}{Ann}
\DeclareMathOperator{\Int}{Int}
\DeclareMathOperator{\Hom}{Hom}
\DeclareMathOperator{\Id}{Id}
\DeclareMathOperator{\degree}{degree}
\DeclareMathOperator{\rad}{rad}
\DeclareMathOperator{\Ker}{ker}
\DeclareMathOperator{\TQR}{Q}
\newcommand{\singular}{{\sc Singular}}
\newcommand{\maple}{{\sc Maple}}
\newcommand{\cc}{{\mathbf c}}
\newcommand{\Q}{{\mathbb Q}}
\newcommand{\N}{{\mathbb N}}
\newcommand{\R}{{\mathbb R}}
\newcommand{\Z}{{\mathbb Z}}
\newcommand{\C}{{\mathbb C}}
\newcommand{\Px}{{\mathcal{P}_X}}

\usepackage{accents}
\newcommand{\dbtilde}[1]{\accentset{\approx}{#1}}
\newcommand{\vardbtilde}[1]{\tilde{\raisebox{0pt}[0.85\height]{$\tilde{#1}$}}}


\begin{document}
\title[Combinatorial approach for integral bases]{A combinatorial approach for computing integral bases}
\author{J. Boehm, W. Decker, S. Laplagne, G. Pfister}

\begin{abstract}
In this paper, we develop an improved algorithm for computing integral bases of plane curves, that is, function fields in one variable, in characteristic zero. The algorithm builds on our previous work \cite{intbas}.  As in \cite{intbas}, the problem is first reduced to the local case of a single singularity at the origin. The main improvement over \cite{intbas} is a combinatorial approach to find the numerators of the integral basis elements. Thus, we do not require to transform the module generators into triangular form. The new approach is particularly useful for curves with many branches at the singularity.
\end{abstract}
\maketitle

\section{Introduction}
\label{section:introduction}

In this paper, we develop an algorithm to compute the integral closure $\bar A$ in terms of an integral basis for the coordinate ring $A$
of an algebraic curve. We focus on the case where the curve is defined over a field $K$ of characteristic zero.
Let $f\in K[X,Y]$ be an irreducible polynomial in two variables, let $C\subset\mathbb{A}^{2}(K)$ be the affine plane curve
defined by $f$, and let
\[
A=K[C]=K[X,Y]/\langle f(X,Y)\rangle
\]
be the {\emph{coordinate ring}} of $C$. We write $x$ and $y$ for the residue
classes of $X$ and $Y$ modulo $f$, respectively. Throughout the paper, we
suppose that $f$ is monic in $Y$ (due to Noether normalization, this can
always be achieved by a linear change of coordinates). Then the
{\emph{function field}} of $C$ is
\[
K(C)=\TQR(A)=K(x)[y]=K(X)[Y]/\langle f(X,Y)\rangle,
\]
where $x$ is a separating transcendence basis of $K(C)$ over $K$, and $y$ is
integral over $K[x]$, with integrality equation $f(x,y)=0$.
Indeed, we have the isomorphism $\TQR(K[x,y]) \rightarrow  K(x)[y]$ defined by
mapping $1/h(x,y) \mapsto b(x,y) / x^c$, where $X^c = a f + b h\in K[X][Y]$
is a representation which arises from a B\'ezout identity in $K(X)[Y]$ by clearing denominators.


Since \(A\) is integral over \(K[x]\), \(\overline{A}\) is equal to the integral closure of \(K[x]\) in \(K(C)\). Hence, \(\overline{A}\) can be represented both as an $A$-module or as a module over $K[x] \cong K[X]$. As noted in \cite[Remark 2]{intbas}, \(\overline{A}\) forms a free \(K[x]\)-module with the rank given by:
\[
n:=\deg_{Y}(f)=[K(C):K(x)].
\]

\begin{remark} (See \cite[Remark 4]{intbas}.)
\label{rem:spec-int-basis}
In the above setting, there exist
polynomials $p_{i} \in K[X][Y]$ of degree $i$ in $Y$ and polynomials
$d_i\in K[X]$ such that
\[
\left\{1,\frac{p_{1}(x,y)}{d_1(x)},\dots,\frac{p_{n-1}(x,y)}{d_{n-1}(x)}\right\}
\]
is an integral basis %\footnote{We say that such an integral basis is in \emph{triangular shape}.}  %\todo{needed?}
for $\overline{A}$ over $K[x]$.

\end{remark}




%In particular, $A$
%is integral over $K[x]$, which implies that $\overline{A}$ coincides with the
%integral closure of $K[x]$ in $K(C)$. We may, hence,
%represent $\overline{A}$ either by module generators over $A$ or by generators over
%$K[x] \cong K[X]$. Note that by \cite[Remark 2]{intbas}, $\overline{A}$
%is a free $K[x]$-module of rank
%\[
%n:=\deg_{Y}(f)=[K(C):K(x)].
%\]
%
%
%\begin{remark}
%\label{rem:spec-int-basis}
%In the context outlined above, there exist
%polynomials $p_{i} \in K[X][Y]$ of degree $i$ in $Y$ and polynomials
%$d_i\in K[X]$ such that
%\[
%\left\{1,\frac{p_{1}(x,y)}{d_1(x)},\dots,\frac{p_{n-1}(x,y)}{d_{n-1}(x)}\right\}
%\]
%is an integral basis %\footnote{We say that such an integral basis is in \emph{triangular shape}.}  %\todo{needed?}
% for $\overline{A}$ over $K[x]$. In fact, such a
%basis is obtained from any given set of
%$K[x]$-module generators for $\overline{A}$ by unimodular row operations over
%the PID $K[X]$: Represent the given generators by polynomials of type
%$c_{i}=\sum_{j=0}^{n-1}c_{ij}Y^{n-1-j}$, with coefficients $c_{ij}\in K(X)$.
%Then take $d$ to be the least common denominator of the $c_{ij}$,
%transform the matrix $(d\cdot c_{ij})$ into Hermite normal form $(p_{ij})$,
%set $\widetilde{p}_{i}=\sum_{j=0}^{n-1}p_{n-1-i ,j}Y^{n-1-j}$ for each $i=0,\dots, n-1$,
%and let the ${p}_{i}(X,Y)/d_i(X)$ be obtained by reducing the
%$\widetilde{p}_{i}(X,Y)/d(X)$ to lowest terms.
%% NOT TRUE:
%%Then the $p_i$ are monic in $Y$
%%since otherwise we would not be able to represent the powers $y^i$ as
%%$K[x]$-linear combinations of $1,\dots, {p}_{i}(x,y)/d_{i} (x)$. \todo{wirklich richtig?}
%\end{remark}

In \cite{intbas} we presented an algorithm to compute integral bases computing the local contributions to the normalization at each branch of the curve at the singularities, and merging the contributions using the Chinese remainder theorem. This local approach is usually fast when the number of branches at each singularity is small, but it can be computationally
slow when there are many branches, since the cofactors required for the Chinese remainder force the computations to be developed up to a higher order. Moreover, summing up the result may lead to generators which are not in triangular shape, and computing the triangular shape can be computationally expensive.

In this paper we present a modification of the algorithm in \cite{intbas} which uses a combinatorial procedure to obtain an integral basis at the origin
handling all the different conjugacy classes at once, and directly return the generators in triangular shape.

Using the techniques presented in \cite{intbas}, computing a global integral basis can be reduced the problem of computing polynomial generators for the normalization of the localization of $A$ at $x = 0$, assuming that the origin is the only singularity of the localization.

%For simplicity we assume in this paper that $f$ has only one singularity and that this singularity is located at the origin.


Given $f \in K[X,Y]$, we denote by $\tilde f \in K[[X]][Y]$ the product of the branches of $f$ at the origin. Considering the decomposition $f = f_0 \tilde f$ given by the Weirstrass preparation theorem, where $f_0 \in K[[X]][Y]$ is a unit in $K[[X,Y]]$, we have shown in \cite[Proposition 41]{intbas} how to easily obtain an integral basis for $f$ from an integral basis from $\tilde f$. Hence, we will focus in this paper on computing the integral basis for $\tilde f$.

\subsection{Valuations}

We recall some useful valuation formulas. We denote $L\{\{X\}\}$ the field of Puiseux series over $L$. A nonzero $f \in L\{\{X\}\}$ can be written uniquely as
$$
f = \sum_{k = k_0}^\infty c_k X^{k / n}
$$
with $c_k \in L$ for all $k \ge k_0$, $c_{k_0} \neq 0$ and $n \in \N$. We define $\upsilon_X(f) = \frac{k_0}{n}$, the valuation of $f$.
The corresponding \emph{valuation ring}
$L\{\{X\}\}_{\upsilon\geq0}=\bigcup_{k=1}^{\infty}L[[X^{1/k}]]$ consists of all Puiseux series
with non-negative exponents only. Henceforth it will be denoted by ${\mathcal{P}_{X}}$.


\begin{defn}[Valuation of a polynomial at a Puiseux expansion]
If $q\in L\{\{X\}\}[Y]$ is any polynomial in $Y$ with coefficients in
$L\{\{X\}\}$, the {\emph{valuation}} of $q$ at $\gamma\in L\{\{X\}\}$ is
defined to be
$$\upsilon_{\gamma}(q)=\upsilon_X(q(\gamma)).$$
\end{defn}

By the properties of valuations, we obtain
$$
\upsilon_{\gamma}(pq)= \upsilon_{\gamma}(p) + \upsilon_{\gamma}(q).
$$




\begin{defn}[Valuation of a polynomial at another polynomial]
Let $\Gamma=\{\gamma_{1},\dots,\gamma_{m}\}$ be the set of Puiseux
expansions of a polynomial $g \in L\{\{X\}\}[Y]$. The {\emph{valuation}} of a polynomial $q\in L\{\{X\}\}[Y]$
at $g$ is defined to be
$$\upsilon_{g}(q)=\min_{1 \leq i \leq m} \upsilon_{\gamma_{i}}(q),$$
which we also denote by $\upsilon_{\Gamma}(q)$.
\end{defn}

From the definitions, we obtain the following formulae.

\begin{lemma}
\label{lemma:gammaAtq}
Let $\gamma \in L\{\{X\}\}$ and let $q \in L\{\{X\}\}[Y]$ be a monic polynomial of degree $d\geq 1$ in $Y$.  %where $1\leq d \leq m-1$,
If $q=(Y-\eta_{1}(X))\cdots(Y-\eta_{d}(X))$ is the factorization of $q$ in $L\{\{X\}\}[Y]$, then
\[
\upsilon_{\gamma}(q)= \sum_{j=1}^{d}\upsilon_X(\gamma-\eta_{j}).
\]

For a polynomial $g \in L\{\{X\}\}[Y]$ with Puiseux expansions $\{\gamma_{1},\dots,\gamma_{m}\}$,
\[
\upsilon_{g}(q)=\min_{1 \leq i \leq m}\sum_{j=1}^{d}\upsilon_X(\gamma_{i}-\eta_{j})\text{.}%
\]
\end{lemma}

\subsection{Polynomials with maximal valuation}
We recall two results from \cite{intbas} that are central for our combinatorial approach.

The first lemma says that if we look for a polynomial $p \in {\mathcal{P}_{X}}[Y]$ with maximal valuation at $g$, then we can always take a polynomial $p$ whose Puiseux expansions are a subset of the expansions of $g$.

\begin{lemma}[{\cite[Lemma 21]{intbas}}]
\label{lemma:intA} Let $g\in K[[X]][Y]$ be a square-free monic polynomial of degree
$m\geq 1$ in $Y$, with Puiseux expansions $\gamma_{1}, \dots, \gamma_{m}$. Fix
an integer $d$ with $1\leq d \leq m-1$. If $\mathcal{A}\subset\{1,\dots,m\}$ is
a subset of cardinality $d$, set
\[
\Int({\mathcal{A}})=\min_{i\not \in \mathcal{A}}\left(  \sum_{j\in\mathcal{A}%
}\upsilon_X(\gamma_{i}-\gamma_{j})\right).
\]
Choose a subset $\widetilde{\mathcal{A}}\subset\{1,\dots,m\}$ of cardinality
$d$ such that $\Int({\widetilde{\mathcal{A}}})$ is maximal among all
$\Int({\mathcal{A}})$ as above, and set $\widetilde{p}_d=\prod_{j\in
\widetilde{\mathcal{A}}}(Y-\gamma_{j})\in{\mathcal{P}_{X}}[Y]$. Then
$\upsilon_{g}(\widetilde{p}_d)=\Int({\widetilde{\mathcal{A}}})$, and this number is the
maximal valuation $\upsilon_{g}(q)$, for
$q\in L\{\{X\}\}[Y]$ monic of degree $d$ in $Y$.
\end{lemma}

For $d = m - 1$, we call $E(g):=\lfloor \upsilon_{g}(\widetilde{p}_d) \rfloor$ the maximal integrality exponent with respect to $g$. It is the maximum exponent of the denominators in an integral basis of $g$.

For our combinatorial approach, it will be easier to work in the ring $\mathcal{P}_{X}[Y]$ and once we determine which is an optimal subset of expansions for each degree, we construct a polynomial in $K[X][Y]$ using the following lemma, for which we recall also the proof since it gives a constructive way to go from $\mathcal{P}_{X}[Y]$  to $K[X][Y]$.

\begin{lemma}
\label{same exp} Suppose $g\in K[[X]][Y]$ is a square-free monic polynomial of degree
$m\geq 1$ in $Y$. Let $1\leq d \leq m-1$ be an integer, and denote by $R$ any of the rings $K[X]$,
$K[X]_{\left\langle X\right\rangle }$, $K[[X]]$, $K((X))$, ${\mathcal{P}_{X}}$, or $L\{\{X\}\}$.
For a polynomial $q\in R[Y]$ which is monic of degree $d$ in $Y$, the maximal valuation $\upsilon_{g}(q)$ is
independent of the choice of $R$ from the above list.
\end{lemma}
\begin{proof}
For any $R$ as in the statement, there are natural inclusions $K[X]\subset R \subset L\{\{X\}\}$.
The valuation $\upsilon_{g}(q)$  is thus defined for any $q\in R[Y]$ and it is sufficient to prove that
there is a polynomial $p_d\in K[X][Y]$ which maximizes the valuation $\upsilon_{g}(p)$ among all $p$ in $L\{\{X\}\}[Y]$
of degree $d$. To see this, recall from Lemma \ref{lemma:intA} that there exist 
$\widetilde{p}_d=\prod_{j\in\widetilde{\mathcal{A}}}(Y-\gamma_{j}) \in \mathcal{P}_{X}[Y]$
which maximizes the valuation over $L\{\{X\}\}$ in degree $d$. There exists an
integer $k$ such that $\widetilde{p}_d\in L[[X^{1/k}]][Y]$. Truncating every 
$\gamma_{j}$ to degree $\upsilon_{g}(\widetilde{p}_d)$, we obtain a polynomial $\overline
{p}_d = \prod_{j\in\widetilde{\mathcal{A}}}(Y-\overline{\gamma}_{j}) \in L[X^{1/k}][Y]$ with the same valuation as $\widetilde{p}_d$. 
Applying the trace map for
$L(X^{1/k})$ over $L(X)$ to $\overline{p}_d$ to the $Y$-monic polynomial $\overline{p}_d$,  and dividing by the integer leading coefficient of the trace, we obtain a monic polynomial $p_d^\prime\in L[X][Y]$ of degree $d$ in $Y$ with $\upsilon_{g}({p_{d}^\prime})
\geq  \upsilon_{g}(\widetilde{p}_d)$ \footnote{Note that the trace map sends $X^{1/k}$ to zero.}
Considering $p_d^\prime$ as a polynomial in $X$ and $Y$ with coefficients in $L$ and
adjoining these coefficients to $K$, we obtain an extension $K \subset K^\prime$ of finite fields such that
$p_d^\prime$ is in $K^\prime[X][Y]$. We then apply the trace map of the extension $K \subset K^\prime$ to $p_d^\prime$ and
divide the result by its leading coefficient. This gives a monic polynomial
$p_d\in K[X][Y]$ of degree $d$ in $Y$ satisfying $\upsilon_{g}({p_d}) \geq  \upsilon_{g}(\widetilde{p}_d)$.
Note that Lemma \ref{lemma:intA} and the choice of $\widetilde{p}_d$, the inequality is in fact an equality since
$\widetilde{p}_d$ maximizes the valuation over $L\{\{X\}\}$.
%Note that $K[X]$ is included in all the rings $R$ in the above list, hence the reverse inequalities are trivial, and the result follows.
\end{proof}

\section{One Puiseux block}

Let $\Gamma \subset \Px$ be the set of all Puiseux expansions of $f$. The Puiseux blocks of $f$ are a partition of the Puiseux expansions of $f$ such that in each set the first non--rational term of every expansion is identical or conjugated. We assume first that $f$ has only one Puiseux block.

To compute an integral basis of $f$, we compute for each $0 \le d < \deg(f)$ a monic polynomial $p \in K[X][Y]$ of degree $d$ with maximal valuation at $f$ among all monic polynomials of degree $d$.

Our strategy is to compute a factorization of $p$. If $\eta$ is a Puiseux expansion of $p$ and $\{\eta_1, \dots, \eta_s\}$ is the conjugacy class of $\eta$ for the extension $K[X][Y] \hookrightarrow \Px[Y]$, then $q = \prod_{i = 1}^s (Y - \eta_i)$ is a factor of $p$.
By Lemma \ref{lemma:intA}, we can assume that any expansion $\eta$ of $p$ is a truncation of an expansion $\gamma$ of $f$. Moreover, we can assume that there exists $\gamma \in \Gamma$ such that $\eta = \overline{\gamma}^{<t}$ for $t$ an extended characteristic exponent of $\gamma$ or $\eta = \overline{\gamma}^{\le N}$, for $N$ the integrality exponent of $f$.

%The key argument for the algorithm is that although the order of $\gamma - \eta$ may be different for different expansions in the conjugacy classes, the valuation of $\gamma$ at a polynomial $p \in K[X,Y]$ is always the same for all expansions $\gamma$ in the same conjugacy class.
%
%Hence, instead of computing the polynomial $\tilde p \in \Px$ of maximal valuation by determining the number of expansions in each conjugacy class appearing in $\tilde p$, we will consider the factorization of a polynomial $p$ over the ground field and determine which factors appear in $p$ and to which power.

Following \cite[Algorithm 6]{intbas}, let $\Delta=\left\{  \delta_{1},\ldots,\delta_{m}\right\}$ be
the set of Puiseux expansions in a conjugacy class of $f$. In Algorithm \ref{algo:factors} we provide a procedure to compute all possible factors of $p$ coming from this class of expansions.

\begin{algorithm}[h]                      % enter the algorithm environment
\caption{\texttt{PolynomialFactors}}          % give the algorithm a caption
\label{algo:factors}
\begin{algorithmic}[1]
\REQUIRE $\Delta=\left\{  \delta_{1},\ldots,\delta_{m}\right\}$ the set of Puiseux expansions at the origin in a conjugacy class of $f$, developed up to the integrality exponent $N := E(f)$ of $f$.
\ENSURE A set $Q \subset K[X][Y]$ of all the possible factors of an integral basis element coming from the conjugacy class $\Delta$.
\STATE Let $\{t_1, \dots, t_s\}$ be the extended characteristic exponents of the expansions.
\FORALL {$t \in \{t_1, \dots, t_s\}$}
\STATE Let $\rho_{1},\ldots, \rho_{\overline{m}}$ be the pairwise different elements in $\left\{  \overline{\delta}_{1}^{<t},\ldots,\overline{\delta}_{m}^{<t}\right\}$.
\STATE Set
$$
q_t:=\prod\nolimits_{i=1}^{\overline{m}}(Y-\rho_{i}(X)).
$$
\ENDFOR
\STATE \label{algo:step:qN} For $N$ the integrality exponent of $f$, set $\overline{f_{\Delta}}^{\le N} := \prod\nolimits_{i=1}^{m}(Y-\overline{\delta_{i}}^{\le N}(X))$.
\RETURN $Q = \{q_{t_1}, \dots, q_{t_s}, \overline{f_{\Delta}}^{\le N}\}$
\end{algorithmic}
\end{algorithm}

\begin{example}
\label{example:factors}
Let $f = (Y^4 + 2X^3Y^2 + 2X^5Y + X^6 + 1/4X^7) + Y^5 \in{\mathbb{Q}}[X,Y]$.
The Puiseux expansions of $g$ are
\begin{equation*}
\begin{aligned}[c]
\gamma_{1} &=IX^{3/2} +(-1/2I-1/2)X^{7/4} + \dots,\\
\gamma_{2} &=IX^{3/2} +(1/2I+1/2)X^{7/4} + \dots,\\
\gamma_{3} &=-IX^{3/2}+(1/2I-1/2)X^{7/4} + \dots,\\
\gamma_{4} &=-IX^{3/2}+(-1/2I+1/2)X^{7/4} + \dots,\\
\gamma_{5} &=-1 + \dots
\end{aligned}
\end{equation*}
where $I$ is a root of $Z^{2}+1$.

There is only one class of expansions at the origin, $\Delta = \{\gamma_1, \gamma_2, \gamma_3, \gamma_4\}$. The characteristic exponents are $3/2$ and $7/4$. The integrality exponent is 4. Applying Algorithm 1, we obtain the following polynomials:
\begin{equation*}
\begin{aligned}[c]
q_{3/2} &= Y,\\
q_{7/2} &= (Y-IX^{3/2})(Y+IX^{3/2})) = Y^2 + X^3,\\
\overline{f_{\Delta}}^{\le 4} &= Y^4 + 2X^3Y^2 + 2X^5Y + X^6 + 1/4X^7.%
\end{aligned}
\end{equation*}
\end{example}

\vspace{.5cm}

Computing these polynomials for all the conjugacy classes of $f$ we get all possible factors of $p$.
The next step is to determine the multiplicity of these factors in the factorization of $p$ so that the resulting polynomial has the desired degree and maximal valuation. We do this by exhaustive search among all possible combinations.
The key argument for our algorithm in this case is that the valuation of $\gamma \in \Gamma$ at a polynomial $q \in K[X][Y]$ is always the same for all expansions $\gamma$ in the same conjugacy class.
We obtain Algorithm \ref{algo:oneBlock}.


\begin{algorithm}[h]                      % enter the algorithm environment
\caption{\texttt{IntegralElementOneBlock}}          % give the algorithm a caption
\label{algo:oneBlock}
\begin{algorithmic}[1]
\REQUIRE $\Delta_1, \dots, \Delta_s$ the conjugacy classes of Puiseux expansions at the origin of a monic polynomial $f \in K[X][Y]$, developed up to the maximum integrality exponent of $f$; a non-negative integer $d$, $0 \leq d \leq n = \deg_Y(\tilde f)$.
\ENSURE a polynomial $p \in K[X][Y]$ of degree $d$ of maximal valuation at the set of expansions $\Delta = \Delta_1 \cup \dots \cup \Delta_s$ among all monic polynomials of degree $d$; $o \in \Q_{\geq 0}$, the valuation of $p$ at $f$.
\STATE For each $1 \le i \le s$, let $P_i = \texttt{PolynomialFactors}(\Delta_i)$, the polynomials factors corresponding to $\Delta_i$.
\STATE Consider the set $\{p_1, \dots, p_m\} = \cup_{i=1}^s P_i \subset K[X][Y]$ of all the polynomials obtained from all the conjugacy classes, and let $d_1, \dots, d_m$ be the corresponding degrees.
\STATE Define $C = \{(c_1, \dots, c_m) \in \Z_{\ge 0}^m : c_1 d_1  + \dots c_m d_m = d\}$, the set of all possible $m$-tuples.
\STATE For each $\bm{c} \in C$, compute the valuation of $p_{\bm c} = p_1^{c_1}\cdots p_m^{c_m}$ at $\tilde f$ by the second formula in Lemma \ref{lemma:gammaAtq}.
\RETURN $(p, \upsilon_{\tilde f}(p))$, for $p$ the polynomial with maximal valuation at $\tilde f$ among all the polynomials computed.
\end{algorithmic}
\end{algorithm}


We have seen in Lemma \ref{same exp} that the maximal valuation over monic polynomials in $K[X][Y]$ of a given degree $d$ is the same as the maximal valuation over polynomials in $\Px[Y]$ of degree $d$. Hence Algorithm \ref{algo:oneBlock} provides an effective way to compute this valuation, which we call $o(g,d)$ or $o(\Gamma, d)$ for $\Gamma$ the set of Puiseux expansions of $g$.

\begin{example}
\label{example:oneBlock}
Let $f = (Y^2+X^3)(Y^4 + 2X^3Y^2 + 2X^5Y + X^6 + 1/4X^7) + Y^7 \in{\mathbb{Q}}[X,Y]$.
The Puiseux expansions of $f$ are
\begin{equation*}
\begin{aligned}[c]
\gamma_{1} &=IX^{3/2} +(-1/2I-1/2)X^{7/4} + \dots,\\
\gamma_{2} &=IX^{3/2} +(1/2I+1/2)X^{7/4} + \dots,\\
\gamma_{3} &=-IX^{3/2}+(1/2I-1/2)X^{7/4} + \dots,\\
\gamma_{4} &=-IX^{3/2}+(-1/2I+1/2)X^{7/4} + \dots,\\
\gamma_{5} &=IX^{3/2}+ 1/4 I X^{5/2} + \dots,\\
\gamma_{6} &=-IX^{3/2} - 1/4 I X^{5/2} +  \dots,\\
\gamma_{7} &=-1 + \dots.
\end{aligned}
\end{equation*}

The integrality exponent of $f$ is 8. There are two classes of Puiseux expansions: $\Delta_1 = \{\gamma_1, \gamma_2, \gamma_3, \gamma_4\}$ and $\Delta_2 = \{\gamma_5, \gamma_6\}$, and both classes are in the same Puiseux block. Applying Algorithm \ref{algo:factors} to $\Delta_1$ we obtain the factors $\{Y, Y^2 + X^3, \overline{f_{\Delta_1}}^{\le 8}\}$. Applying Algorithm \ref{algo:factors} to $\Delta_2$ we obtain the factors $\{Y, \overline{f_{\Delta_2}}^{\le 8}\}$. Now we apply Algorithm \ref{algo:oneBlock} for every $0 \le d \le 6$. We obtain the following elements:
$$
\begin{aligned}[c]
p_0 &= 1, \upsilon_{\tilde f}(p_0) = 0 \\
p_1 &= Y, \upsilon_{\tilde f}(p_1) = 3/2 \\
p_2 &= Y^2+X^3, \upsilon_{\tilde f}(p_2) = 3/2+7/4 = 13/4\\
p_3 &= Y(Y^2+X^3), \upsilon_{\tilde f}(p_3) = 3/2+3/2+7/4 = 19/4 \\
%p_4 &= Y^4 + 2X^3Y^2 + 2X^5Y + X^6 + 1/4X^7, \upsilon_f(p_4) = 3/2+3/2+7/4+7/4=13/2 \\
p_4 &= \overline{f_{\Delta_1}}^{\le 8}, \upsilon_{\tilde f}(p_4) = 13/2 \\
p_5 &= Y\cdot\overline{f_{\Delta_1}}^{\le 8}, \upsilon_{\tilde f}(p_5) = 13/2+3/2=8
\end{aligned}
$$
\end{example}

\section{Direct approach}
\label{section:directApproach}

% In the direct approach we use Puiseux blocks. In the combinatorial approach we will use Puiseux segments.

We consider now the case of a polynomial $f \in K[X][Y]$ whose Puiseux expansions at the origin are grouped into several Puiseux blocks.
%We will take first a theoretical approach, working over the Puiseux field, and we will explain later how to turn it into an effective algorithm.
%The approach we will take is to consider how many expansions in each Puiseux block.
Let $\Gamma$ be the set of all Puiseux expansions of $f$ and let $\Pi_1, \dots, \Pi_s$ be the Puiseux blocks of $f$.
For each Puiseux block $\Pi_{i}$, let $f_i$ be the corresponding
factor of $f$ in $K[[X]][Y]$ (that is, $f_i = \prod_{\gamma \in \Pi_i} (Y - \gamma)$). Let $m_i$ be the cardinal of $\Pi_i$ (and hence also the $Y$-degree of $f_i$).

We address first the (theoretical) problem of finding for each $0 \le d < n$ a polynomial $p_d \in \Px[Y]$ of maximal valuation at $\tilde f$ among all polynomials of degree $d$.
We know that we can take the expansions of $p_d$ as a subset of the expansions of $\tilde f$, hence we can factorize
$$
p_d = p_{(1)} \cdot \dots \cdot p_{(s)},
$$
where $p_{(i)} \in \mathcal{P}_{X}[Y]$, $1 \le i \le s$, is a polynomial whose Puiseux expansions are a subset of the expansions of $\Pi_i$.

Note that although $\upsilon_{\gamma}(pq) = \upsilon_{\gamma}(p) + \upsilon_{\gamma}(q)$ for a single Puiseux expansion $\gamma$, it is not true in general that $\upsilon_{g}(pq) = \upsilon_{g}(p) + \upsilon_{g}(q)$ for a polynomial $g$, so even if we fix the degrees $c_1, \dots, c_s$ of the polynomials $p_{(1)}, \dots, p_{(s)}$, we cannot directly split the problem into smaller problems, one for each branch. In \cite{intbas} we used the Chinese remainder theorem to merge the integral bases for the branches. In this section we will compute the polynomials $p_d$, $0 \le d < n$ by exhaustive search over all possible tuples of degrees $(c_1, \dots, c_s)$. In the next section we will show how to optimize the strategy using a combinatorial approach.

We recall the valuation formula from Lemma \ref{lemma:gammaAtq}. For $q \in \mathcal{P}_{X}[Y]$ of degree $d$ with Puiseux expansions $\{\eta_1, \dots, \eta_d\}$,
$$
\upsilon_{\gamma}(q)= \sum_{j=1}^{d}\upsilon_X(\gamma-\eta_{j}).
$$
By the definition of Puiseux blocks, we deduce that if $\gamma \in \Gamma$ is not in $\Pi_{j}$ then $\upsilon_{\gamma} (p_{(j)})$ only depends on the degree $c_j$ of $p_{(j)}$ and not on the specific expansions of $p_{(j)}$.
Since $\upsilon_X(\gamma - \eta)$ is the same for any $\gamma \in\Pi_{i}$ and $\eta \in \Pi_j$, $i \ne j$, we note $\upsilon_{ij}$ this value. We obtain the following formulae.

\begin{lemma}
\label{formula:blocks}
Let $\Pi_1, \dots, \Pi_s$ be the Puiseux blocks of a polynomial $f \in K[X][Y]$. Let $p_{(1)}, \dots, p_{(s)} \in \mathcal{P}_{X}[Y]$ be monic polynomials of degree $c_1, \dots, c_s$ respectively such that for all $1 \le i \le s$, the Puiseux expansions of $p_{(i)}$ are a subset of the expansions in $\Pi_i$. Then, if $\gamma \in \Pi_i$ and $\eta \in \Pi_j$,
$$
\upsilon_{\gamma}(p_{(j)}) = c_j \upsilon_{ij} \quad \text{ and } \quad \upsilon_{\eta}(p_{(i)}) = c_i \upsilon_{ij}.
$$

For $p = p_{(1)} \cdots p_{(s)}$ and any $\gamma \in \Pi_i$,
\[
\boxed{
\upsilon_{\gamma}(p) = \left(\textstyle \sum_{j \ne i} c_j \upsilon_{ij}\right) + \upsilon_{\gamma}(p_{(i)})}
\]
\end{lemma}

For $p = p_{(1)} \cdots p_{(s)}$ as in the lemma, we call $\bm{c} = (c_1, \dots, c_s)$ the multiplicity of $p$ with respect to the sets $\Pi_1, \dots, \Pi_s$.
As observed before, only $\upsilon_{\gamma}(p_{(i)})$ depends on the actual Puiseux expansions of $p$ and not on the number of them in each block.

For any $0 \le k < m_i := \#\Pi_i$, we note $\vardbtilde p_{(i, k)}$ the polynomial in $\Px[Y]$ of degree $c_i$ in $Y$ of maximal valuation at $f_i$, whose Puiseux expansions are a subset of the expansions of $f_i$.
%, which can be computed using Algorithm \ref{algo:oneBlock} for any $1 \le i \le s$ and $0 \le k \le m_i$
By the observation above, if we fix the degrees $c_1, \dots, c_s$ of the polynomials
$$p_{(1)}, \dots, p_{(s)},$$
then the best choice for $p = p_{(1)}\cdots p_{(s)}$ is to take $p_{(i)} := \vardbtilde  p_{(i, c_i)}$.

For $\bm{c} = (c_1, \dots, c_s)$ ($0 \le c_i \le m_i$), we define
$$
\vardbtilde p_{\bm{c}} = \vardbtilde p_{(1, c_1)} \dots \vardbtilde p_{(s, c_s)},
$$
a polynomial with maximal valuation at $f$ among all polynomials with multiplicity $(c_1, \dots, c_s)$.

%for $\gamma \in \Pi_i$, $\upsilon_{\gamma}(p_{(i,c_i)}) = o(\Pi_i, c_i)$, which we have defined at the end of the previous section.

Hence for determining the polynomial $p \in {\mathcal{P}_{X}}[Y]$ of degree $d$ of maximal valuation at $f$ among all monic polynomials of degree $d$ it is enough to consider all tuples $\bm{c}=(c_1, \dots, c_s)$ such that $c_1 + \dots + c_s = d$, compute for each of these tuples the valuation at $\tilde f$ of the polynomial $\vardbtilde p_{\bm{c}} = \prod_{i=1}^s \vardbtilde p_{(i, c_i)}$ and take the one with maximal valuation.

The polynomials $\vardbtilde p_{(i, c_i)}$ cannot be effectively computed because they involve infinite series. We note $\tilde p_{(i, c_i)}$ the polynomial in $K[X][Y]$ of degree $c_i$ in $Y$ of maximal valuation at $f_i$, which can be computed using Algorithm \ref{algo:oneBlock}. The formula $$\upsilon_{\gamma}(\tilde p_{(j, c_j)}) = c_j \upsilon_{ij}$$
still holds for $\gamma \in \Pi_i$, $i \neq j$, because the truncations in the expansions in $\tilde p_{(j, c_j)}$ only occur at degrees equal or higher than the first extended characteristic exponent.

We conclude  that
$$
\tilde p_{\bm{c}} = \tilde p_{(1, c_1)} \dots \tilde p_{(s, c_s)},
$$
is a polynomial in $K[X][Y]$ with maximal valuation at $\tilde f$ among all polynomials with multiplicity $(c_1, \dots, c_s)$. Using these polynomials, we obtain Algorithm \ref{algo:exhaustive} to compute effectively the elements of an integral basis of $\tilde f$.


\begin{algorithm}[h]                      % enter the algorithm environment
\caption{\texttt{ExhaustiveSearch}}          % give the algorithm a caption
\label{algo:exhaustive}
\begin{algorithmic}[1]
\REQUIRE $\Pi_1, \dots, \Pi_s$ the Puiseux blocks of expansions at the origin of a polynomial $f \in K[X,Y]$ monic in $Y$; $0 \leq d < n = \deg_Y(\tilde f)$.
\ENSURE $p \in {\mathcal{P}_{X}}[Y]$ of $Y$--degree $d$ of
maximal valuation at $\tilde f$; $o \in \Q_{\geq 0}$, the valuation of $p$ at $\tilde f$.
\STATE $m_i = \#\Pi_i$ for $i = 1, \dots, s$, the number of expansions in each Puiseux block
\STATE $C_d = \{(c_1, \dots, c_s) \in \Z_{\ge 0}^s : 0 \leq c_i \leq m_i \mbox{, } c_1 + \dots + c_s = d\}$
\FORALL {$\bm{c} = (c_1, \dots, c_s) \in C_d$}
\FOR {$i = 1, \dots, s$}
\STATE $\tilde p_{(i, c_i)} = \texttt{IntegralElementOneBlock}(\Pi_i, c_i)$.
\ENDFOR
\STATE $p_{\bm{c}} = \tilde p_{(1,c_{1})}\cdots \tilde p_{(s, c_{s})}$
\STATE $\upsilon_{\tilde f}(p_{\bm{c}})=\min_{1 \leq i \leq s}\left\{\left(\sum_{j \ne i} c_j \upsilon_{ij}\right) + \upsilon_{\Pi_i}(\tilde p_{(i, c_i)})\right\}$.
\ENDFOR
\STATE $p^{*} = p_{\bm{c}}$ for $\bm{c} \in C_d$ such that $\upsilon_{\tilde f}(p_{\bm{c}})$ is maximal
\RETURN $(p^{*}, \upsilon_{\tilde f}(p^*))$.
\end{algorithmic}
\end{algorithm}

\begin{example}
Let $f = (Y^3-X^2)(Y^2+X^3)(Y^4 + 2X^3Y^2 + 2X^5Y + X^6 + 1/4X^7) + Y^{10} \in{\mathbb{Q}}[X,Y]$.
The Puiseux expansions of $f$ are
$$
\begin{aligned}[c]
\gamma_{1} &=IX^{3/2} +(-1/2I-1/2)X^{7/4} + \dots,\\
\gamma_{2} &=IX^{3/2} +(1/2I+1/2)X^{7/4} + \dots,\\
\gamma_{3} &=-IX^{3/2}+(1/2I-1/2)X^{7/4} + \dots,\\
\gamma_{4} &=-IX^{3/2}+(-1/2I+1/2)X^{7/4} + \dots,\\
\gamma_{5} &=IX^{3/2}+ 1/4 I X^{5/2} + \dots,\\
\gamma_{6} &=-IX^{3/2} - 1/4 I X^{5/2} +  \dots,\\
\gamma_{7} &=\alpha_1 X^{2/3} - 1/3 \alpha_1 X^{4/3} + \dots,\\
\gamma_{8} &=\alpha_2 X^{2/3} - 1/3 \alpha_2 X^{4/3} + \dots,\\
\gamma_{9} &=\alpha_3 X^{2/3} - 1/3 \alpha_3 X^{4/3} + \dots,\\
\gamma_{10} &=-1 + \dots,
\end{aligned}
$$
where $\alpha_1, \alpha_2, \alpha_3$ are the roots of $Z^3-1$.

The integrality exponent of $f$ is 10. There are 3 classes of expansions $\Delta_1 = \{\gamma_1, \gamma_2, \gamma_3, \gamma_4\}$, $\Delta_2 = \{\gamma_5, \gamma_6\}$ and $\Delta_3 = \{\gamma_7, \gamma_8, \gamma_9\}$, and 2 blocks $\Pi_1 = \Delta_3$ and $\Pi_2 = \Delta_1 \cup \Delta_2$. By similar computations as in Example \ref{example:oneBlock}, applying Algorithm \ref{algo:oneBlock} to $\Pi_2$ we obtain the same elements and valuations as in that example. For $\Pi_1$, applying Algorithm \ref{algo:oneBlock}, we obtain the elements $1, Y, Y^2, \overline{f_{\Delta_3}}^{\le 10}$ whose valuations at $\Pi_1$ are $0, 2/3, 4/3, 10$ respectively.

Now we apply Algorithm \ref{algo:exhaustive} to combine the two blocks. For example, for $d = 5$ we test all combinations of degrees $(c_1, c_2)$ with $c_1 + c_2 = 5$ and $c_1 \le 3$. We obtain that the element with highest valuation at the origin is achieved for $\bm{c} = (3,2)$. The corresponding polynomial is $p_{\bm{c}} = \overline{f_{\Delta_3}}^{\le 10} \cdot (Y^2 + X^3)$ and the valuation at $f_0$ is $\frac{13}{4} + 3\frac{2}{3} = \frac{21}{4}$.
\end{example}


%Note that although this algorithm deals with elements of $\mathcal{P}_{X}[Y]$ that are infinite series and cannot be computed, if we work instead with the singular parts of the expansions, we can easily adapt the algorithm to compute which set of expansions gives the maximal valuation for each degree $0 \leq k \leq d = \deg_Y(f)$ and the valuations at $f$ of the corresponding polynomials.

\section{Combinatorial approach}

%
%We assume first that we can compute $\tilde p_{(i, c_i)}$ for any $1 \le i \le s$ and $0 \le c_i \le m_i$. Note that if the Puiseux segment $\Gamma_i$ consists of only one Puiseux block, we can apply Algorithm \ref{algo:oneBlock} to compute $\tilde p_{(i, c_i)}$.
%If the Puiseux segment contains different Puiseux blocks, we will show later how to apply a recursive strategy to compute $\tilde p_{(i, c_i)}$.
%
%%for $\gamma \in \Pi_i$, $\upsilon_{\gamma}(p_{(i,c_i)}) = o(\Pi_i, c_i)$, which we have defined at the end of the previous section.
%
%Hence for computing the polynomial $p_d \in {\mathcal{P}_{X}}[Y]$ of maximal valuation at $f$ among all monic polynomials of degree $d$ it is enough to consider all tuples $(c_1, \dots, c_s)$ such that $c_1 + \dots + c_s = d$, test for each of these tuples the valuation at $f$ of the polynomial $\prod_{i=1}^s \tilde p_{(i, c_i)}$ and take the one with maximal valuation.
%

%\todo{not true if same rational terms, we have to use puiseux segments}
Let $f \in K[X][Y]$ and let $\Pi_1, \dots, \Pi_s$ be the Puiseux blocks of expansions $f$ at the origin, with cardinalities $m_1, \dots, m_s$, as before. To apply Algorithm \ref{algo:exhaustive} for computing the element of the integral basis of $f$ of degree $d$ we must run over all tuples in $C_{d} = \{(c_1, \dots, c_s) \in \Z_{\ge 0}^s : 0 \leq c_i \leq m_i \mbox{, } c_1 + \dots + c_s = d\}$.
This can be very slow when $f$ has a large number of Puiseux classes, since the number of tuples to test grows exponentially with the number of Puiseux blocks.
We explain in this section how to find the optimal $(c_{1}, \dots, c_{s}) \in C_{d}$ in an
efficient way. Instead of considering all tuples of $s$ elements, we will always
consider ordered pairs and proceed iteratively.

For this approach we group the Puiseux classes in sets by the initial term. All the Puiseux classes with the same (or conjugated) initial term are grouped in the same set. Let $\Gamma_1, \dots, \Gamma_s$ be the resulting sets of Puiseux expansions of $f$, and let $f_1, \dots, f_s$ be the corresponding polynomials (this classification is similar to Puiseux segments defined in \cite[Section 7.2]{intbas} except that if two classes have the same initial exponent but the corresponding coefficients are different, they are grouped in different sets).
%Recall that the expansions in each Puiseux segment $\Gamma_i$ have the same initial exponent (they correspond to the same face of the Newton polygon).
We assume that the sets $\Gamma_1, \dots, \Gamma_s$ are ordered by the initial exponent increasingly (the order between sets with the same initial exponent is not important).

\subsection{Theoretical approach}
As before, for each $0 \le d < n$, we look first for a polynomial $p_d \in \Px[Y]$ with maximal valuation at $\tilde f$. We now that any such polynomial can be factorized as
$$
p_d = p_{(1)} \cdot \dots \cdot p_{(s)},
$$
where $p_{(i)} \in \Px[Y]$, $1 \le i \le s$, is a polynomial whose Puiseux expansions are a subset of the expansions of $\Gamma_i$.

The key
property for the combinatorial approach is that if $1 \leq i < j \leq s$, then for any $\gamma \in\Gamma_{i}$ and $\eta \in \Gamma_{j}$,
$$\upsilon_X(\gamma - \eta) = \upsilon_X(\gamma),$$
because the initial term of $\gamma$ has smaller or equal degree than the initial term of $\eta$ (and if they have the same degree, they have different coefficients). We define $\upsilon_i = \upsilon_X(\gamma)$, for any $\gamma \in \Gamma_i$.

We obtain the following formulae (compare with Lemma \ref{formula:blocks}).
\begin{lemma}
\label{formula:sets}
Let $\Gamma_1, \dots, \Gamma_s$ be the sets of Puiseux expansions of a polynomial $f \in K[X][Y]$, and assume that for $i < j$ the valuation of the expansions in $\Gamma_i$ is smaller or equal than the valuation of the expansions in $\Gamma_j$. Let $p_{(1)}, \dots, p_{(s)} \in \mathcal{P}_{X}[Y]$ be polynomials of degree $c_1, \dots, c_s$ respectively such that for all $1 \le i \le s$, the Puiseux expansions of $p_{(i)}$ are a subset of $\Gamma_i$. Then, for $i < j$, if $\gamma \in \Gamma_i$ and $\eta \in \Gamma_j$,
$$
\upsilon_{\gamma}(p_{(j)}) = c_j \upsilon_X(\gamma) = c_j \upsilon_i \quad \text{ and } \quad \upsilon_{\eta}(p_{(i)}) = c_i \upsilon_X(\gamma) = c_i \upsilon_i.
$$

For $p = p_{(1)} \cdots p_{(s)}$ and any $\gamma \in \Gamma_i$,
\[
\boxed{
\upsilon_{\gamma}(p) = \left(\textstyle \sum_{j < i} c_j \upsilon_j\right) + \upsilon_{\gamma}(p_{(i)}) + \left(\textstyle  \sum_{j > i} c_j \upsilon_i \right).}
\]
\end{lemma}

As in Lemma \ref{formula:blocks}, only $\upsilon_{\gamma}(p_{(i)})$ depends on the actual Puiseux expansions of $p$ and not only on the number of expansions in each set. Hence, for fixed multiplicities $\bm{c} = (c_1, \dots, c_s)$, a polynomial with
maximal valuation at $f$ is
$$\vardbtilde p_{\bm c} := \prod_{i = 1}^s \vardbtilde p_{(i, c_i)},$$
where $\vardbtilde p_{(i, k)}$ is the polynomial in $\mathcal{P}_{X}[Y]$ of degree $k$ in $Y$ of maximal valuation at $\Gamma_i$, whose expansions are a subset of the expansions of $f_i$.

For our combinatorial approach, we define $\Theta_{i} := \Gamma_{i} \cup \dots \cup \Gamma_{s}$, $1 \le i \le s$.
For any subset $N_{1}$ of $c_{1}$ expansions of $\Gamma_{i}$ and any
subset $N_{2}$ of $c_{2}$ expansions of $\Theta_{i+1}$, if we define
$$
q_{1} = \prod_{\gamma\in N_{1}}(Y - \gamma), \quad
q_{2} = \prod_{\eta\in N_{2}}(Y - \eta), \quad
\text{ and } q = q_{1}q_{2},$$
we have
\[
\upsilon_{\gamma}(q_{2}) = c_{2} \upsilon_i \ \text{ and } \ \upsilon_{\eta}(q_{1}) = c_{1} \upsilon_i,
\]
for any $\gamma \in \Gamma_i$ and $\eta \in \Theta_{i+1}$, by the formulae we obtained before.

Since $\upsilon_{{\Gamma_{i}}}(q_{1})$ is the minimum of $\upsilon_{\gamma_{i}}%
(q_{1})$ for $\gamma_{i} \in\Gamma_{i}$, we obtain that
\[
\min_{\gamma\in\Gamma_{i}} \upsilon_{\gamma}(q) = \upsilon_{f_{\Gamma_{i}}}(q_{1}) +
c_{2} \upsilon_i.
\]

Similarly,
\[
\min_{\eta\in\Theta_{i+1}} \upsilon_{\eta}(q) = c_{1} \upsilon_i +
\upsilon_{{\Theta_{i+1}}}(q_{2}).
\]

We obtain the following formula.
\begin{lemma}
\label{formula:pairs}
For $q = q_1 q_2$ as above,
\[\boxed{
\upsilon_{{\Theta_{i}}}(q) = \min\{\upsilon_{{\Gamma_{i}}}(q_{1}) + c_{2} \upsilon_i,
c_{1} \upsilon_i + \upsilon_{{\Theta_{i+1}}}(q_{2})\}.}
\]
\end{lemma}

\begin{remark}
\label{remark:pairs}
In this formula, only $\upsilon_{{\Gamma_{i}}}(q_{1})$ and $\upsilon_{{\Theta_{i+1}}}(q_{2})$ depend on the actual expansions and not only on the number of expansions. Hence, if we fix the degrees $c_1, c_2$ of $q_1, q_2$ respectively, we can split the problem of computing the polynomial $q$ with maximal valuation at $f_{\Theta_{i}}$ into the two smaller problems of computing the polynomial $q_1$ with maximal valuation at $f_{\Gamma_{i}}$ and the polynomial $q_2$ with maximal valuation at ${\Theta_{i+1}}$.
\end{remark}

\subsection{Effective algorithm}
We will use Remark \ref{remark:pairs} to determine, for $0 \leq c \leq m_{i} +
\dots+ m_{s}$, the polynomial $p_{\Theta_{i}}(c)$ in $K[X][Y]$ of $Y$-degree $c$ with maximal valuation at $f_i \cdots f_s$, by decreasing induction on $i$, starting with $i = s$.

As with the formulae in the previous sections, Lemma \ref{formula:pairs} is still valid if we replace the polynomials $q_1, q_2 \in \Px[Y]$ with polynomials $\bar q_1, \bar q_2 \in K[X][Y]$ whose Puiseux expansions are truncations of the expansions in $q_1, q_2$ at degrees equal or higher than the first extended characeteristic exponents.

For each $1 \le i \le s$ and $1 \le c_i \le m_i$, we define $p_{\Gamma_{i}}(c_i) := \tilde p_{(i, c_i)} \in K[X][Y]$ (as defined in Section \ref{section:directApproach}).
We can compute $\tilde p_{(i, c_i)}$ as before by exhaustive search using Algorithm \ref{algo:exhaustive} or, if the Puiseux set contains several Puiseux blocks, we can apply recursively the combinatorial approach  we develop now, as we will see below.

As the first step, we set $p_{\Theta_{s}}(c) = p_{\Gamma_{s}}(c)$ for $0 \leq c \leq m_{s}$.
Proceeding inductively, once we have determined $p_{\Theta{i+1}}(c)$ for
all $0 \leq c \leq m_{i+1} + \dots+ m_{s}$, we want to compute
$p_{\Theta_{i}}(c)$ for all $0 \leq c \leq m_{i} + \dots+ m_{s}$.

Using Lemma \ref{formula:pairs} and Remark \ref{remark:pairs} we can compute inductively
\[
o(\Theta_{i}, c) = \max_{\substack{c_{1} + c_{2} = c \\ c_{1} \le m_i}} \upsilon_{{\Theta_{i}}%
}(p_{\Gamma_{i}}(c_{1})p_{\Theta_{i+1}}(c_{2}))
\]
and define $p_{\Theta_{i}}(c)$ as the polynomial for which the maximum is obtained.
We obtain Algorithm \ref{algo:iterative}.

%We show now how to choose, for a given $0 \le d < n$ the vector $\bm c = (c_1, \dots, c_s)$ with $|\bm c|=d$ such that $\tilde p_{\bm c}$ has maximal valuation among all polynomials of degree $d$.

%Defining
%$$\Theta_{i} := \Gamma_{i} \cup \dots \cup \Gamma_{s},$$
%we will
%compute $p_{\Theta_{i}}(c)$, for $0 \leq c \leq m_{i} +
%\dots+ m_{s}$, the polynomial of degree $c$ with maximal valuation at $f_i \cdots f_s$, by decreasing induction on $i$, starting with $i = s$.


%The numerator of the element of degree $d$ in the integral basis is
%\[
%p_{d} = p_{\Theta_{1}}(d)
%\]
%and the denominator is $x^{\lfloor \upsilon_{f}(p_{d}) \rfloor}$.

%For computing the best polynomials in each block when a block has more than one conjugacy class of expansions, we could use this strategy
%recursively, but we do not explain this here.

%As in the previous section, we can replace $\vardbtilde p_{(i, c_i)}$ by $\tilde p_{(i, c_i)}$ and the formulae in Lemma \ref{formula:sets} are still valid. That is,
%$$\tilde p_{\bm c} := \prod_{i = 1}^s \tilde p_{(i, c_i)}$$
%is a polynomial in $K[X][Y]$ with maximal valuation at $f$ among all monic polynomials with multiplicity $\bm{c}$ (hence $\upsilon_f(\tilde p_{\bm c}) = \upsilon_f(\vardbtilde p_{\bm c})$).


\begin{algorithm}[h]                      % enter the algorithm environment
\caption{\texttt{IntegralBasisIterative}}          % give the algorithm a caption
\label{algo:iterative}
\begin{algorithmic}[1]
\REQUIRE $\Gamma_1, \dots, \Gamma_s$ the sets of Puiseux expansions at the origin of a polynomial $f \in K[X,Y]$ monic in $Y$ of degree $n$, ordered increasingly by the order of the expansions, developed up to the integrality exponent $N$ of $f$; $m_i, 1 \le i \le s$, the cardinal of $\Gamma_i$.
\ENSURE $\{(p_0, o_0), \dots, (p_{n}, o_{n})\}$ such that $p_d \in K[X][Y]$ has $Y$--degree $d$ and
maximal valuation at $f$ among all polynomials of $Y$-degree $d$ and $o_d \in \Q_{\geq 0}$, $o_d = \upsilon_f(p_d)$.
\IF{$s = 1$}
\RETURN $\{(\tilde p_{(1, c)}, \upsilon_{f}(\tilde p_{(1, c)}) = \texttt{ExhaustiveSearch}(\Gamma_1, c))\}_{c= 0, \dots, n}$ % (Algorithm \ref{algo:exhaustive})
\ELSE
%\STATE $\Theta_s := \Gamma_s$, $f_{\Theta_s} := f_{\Gamma_s}$
\STATE $\Theta_s = \Gamma_s$
\STATE $\{(p_{(\Theta_s,c)}, o(\Theta_s, c))\}_{c = 0, \dots, m_s} = \IntegralBasisIterative(\Theta_s)$
%\STATE $(p_{\Theta_s}(m_s), o(\Theta_s, m_s)) = q_N$ as defined in Algorithm \ref{algo:factors}.
\FOR{$i = s-1, \dots, 1$}
\STATE $\Theta_i = \Gamma_i \cup \Theta_{i+1}$, $f_{\Theta_i} = f_{\Gamma_i} f_{\Theta_{i+1}}$
\STATE $\{(p_{(\Gamma_i,c)}, o(\Gamma_i, c))\}_{c = 0, \dots, m_i} = \IntegralBasisIterative(\Gamma_i)$
\FOR{$0 \leq d \leq m_i + \dots + m_s$}
\STATE $C_d = \{(c_1, c_2) \in \Z_{\ge 0}^2 \mid c_1 + c_2 = d\mbox{, }0 \leq c_1 \leq m_i\mbox{, }0 \leq c_2 \leq m_{i+1} + \dots + m_s\}$
\STATE $o(\Theta_{i}, d) = \max_{(c_1, c_2) \in C_d} \upsilon_{{\Theta_{i}}%
}(p_{\Gamma_{i}}(c_{1})p_{\Theta_{i+1}}(c_{2}))$
\STATE $p_{(\Theta_i,d)} = $ the polynomial for which the maximum is obtained
\ENDFOR
\ENDFOR
\RETURN $\{(p_{(\Theta_1,d)}, o(\Theta_1, d))\}_{0 \le d \le n}$.
\ENDIF
\end{algorithmic}
\end{algorithm}

We note that with this approach the number of cases to test grows linearly with the number of conjugacy classes, which is much more efficient than the previous approach with exponential growth.

We apply the algorithm to an example.

\begin{example}
Let $f = (Y^3-X^2)(Y^4 + 2X^3Y^2 + 2X^5Y + X^6 + 1/4X^7)(Y^2-X^5) + Y^{10} \in{\mathbb{Q}}[X,Y]$.
The Puiseux expansions of $f$ are
$$
\begin{aligned}[c]
\gamma_{1} &=\alpha_1 X^{2/3} - 1/3 \alpha_1 X^{4/3} + \dots,\\
\gamma_{2} &=\alpha_2 X^{2/3} - 1/3 \alpha_2 X^{4/3} + \dots,\\
\gamma_{3} &=\alpha_3 X^{2/3} - 1/3 \alpha_3 X^{4/3} + \dots,\\
\gamma_{4} &=IX^{3/2} +(-1/2I-1/2)X^{7/4} + \dots,\\
\gamma_{5} &=IX^{3/2} +(1/2I+1/2)X^{7/4} + \dots,\\
\gamma_{6} &=-IX^{3/2}+(1/2I-1/2)X^{7/4} + \dots,\\
\gamma_{7} &=-IX^{3/2}+(-1/2I+1/2)X^{7/4} + \dots,\\
\gamma_{8} &=X^{5/2}+ 1/2  X^{29/2} + \dots,\\
\gamma_{9} &=-X^{5/2} - 1/2  X^{29/2} +  \dots,\\
\gamma_{10} &=-1 + \dots,
\end{aligned}
$$
where $\alpha_{i}$, $1 \le i \le 3$, are the roots of $Z^{3}-1$.

There are 3 classes of Puiseux expansions at the origin, and each class is a different set: $\Gamma_1 = \{\gamma_1, \gamma_2, \gamma_3\}$, $\Gamma_2 = \{\gamma_4, \gamma_5, \gamma_6, \gamma_7\}$ and $\Gamma_3 = \{\gamma_8, \gamma_9\}$. We have $m=9$, $m_{1} = 3$, $m_2 = 4$  and $m_{3} = 2$, and the integrality exponent is 10.

Hence $\Theta_{3} = \Gamma_{3}$ and
the elements of the local integral basis corresponding to $\Theta_{3}$ are
$$\{(1, 0), (Y, 5/2), (\overline{f_{\Gamma_3}}^{\le 10}, 10)\}.$$

For $i = 2$, we set $\Theta_{2} = \Gamma_2 \cup \Gamma_3$. The
elements of the local integral basis corresponding to $\Gamma_{2}$ are
$$\{(1, 0), (Y,
1), (Y^2+X^3, 13/4), (Y\cdot(Y^2 + X^3), 19/4), (\overline{f_{\Gamma_2}}^{\le 10}, 10)\}.$$
Now we compute the elements with maximal valuation at $\Theta_2$, testing for each $0 \le d \le 6$ all ordered pairs $(c_1, c_2)$ such that $d = c_1 + c_2$. We obtain the following polynomials:
$$
\begin{aligned}
p_0 &= 1, \upsilon_{\Theta_2}(p_0) = 0, \\
p_1 &= Y, \upsilon_{\Theta_2}(p_1) = 3/2, \\
p_2 &= Y^2, \upsilon_{\Theta_2}(p_2) = 3, \\
p_3 &= Y\cdot(Y^2-X^3), \upsilon_{\Theta_2}(p_3) = 7/4+3/2+3/2 = 19/4, \\
p_4 &= (Y^2-X^3)\overline{f_{\Gamma_3}}^{\le 10}, \upsilon_{\Theta_2}(p_4) = 7/4+3/2+3/2+3/2 = 25/4,\\
p_5 &= Y\cdot\overline{f_{\Gamma_2}}^{\le 10}, \upsilon_{\Theta_2}(p_5) = 3/2+3/2+3/2+3/2+5/2 = 17/2, \\
p_6 &= \overline{f_{\Gamma_2}}^{\le 10}\overline{f_{\Gamma_3}}^{\le 10}, \upsilon_{\Theta_2}(p_6) = 10.
\end{aligned}
$$

Finally we consider $\Theta_1 = \Gamma_1 \cup \Theta_2$ and for each $0 \le d \le 9$ we consider all tuples $(c_1, c_2)$ with $d = c_1 + c_2$ and $c_1 \le 3$. For example, for $d = 4$, we consider the tuples
$$
\begin{aligned}
p_{(0,4)} &= (Y^2-X^3)\overline{f_{\Gamma_3}}^{\le 10}, \upsilon_{\Theta_1}(p_{(0,4)}) = 4 \cdot 2/3 = 8/3 \\
p_{(1,3)} &= Y\cdot Y\cdot(Y^2-X^3), \upsilon_{\Theta_1}(p_{(1,3)}) = 4 \cdot 2/3 = 8/3 \\
p_{(2,2)} &= Y^2\cdot Y^2, \upsilon_{\Theta_1}(p_{(1,3)}) = 4 \cdot 2/3 = 8/3 \\
p_{(3,1)} &= \overline{f_{\Gamma_1}}^{\le 10} \cdot Y, \upsilon_{\Theta_1}(p_{(1,3)}) = 3 \cdot 2/3 + 3/2 =  7/2.
\end{aligned}
$$

The best element for $d= 4$ is then $p_4 = \overline{f_{\Gamma_3}}^{\le 10} \cdot Y$, with integrality exponent $\lfloor 7/2 \rfloor = 3$.
\end{example}


%\begin{example}
%We apply Algorithm \ref{algo:combinatorial} to the latter example.
%We compute for $\Theta_2$: $\tilde p_{(2,1)} = Y$, $\tilde p_{(2,2)} = Y^2$, $\tilde p_{(2,3)} = f_2$ developed up to the integrality exponent. Similarly, for $\Gamma_1$: $\tilde p_{(1,1)} = Y$, $\tilde p_{(2,2)} = f_1$ developed up to the integrality exponent.
%\end{example}

%\section{The nodal locus}
%In this section, we explain how to compute the integral basis of the ideal localized at the nodal locus, without decomposing it into its singular points.

%\subsection{Polynomials over the ground field}
%As we mentioned before, if we apply these algorithms using the singular part of the Puiseux expansions, we can compute effectively the best set of expansions for the polynomial of each degree with maximal valuation at $f$.
%
%The last step is to compute, for each $c < n$ a polynomial in $K[X][Y]$ with the same valuation as the one given by the previous algorithm.
%We can in principle apply the steps in the proof of Lemma \ref{same exp} to construct these polynomials. When a Puiseux set $\Gamma_i$ contains only one conjugacy class of expansions, it is easier to use $\TruncatedFactor$ algorithm from \cite{intbas} which computes these elements as products of polynomials of low degree coming from truncating the Puiseux expansions at different orders.

%\section{The case of several conjugacy classes of expansions in the same Puiseux Block}
%\label{section:inblock}

\subsection{Expansions with common rational part}

We consider now the case of a set $\Gamma_i$ of expansions containing more than one Puiseux block. In this case, the expansions have the same initial term, which is rational. Hence, we can subtract from all the expansions in the set the common rational part. After removing the common rational part, the expansions will not be all in the same set, and we can apply Algorithm \ref{algo:iterative} to the resulting sets. To keep this presentation simple, we give an example of this case, but we do not introduce the corresponding modifications in Algorithm \ref{algo:iterative}.


\begin{example}
Consider the polynomial $f=((Y-X)^2-X^3)((Y-X)^2-X^5)+(Y-X)^5$ with Puiseux expansions at the origin
\begin{multicols}{2}\noindent
\begin{align*}
\gamma_1 &= X + X^{3/2} - 1/2 X^3 + \dots \\
\gamma_2 &= X - X^{3/2} - 1/2 X^3 + \dots \\
\gamma_3 &= X + X^{5/2} + 1/2 X^7 + \dots \\
\gamma_4 &= X - X^{5/2} + 1/2 X^7 + \dots
\end{align*}
\end{multicols}
\noindent where $\{\gamma_1, \gamma_2\}$ is a conjugacy class and $\{\gamma_3, \gamma_4\}$ is another conjugacy class, and both classes are in the same set.

All the expansion have $X$ as common rational part. After removing this common part, we obtain
\begin{multicols}{2}\noindent
\begin{align*}
\eta_1 &= X^{3/2} + X^2 + \dots \\
\eta_2 &= -X^{3/2} + X^2 + \dots \\
\eta_3 &= X^{5/2} + X^3 + \dots \\
\eta_4 &= -X^{5/2} + X^3 + \dots
\end{align*}
\end{multicols}

Now, $N_1 = \{\eta_1, \eta_2\}$ is a set of expansion and $N_2 = \{\eta_3, \eta_4\}$ is another set, so we can apply Algorithm \ref{algo:iterative} for $\Theta = N_1 \cup N_2$.

The integrality exponent is 5. We obtain the following elements:
\begin{align*}
p_0 &= 1, \upsilon_{\Theta_1}(p_0) = 0, \\
p_1 &= Y, \upsilon_{\Theta_1}(p_1) = 3/2, \\
p_2 &= Y^2, \upsilon_{\Theta_1}(p_2) = 3, \\
p_3 &= Y \overline{f_{N_1}}^{\le 5}, \upsilon_{\Theta_1}(p_0) = 3 + 5/2 = 11/2.
\end{align*}

Replacing $Y$ by $Y - X$ in these polynomials, we obtain the elements of maximal valuation at $f$.
\end{example}


\section{Timings}

In this section we measure timings in some examples. We generate examples with several branches.

\begin{enumerate}
% 1 second vs 7 seconds
\item \label{example:5branchesAdeg30}$f = (y^4+3x^3y + x^4)(y^7 + 6x^4y^3 + 2xy + x^7)(y^5+7xy-4x^2)(y^3+x^2)(y^2-x^3)+y^{30}$
\item \label{example:5branchesAdeg100}$f = (y^4+3x^3y + x^4)(y^7 + 6x^4y^3 + 2xy + x^7)(y^5+7xy-4x^2)(y^3+x^2)(y^2-x^3)+y^{100}$
% 1 second vs 7 seconds
\item \label{example:5branchesB} $f = (y^4+3x^3y + x^4)(y^7 + 6x^4y^3 + 2xy + x^7)(y^9+7xy2-4x^2)(y^3+x^2)(y^2-x^3)+y^{30}$
% 1 second vs 198 seconds
\item \label{example:6branchesAdeg30} $f = (y^4+ x^4)(y^7 + 2xy + x^2)(y^5+7x^3)(y^3+x^2)(y^3-x^2)(y^2-x^3)+y^{30}$
\item \label{example:6branchesAdeg100} $f = (y^4+ x^4)(y^7 + 2xy + x^2)(y^5+7x^3)(y^3+x^2)(y^3-x^2)(y^2-x^3)+y^{100}$
% 2 second vs. 238 seconds
\item \label{example:6branchesBdeg30} $f = (y^4+3x^3y + x^4)(y^7 + 6x^4y^3 + 2xy + x^7)(y^5+7x^3)(y^3-4x^2)(y^3+x^2)(y^2-x^3)+y^{30}$
\item \label{example:6branchesBdeg100} $f = (y^4+3x^3y + x^4)(y^7 + 6x^4y^3 + 2xy + x^7)(y^5+7x^3)(y^3-4x^2)(y^3+x^2)(y^2-x^3)+y^{100}$

\end{enumerate}

\[%
\begin{tabular}
[c]{|r|r|r|r|r|r|}\hline
No. & Branches & $Y$-degree & Combinatorial & Chinese remainder & Maple          \\ \hline
(\ref{example:5branchesAdeg30})   & 5 & $30$  & $1$   & $7$   & $3$       \\ \hline
(\ref{example:5branchesAdeg100})  & 5 & $100$ & $1$   & $1$   & $45$      \\ \hline
(\ref{example:5branchesB})        & 5 & $30$  & $1$   & $7$   & $4$       \\ \hline
(\ref{example:6branchesAdeg30})   & 6 & $30$  & $1$   & $198$ & $2.4$     \\ \hline
(\ref{example:6branchesAdeg100})  & 6 & $100$ & $1$   & $2$   & $41.1$    \\ \hline
(\ref{example:6branchesBdeg30})   & 6 & $30$  & $2$   & $238$ & $3.5$     \\ \hline
(\ref{example:6branchesBdeg100})  & 6 & $100$ & $2$   & $2$   & $64.7$    \\ \hline
\end{tabular}
\
\]

We observe that the combinatorial approach presented in this work is always fast for these examples. The approach merging the local contributions to the integral basis using the Chinese Remainder theorem is slow when the polynomial has low degree. Surprisingly, it is very fast when the polynomial has large degree. The reason seems to be that in the case of large degree the Groebner basis computations are faster because the large degree monomials are separated from the low degree ones. When compared with Maple, we observe that our approach is always faster than Maple. The computations in Maple are very sensible to the degree of the polynomial.

\bibliographystyle{plain}
\bibliography{mybib}



\end{document}


%\begin{remark}
%We always can find an integral basis of type
%\[
%1,\frac{p_{1}}{x^{e_{1}}},\dots,\frac{p_{n-1}}{x^{e_{n-1}}}.
%\]
%\todo{fix this: if only singularities with $x = 0$ occur. Move to a later section.}
%\end{remark}

\begin{remark}
Let $L=\overline{K}$ and let $A=L[x,y]=L[X,Y]/\left\langle f\right\rangle $ be
denote the coordinate ring of $C$ with $f\in K[X][Y]$ monic in $Y$ of degree
$n$. Let $a_{1},...,a_{s}\in L$ denote the $x$-coordinates of the
singularities of $C$. Then there are $p_{i}\in K[X][Y]$ with $\deg_{Y}%
(p_{i})=i$ for all $i$ and $e_{i,j}\in\mathbb{N}_{0}$ with $e_{i,j}\geq
e_{i-1,j}$, such that, as an $L[X]$-module%
\[
\bar{A}=_{L[X]}\left\langle \frac{\bar{p}_{0}}{\bar{q}_{0}},\ldots,\frac
{\bar{p}_{n-1}}{\bar{q}_{n-1}}\right\rangle \subseteq
Q(A)=L(X)[Y]/\left\langle f\right\rangle
\]
with%
\[
q_{i}=%
%TCIMACRO{\dprod \limits_{j=1}^{s}}%
%BeginExpansion
{\displaystyle\prod\limits_{j=1}^{s}}
%EndExpansion
(X-a_{j})^{e_{i,j}}\in K[X]\text{.}%
\]


If $a_{1}=0$ and $P=\left\langle x,y\right\rangle $ then, as an
$L[X]_{\left\langle X\right\rangle }$-module,%
\[
\overline{A_{P}}=\left\langle \frac{\bar{p}_{0}}{x^{e_{0,1}}},\ldots
,\frac{\bar{p}_{n-1}}{x^{e_{n-1,1}}}\right\rangle
\]
and%
\[
\overline{\widehat{A_{P}}}=\left\langle \frac{\bar{p}_{0}}{x^{e_{0,1}}}%
,\ldots,\frac{\bar{p}_{n-1}}{x^{e_{n-1,1}}}\right\rangle \subseteq
Q(A_{P})\otimes_{A_{P}}\widehat{A_{P}}%
\]
as an $L[[X]]$-module.
\end{remark}

\begin{proof}
By van Hoeij's algorithm the claim for $\bar{A}$ is clear. The second claim
follows since $\_\otimes_{A}A_{P}$ is right exact, $\overline{A_{P}}%
=\overline{A}_{P}$, and $x-a$ is a unit in $\overline{A_{P}}$ if $a\neq0$. The
third claim follows since $\_\otimes_{A_{P}}\widehat{A_{P}}$ is right exact
and $\widehat{\overline{A_{P}}}=\overline{\widehat{A_{P}}}$, since the
semilocal ring $\overline{A_{P}}$ is excellent.
\end{proof}




%% Puiseux blocks with more than one class

\begin{remark}
As we have already mentioned, when $f$ has more than one conjugacy class of expansions in a Puiseux block, Algorithm \ref{algo:iterative} cannot be applied to determine which expansions to choose in each class.
The reason why this algorithm does not work is that if $\Gamma$ and $H$ are two conjugacy classes in the same Puiseux block, the order of $\gamma - \eta$, $\gamma \in \Gamma$ and $\eta \in H$, depends on which expansion we have chosen in each class.
\end{remark}

\begin{example}
Consider a polynomial $f$ with Puiseux expansions
\begin{align*}
\gamma_1 &= x^{3/2} + x^2 + \dots \\
\gamma_2 &= - x^{3/2} + x^2 + \dots \\
\gamma_3 &= x^{3/2} + x^3 + \dots \\
\gamma_4 &= -x^{3/2} + x^3 + \dots
\end{align*}
where $\{\gamma_1, \gamma_2\}$ is a conjugacy class and $\{\gamma_3, \gamma_4\}$ is another conjugacy class, and both classes are in the same block.

Then
$$\gamma_1 - \gamma_3 = x^2 - x^3 + \dots$$
has order 2 but
$$\gamma_1 - \gamma_4 = 2 x^{3/2} + x^2 - x^3 + \dots$$
has order 3/2.
\end{example}

\begin{remark}
For this case we propose a different algorithm using polynomials over the ground field as building blocks.
%, instead of considering the number of Puiseux expansions as before.
This algorithm can be applied in all cases, and in practice it is very similar to Algorithm \ref{algo:exhaustive}. However, with this approach we cannot apply an optimized strategy as in Algorithm \ref{algo:iterative} and hence it can be slow when there are many conjugacy classes.
\end{remark}

