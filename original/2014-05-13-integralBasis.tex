%\usepackage[draft]{hyperref}
%\newcommand{\PE}{{\mathcal P}(x)}
%\newcommand{\PEP}{{\mathcal P}[x]}
%\usepackage{epsfig}
%\usepackage{pstricks}
%\usepackage{pst-node}
%\usepackage{pst-tree}
%\usepackage{pst-grad}
%\usepackage{pst-plot}
%\DeclareMathOperator{\IntegralElementSimple}{IntegralElementSimple}
%\DeclareMathOperator{\LocalIntegralBasisElement}{LocalIntegralBasisElement}

\documentclass[a4paper,11pt]{amsart}%
\usepackage{multicol}
\usepackage[utf8x]{inputenc}
\usepackage{amssymb}
\usepackage{amsmath}
\usepackage{synttree}
\usepackage{amsthm}
\usepackage{algorithm}
\usepackage[noend]{algorithmic}
\usepackage{todonotes}
\usepackage{amsfonts}
\usepackage{graphicx}
\usepackage{comment}
\usepackage{hyperref}%
\usepackage{enumitem}
\setcounter{MaxMatrixCols}{30}
%TCIDATA{OutputFilter=latex2.dll}
%TCIDATA{Version=5.00.0.2570}
%TCIDATA{LastRevised=Wednesday, February 26, 2014 20:49:07}
%TCIDATA{<META NAME="GraphicsSave" CONTENT="32">}
%TCIDATA{<META NAME="SaveForMode" CONTENT="1">}
\allowdisplaybreaks
\renewcommand{\algorithmicrequire}{\textbf{Input:}}
\renewcommand{\algorithmicensure}{\textbf{Output:}}
\theoremstyle{definition}
\newtheorem{defn}{Definition}[section]
\theoremstyle{plain}
\newtheorem{theorem}[defn]{Theorem}
\newtheorem{proposition}[defn]{Proposition}
\newtheorem{lemma}[defn]{Lemma}
\theoremstyle{remark}
\newtheorem{remark}[defn]{Remark}
\newtheorem{example}[defn]{Example}
\newtheorem{notation}[defn]{Notation}
\DeclareMathOperator{\LocalIntegralBasis}{LocalIntegralBasis}
\DeclareMathOperator{\IntegralElement}{IntegralElements}
\DeclareMathOperator{\TruncatedFactorGeneral}{TruncatedFactorGeneral}
\DeclareMathOperator{\TruncatedFactor}{TruncatedFactor}
\DeclareMathOperator{\Splitting}{Splitting}
\DeclareMathOperator{\BlockSplitting}{BlockSplitting}
\DeclareMathOperator{\SegmentSplitting}{SegmentSplitting}
\DeclareMathOperator{\Spec}{Spec}
\DeclareMathOperator{\Sing}{Sing}
\DeclareMathOperator{\Ann}{Ann}
\DeclareMathOperator{\Int}{Int}
\DeclareMathOperator{\Hom}{Hom}
\DeclareMathOperator{\Id}{Id}
\DeclareMathOperator{\degree}{degree}
\DeclareMathOperator{\rad}{rad}
\DeclareMathOperator{\Ker}{ker}
\DeclareMathOperator{\TQR}{Q}
\newcommand{\singular}{{\sc Singular}}
\newcommand{\maple}{{\sc Maple}}
\newcommand{\cc}{{\mathbf c}}
\newcommand{\Q}{{\mathbb Q}}
\newcommand{\N}{{\mathbb N}}
\newcommand{\R}{{\mathbb R}}
\newcommand{\Z}{{\mathbb Z}}
\newcommand{\C}{{\mathbb C}}
\newcommand{\Px}{{\mathcal{P}_X}}
\begin{document}
\title[Integral bases via localization]{Computing integral bases via localization and Hensel Lifting}
\author{J. Boehm, W. Decker, S. Laplagne, G. Pfister}

\begin{abstract}
We present a new algorithm for finding an integral basis (the normalization)
of an algebraic function field of one variable in characteristic zero. 
Our basic strategy is to reduce from global to local and, then, to 
``very local'': Theoretically, this amounts to localization and, then, completion
at each singularity. Practically, we work with approximations by suitably
truncated Puiseux expansions. In contrast to van Hoeij's algorithm
\cite{vanHoeij94}, which also relies on Puiseux expansions, we use Hensel's
lemma as a key ingredient. This allows us to compute factors corresponding to
groups of conjugate Puiseux expansions, without actually computing the 
individual expansions. In this way, we make substantially less use of the Newton-Puiseux
algorithm. In addition, our algorithm is inherently parallel. As a result, it
outperforms in most cases any other algorithm known to us by far. Typical
applications are the computation of adjoint ideals and, based on this, the
computation of Riemann-Roch spaces and the parametrization of rational curves.

\end{abstract}
\maketitle


\section{Introduction}

\label{sect:intro} Let $A$ be a reduced Noetherian ring, and let $\TQR(A)$ be
its total ring of fractions. The {\emph{normalization}} of $A$ is the integral
closure of $A$ in $\TQR(A)$. We denote the normalization by $\overline{A}$ and
call $A$ {\emph{normal}} if $A=\overline{A}$. Recall that if $A$ is a reduced
affine (that is, finitely generated) algebra over a field, then $\overline{A}$
is a finite $A$-module by Emmy Noether's finiteness theorem (see \cite{SH}).

In this paper, we are interested in the case where $A$ is the coordinate ring
of an algebraic curve defined over a field $K$ of characteristic zero. More
precisely, let $f\in K[X,Y]$ be an irreducible polynomial
%\todo{do we want irreducible or irreducible? See Remark 1.3 and Proposition 3.7.} 
in two variables, let
$C\subset\mathbb{A}^{2}(K)$ be the affine plane curve defined by $f$, and let
\[
A=K[C]=K[X,Y]/\langle f(X,Y)\rangle
\]
be the {\emph{coordinate ring}} of $C$. We write $x$ and $y$ for the residue
classes of $X$ and $Y$ in $A$, respectively. Throughout the paper, we suppose
that $f$ is monic in $Y$ (due to Noether normalization, this can always be
achieved by a linear change of coordinates). Then the {\emph{function field}}
of $C$ is of type
\[
K(C)=\TQR(A)=K(x,y)=K(X)[Y]/\langle f(X,Y)\rangle,
\]
$x$ is a separating transcendence basis of $K(C)$ over $K$, and $y$ is
integral over $K[x]$, with integral equation $f(x,y)=0$. In particular, $A$ is
integral over $K[x]$, which implies that $\overline{A}$ coincides with the
integral closure $\overline{K[x]}$ of $K[x]$ in $K(C)$. We may, hence,
represent $\overline{A}$ either by generators over $A$ or by generators over
$K[x]$. For the latter, note that $\overline{A}=\overline{K[x]}$ is a free
$K[x]$--module of rank
\[
n:=\deg_{y}(f)=[K(C):K(x)].
\]
Indeed, this follows by applying \cite[Theorem 3.3.4]{Stichtenoth08} to the PID 
$$
K[x]\subset K(x)\subset K(C) = K(x)[y].
$$

\begin{defn}
An {\emph{integral basis}} for $\overline{A}$ over $K[x]$ is a set $b_{0}, \dots, b_{n-1}$
of free generators for $\overline{A}$ over $K[x]$:
\[
\overline{A} = K[x] b_{0} \oplus\dots\oplus K[x] b_{n-1}.
\]
\end{defn}

\begin{remark}
\label{rem:spec-int-basis}
There is always an integral basis of type
$$
1, \frac{p_1(x,y)}{d(x)}, \dots, \frac{p_{n-1}(x,y)}{d(x)},
$$
with $d \in K[x]$, and with polynomials $p_i\in K[x][y]$ of degree $i$ in $y$. 
Such a basis is obtained from any 
given set $1=c_0, \dots, c_{m-1}$ of $K[x]$-module generators for $\overline{A}$
by unimodular row operations over the PID $K[x]$: For each $i$, write
$c_{n-1-i}=\sum_{j=0}^{n-1}c_{ij}y^{n-1-j}$, with coefficients $c_{ij}\in K(x)$.
Then take $d$ to be the least common denominator of the $c_{ij}$, 
transform the matrix $(d\cdot c_{ij})$ into row echolon form 
%(Hermite normsl form) 
$(p_{ij})$, and set $p_{n-1-i}=\sum_{j=0}^{n-1}p_{ij}y^{n-1-j}$, 
for each $i$.
\end{remark}



%\begin{remark}
%\label{rem:spec-int-basis} \noindent Viewing the elements of $K(C) = K(x)[y]$ as
% polynomials in $y$ of degree $<n$, with coefficients in $K(x)$, there 
% is an integral basis $b_{0},\dots,b_{n-1}$ such that $\deg_{y}(b_{i})=i$
% for all $i$. In fact, as already pointed out by van Hoeij \cite{vanHoeij94},
% such a basis is obtained from any given set of $K[x]$-module generators for
%$\overline{A}$ by means of elementary matrix operations over the PID $K[x]$.
%\end{remark}


\begin{remark}
\label{rem:spec-int-basis-II} 
To find an integral basis, we may use any normalization algorithm, regardless 
of how $\overline{A}$ is represented by the algorithm. The algorithms in \cite{GLS10}, \cite{BDLPSS},
for example, are designed to find $A$-module generators for $\overline{A}$. More precisely,  they 
return an ideal $U\subset A$ together with an element $d\in A$ such that
$\overline{A}=\frac{1} {d}U\subset\TQR(A)$. Here, as we will recall in Section 
\ref{sect:Int-basis-via-norm}, any non-zero element of the Jacobian ideal $M$ of $A=K[x,y]$
can be taken to be $d$. In particular, we can choose $d$ to be a generator of the elimination 
ideal $M\cap K[x]$. The roots of $d$ in the  algebraic closure $\overline{K}$ of $K$
are then precisely the $x$-coordinates of the singularities of the curve defined by $f$ in 
$\mathbb{A}^{2}(\overline{K})$. If $u_{0}=d(x),u_{1},\dots,u_{r}$ generate the ideal $U$, 
the $y^{i}u_{j}(x,y)/d(x)$, $0\leq i\leq n-1$, $0\leq j\leq r$, 
generate $\overline{A}$ over $K[x]$. An integral basis is then obtained
by operations as described in Remark \ref{rem:spec-int-basis}.
%Turning these generators into an integral basis
%$b_{0}=1, b_1, \dots,b_{n-1}$ as in Remark \ref{rem:spec-int-basis}, we get fractions
%of type
%$$
%b_i=\frac{q_i(x,y)}{d_1^{e_{i,1}}\cdots d_s^{e_{i,r}}},
%$$
%where the $d_j$ are the irreducible factors of $d$ in $K[x]$, and where each 
%$q_i(x,y)\in K(x)[y]$ is a polynomial of degree $i$ in $y$ not divisible by any of the $d_j$. 
%Then $0\leq e_{i,j}\leq e_{i+1,j}$ for all $i,j$. Indeed, to see this, multiply $b_i$ by $y$ and 
%express the result as a $K[x]$-linear combination of the $b_k$. Note that van Hoeij's
%algorithm also computes an integral basis of this type (see \cite{vanHoeij94}).
\end{remark}

\begin{example}
\label{exampleCusp} Consider the standard cusp: Let
\[
A=K[x,y]=K[X,Y]/\langle Y^{3}-X^{2}\rangle.
\]
As a module over $A$, we may represent $\overline{A}$ as
\[
\overline{A}=A \cdot\frac{y^{2}}{x}+A \cdot1 = \frac{1}{x}\left\langle y^{2},
x\right\rangle _{A}%
\]
(see \cite[Example 2.5]{GLS10}). Considering $\overline{A}$ over $K[x]$, we
get
\[
\overline{A}=K[x]\cdot\frac{y^{2}}{x}+K[x]\cdot y\cdot\frac{y^{2}}%
{x}+K[x]\cdot y^{2}\cdot\frac{y^{2}}{x}+K[x]\cdot1+K[x]\cdot y+K[x]\cdot
y^{2}.
\]
Since $y^{3}=x^{2}$ and $K[x]\cdot y^{2}\subset K[x]\cdot y^{2}/x$, we have
\[
\overline{A}=K[x]\cdot\frac{y^{2}}{x} \oplus K[x]\cdot1\oplus K[x]\cdot y.
\]
Hence, $1,y,y^{2}/x$ is an integral basis as in Remark
\ref{rem:spec-int-basis}.
\end{example}

%\noindent
%Since $\overline{K[x]}=\overline{A}$, o
\todo{From here on adjust to writing the algorithm in a better way.}



So integral bases can be found using any normalization algorithm. In Section
\ref{sect:Int-basis-via-norm}, to fix some of our notation and give a first
example, we briefly discuss the normalization algorithm of Greuel et al.
\cite{GLS10}, which is of global nature. In Section \ref{sect:noem-via-loc},
as a first step towards a better performance, we recall the local to global
normalization algorithm of B\"ohm et al. \cite{BDLPSS}, which finds
$\overline{A}$ by computing a local contribution at each prime ideal contained
in the singular locus of $A$, and putting these together. In Section
\ref{sect:norm-via-loc-and-compl-the-theory}, we start describing our new
approach by showing that at least theoretically, we can go one step further.
In fact, taking an analytic point of view, we explain how to obtain the local
contribution at a singularity from the normalization of the completion, which
in turn is obtained by splitting into branches, and finding the normalization
of each branch. In Section \ref{sect:loc-contr-via-Hensel}, which is the heart
of this paper, we show how to carry this out in practical terms, working with
approximations by suitably truncated Puiseux series. This approach is inspired
by van Hoeij's paper \cite{vanHoeij94}, but differs completely from van
Hoeij's algorithm, with Hensel lifting as a crucial new ingredient. We have
implemented our algorithms in the computer algebra system \textsc{Singular}{}.
In Section \ref{sec timings}, we compare the different approaches, relying on
our implementations as well as on implementations of van Hoeij's algorithm in
\textsc{Maple} and \textsc{Magma}, and running various examples coming from
algebraic geometry.

\section{The Global Normalization Algorithm}

\label{sect:Int-basis-via-norm}

We first fix our notation and give some general facts on normalization. For
this, let $A$ be any reduced Noetherian ring. We write
\[
\Spec(A)=\{P\subset A\mid P{\text{ prime ideal}}\}
\]
for the {\emph{spectrum}} of $A$. The {\emph{vanishing locus}} of an ideal $J$
of $A$ in $\Spec(A)$ is the set $V(J)=\{P\in\Spec(A)\mid P\supset J\}$. We
denote by
\[
N(A)=\{P\in\operatorname{Spec}(A)\mid A_{P}\text{ is not normal}\}
\]
the {\emph{non-normal locus}} of $A$, and by
\[
\operatorname{Sing}(A)=\{P\in\operatorname{Spec}(A)\mid A_{P}\text{ is not
regular}\}
\]
the {\emph{singular locus}} of $A$. Then $N(A)\subset\Sing(A)$, with equality
holding if $A$ is the coordinate ring of a curve (see \cite[Theorem 4.4.9]{JP}).

\begin{defn}
The \emph{conductor} of $A$ is
\[
\mathcal{C}_{A}=\Ann_{A}(\overline{A}/A)=\{a\in A\mid a\overline{A}\subset
A\}.
\]

\end{defn}

Note that $\mathcal{C}_{A}$ is the largest ideal of $A$ which is also an ideal
of $\overline{A}$.

To emphasize the role of the conductor, we note:

\begin{lemma}
\label{lemma:role-of-cond} Let $A$ be a reduced Noetherian ring. Then
$N(A)\subset V(\mathcal{C}_{A})$. Furthermore, $\overline{A}$ is a finite
$A$-module iff $\mathcal{C}_{A}$ contains a non-zerodivisor of $A$. In this
case, $N(A)=V(\mathcal{C}_{A})$.
\end{lemma}

Note, however, that the conductor can only be computed a posteriori when
$\overline{A}$ is already known.

\begin{defn}
\label{def:test-ideal} Let $A$ be a reduced Noetherian ring. A {\emph{test
ideal}} for $A$ is a radical ideal $J\subset A$ such that $V(\mathcal{C}%
_{A})\subset V(J)$. A {\emph{test pair}} for $A$ consists of a test ideal $J$
together with a non--zerodivisor $g\in J$ of $A$.
\end{defn}

Test pairs appear in the Grauert and Remmert normality criterion which is
fundamental to algorithmic normalization (see \cite{GR}, \cite[Prop.~3.6.5]{GP}).
The normalization algorithm of de Jong  (see \cite{deJong98}, \cite{DGPJ}) and its 
improvement, the algorithm of Greuel et al. \cite{GLS10}, are based on this criterion. 
Both algorithms apply to any reduced affine algebra $A=K[X_1,\dots,X_n]/I$ over a 
perfect field $K$. By means of primary decomposition, we can reduce to the case 
where $A$ is equidimensional. In this case, since we assume that $K$ is perfect, the 
Jacobian ideal\footnote{The {\emph{Jacobian ideal}} $M$ of $A=K[X_1,\dots,X_n]/I$ is generated by the 
images of the $c\times c$ minors of the Jacobian matrix $\big(\frac{\partial f_{i}}{\partial X_{j}}\big)$, 
where $c$ is the codimension, and $f_{1},\dots,f_{r}$ are generators 
for $I$. By the Jacobian criterion, $V(M) = \Sing(A)$ (see 
\cite[Theorem 16.19]{Eis}).} $M$ of A is non-zero and contained in the conductor $\mathcal{C}_{A}$, 
so that we may choose the radical $J=\sqrt{M}$ together with any
non--zero element $g$ of $M$ as a test pair (see \cite[Lemma 4.1]{GLS10}). The idea of
finding $\overline{A}$ is then to successively enlarge $A$ by finite extensions $A_{i+1}\cong\Hom_{A_{i}}%
(J_{i},J_{i})\cong\frac{1}{g}(gJ_{i}:_{A_{i}}J_{i})\subset\overline{A}%
\subset\TQR(A)$, with $A_{0}=A$ and $J_{i}=\sqrt{JA_{i}}$, until the normality
criterion of Grauert and Remmert  allows us stop. As already pointed out in Remark \ref{rem:spec-int-basis-II}, 
the algorithm of Greuel et al. is designed so that it returns an ideal $U\subset A$ together 
with an element $d\in A$ such that $\overline{A}=\frac{1} {d}U\subset\TQR(A)$.

\begin{remark}
\label{rem:choice-of denominator}
If $M$ is non-zero and contained in $\mathcal{C}_{A}$ as above, then any non-zero element of $M$ is 
valid as a denominator: If $0\neq c \in M$, then $c \cdot \frac{1} {d}U
=: U'$ is an ideal of A, and $\frac{1} {d}U  = \frac{1} {c}U'$.
%In the case where $A = K[x,y]$ is the coordinate ring of a plane curve as in the introduction, 
%$\Sing(A)$ consists of finitely many prime ideals, which correspond 
%to the conjugacy classes of the singular points of $C$. In particular, 
%$M$ contains a non-zero element of $K[x]$, and we can take any such element
%to be the denominator $d$.
\end{remark}


\begin{example}
\label{example:the global algorithm} Let $A$ be the coordinate ring of the
curve $C$ with defining polynomial $f(X,Y)=X^{5} -Y^{2}(Y-1)^{3}$. Then
\[
J:=\left\langle x, y\left(  y-1\right)  \right\rangle _{A}
\]
is the radical of the Jacobian ideal, so that we can take $(J,x)$ as a test
pair. In its first step, the normalization algorithm yields
\[
A_{1}=\frac{1}{x}U_{1}=\frac{1}{x}\left\langle x,y(y-1)^{2}\right\rangle _{A}
\text{.}%
\]
In the next steps, we get
\[
A_{2}=\frac{1}{x^{2}}U_{2}=\frac{1}{x^{2}}\left\langle x^{2}%
,xy(y-1),y(y-1)^{2}\right\rangle _{A}%
\]
and%
\[
A_{3}=\frac{1}{x^{3}}U_{3}=\frac{1}{x^{3}}\left\langle x^{3},x^{2}%
y(y-1),xy(y-1)^{2},y^{2}(y-1)^{2}\right\rangle _{A}\text{.}%
\]
In the final step, we find that $A_{3}$ is normal and, hence, equal to
$\overline{A}$.
\end{example}

\section{Normalization of Curves via Localization}

\label{sect:noem-via-loc}

In this section, we discuss the local to global variant of the normalization 
algorithm presented in \cite{BDLPSS}. Our starting point is Proposition 
\ref{prop:local-to-global} below. In formulating the proposition,
if $P\in\Spec(A)$ and $A\subset A^{\prime}\subset\overline{A}$ is an 
intermediate ring, we write $A^{\prime}_P$ for the localizaton of 
$A^{\prime}$ at $A\setminus P\subset A^{\prime}$.

\begin{proposition}
\label{prop:local-to-global} 
Let A be a reduced Noetherian ring with a finite singular locus 
$\Sing(A)=\{P_{1},\dots,P_{s}\}$. For $i=1,\dots,s$, let an intermediate
ring $A\subset A^{(i)}\subset\overline{A}$ be given such that 
$A^{(i)}_{P_i}=\overline{A _{P_i}}$. Then
\[
\sum_{i=1}^{s}A^{(i)}=\overline{A}.
\]
\end{proposition}

\begin{proof}
This is a special case of \cite[Proposition 3.2]{BDLPSS}.
\end{proof}

\begin{defn}
We call any ring $A^{(i)}$ as in the proposition a \emph{local contribution}
to $\overline{A}$ at $P_{i}$. If in addition $A^{(i)}_{P_j}={A _{P_j}}$
for $j\neq i$, we speak of a \emph{minimal local contribution} to
$\overline{A}$ at $P_{i}$.
\end{defn}

\begin{remark}
The algorithms discussed in this paper will return minimal local
contributions. Note that such a contribution is uniquely determined since, by
definition, its localization at each $P\in\operatorname*{Spec}(A)$ is determined.
\end{remark}

%\begin{remark}
%\label{alg1}
Proposition \ref{prop:local-to-global} applies, in particular, if $A$ is the 
coordinate ring of a curve. From now on, 
\[
A=K[C]=K[X,Y]/\langle f(X,Y)\rangle
\]
will always denote the coordinate ring of an irreducible plane curve $C$ as in 
the introduction. Proposition \ref{prop:local-to-global}
allows us, then,  to split the computation of $\overline{A}$
into local tasks at the primes $P_{i}\in\Sing(A)$. One way of
finding the local contributions to $\overline{A}$ at the $P_{i}$ is to apply the local version of the
normalization algorithm discussed in \cite{BDLPSS}. For each $i$, the basic
idea is to use $P_{i}$ together with a suitable element $g_{i}$ of the
Jacobian ideal instead of a test pair as in Definition \ref{def:test-ideal}.

\begin{example}
\label{example:localContribution} Let $A$ be the coordinate ring of the curve
$C$ with defining polynomial $f(X,Y)=X^{5} -Y^{2}(Y-1)^{3}$ from Example
\ref{example:the global algorithm}. Note that $C$ has a double point of type
$A_{4}$ at $(0,0)$ and a triple point of type $E_{8}$ at $(0,1)$. If we apply
the strategy above, taking $P_{1}=\left\langle x,y\right\rangle _{A}$,
$P_{2}=\left\langle y - 1, x\right\rangle _{A}$ and $g_{1} = g_{2} = x$, we
get local contributions $\frac{1}{d_{i}} U_{i}$, $i =1, 2$. Exactly,
\[%
\begin{tabular}
[c]{lll}%
$d_{1}=x^{2}$ & and & $U_{1}=\left\langle x^{2}, y(y-1)^{3}\right\rangle _{A}%
$,\\
$d_{2}=x^{3}$ & and & $U_{2}=\left\langle x^{3}, x^{2}y^{2}\left(  y-1\right)
, y^{2}\left(  y-1\right)  ^{2}\right\rangle _{A}$.
\end{tabular}
\]
Summing up the local contributions, we get $\bar A = \frac{1}{d}U$ with
$d=x^{3}$ and
\[
U=\left\langle x^{3},\text{ }y(y-1)^{3}x,\text{ }y^{2}\left(  y-1\right)
x^{2} ,\text{ }y^{2}\left(  y-1\right)  ^{2}\right\rangle _{A}\text{.}%
\]
Note that $U$ coincides with the ideal $U_{3}$ computed in Example
\ref{example:localContribution}.
\end{example}

\begin{remark}
In Example \ref{example:localContribution}, the normalization of
the local ring $A_{P_2}$ is $\overline{A_{P_2}}=
\frac{1}{x^3}\langle x^3, x^2(y-1), (y-1)^2\rangle_{A_{P_2}}$. Indeed,
since $y^2$ is a unit in $\overline{A_{P_2}}$, this follows by localizing
$U_2$ at $P_2$. Note, however, that $(y-1)/x$ and $(y-1)^2/x^3$ 
are not integral over $A$. Hence, $\frac{1}{x^3}\langle x^3, x^2(y-1), 
(y-1)^2\rangle_{A}$ is not a local contribution to $A$ at $P_2$.
%Note that a local contribution to the normalization of a ring $A$ at 
%prime ideal $P$ is not the same as the normalization of the local 
%ring $A_P$, since the elements of the local contribution must be 
%integral over $A$. In Example \ref{example:localContribution}, 
%the normalization of $A_{P_2}$ can be written as 
%$\frac{1}{y^3}\langle y^3, y^2(x-1), (x-1)^2\rangle_{A_{P_2}}$, 
%but $(x-1)/y$ and $(x-1)^2/y^3$ are not integral over $A$. 
%The units in $A_P$ must be taken into account when computing local contributions.
\end{remark}

In what follows, we describe a way of finding the local contributions which is
custom-made for the case of plane curves.

\section{Normalization of Plane Curves via Localization and Completion: 
Decomposing into Branches}
\label{sect:norm-via-loc-and-compl-the-theory}



In this section, we focus on the case where the origin is a singularity of 
$C$, that is, we suppose that $P=\langle x,y \rangle \in \Sing(A)$. Our
goal is to reduce the problem of finding the minimal local contribution 
to $\overline{A}$ at $P$ to the problem of finding an integral basis at 
each branch of the singularity. At the same 
time, we indicate one possible approach to finding the integral bases
at the branches. In the next section, which is the heart of this paper,
we will carry out the details of this approach.

Focusing on the singularity at the origin means to consider $f$ as
an element of the formal power series ring $K[[X,Y]]=K[[X]][Y]$. Then $f$ is
regular of order $n$ in $Y$. By the Weierstrass preparation theorem,
we may write $f$ as a product $f = u \cdot \widetilde{f}$, where $u$ is a unit
in $K[[X,Y]]$, and $\widetilde{f}$ is a Weierstrass polynomial (see, for 
example, \cite{JP}). Decomposing $\widetilde{f}$ into its irreducible factors, we 
obtain a factorization of type
$$
f = u \prod_{i=1}^{r} g_{i},
$$
with irreducible Weierstrass polynomials $g_i$. Then, passing from the
local ring $A_P$ to its completion and normalizing, we get
$$
\overline{\widehat{A_P}} =\overline{K[[X]][Y]/\left\langle f\right\rangle} \cong
\bigoplus_{i=1}^s \overline{K[[X]][Y]/\left\langle g_i\right\rangle}.
$$
We call each ring $K[[X]][Y]/\left\langle g_i\right\rangle$ a \emph{branch} of $A$
at $P$.

\begin{remark}
Let $g\in K[[X,Y]]$ be an irreducible Weierstra\ss \ polynomial of degree $m$
in $Y$. Write $K((x))[y]=K(x,y)=K[[X,Y]]/ \langle g \rangle$. Then the
normalization of $K[[x]][y]$ coincides with the integral closure $\overline{K[[x]]}$ of
$K[[x]]$ in $K((x))[y]$. Applying \cite[Theorem 3.3.4]{Stichtenoth08} to the PID 
$$K[[x]]\subset K((x))\subset K((x))[y],$$ we see that
$\overline{K[[x]][y]}=\overline{K[[x]]}$
is a free $K[[x]]$-module of rank $m$. 
\end{remark}

\begin{defn}
An \emph{integral basis} for $\overline{K[[x]][y]}$ over $K[[x]]$ is a set
of free generators for $\overline{K[[x]][y]}$ over $K[[x]]$.
\end{defn}

\begin{lemma}
\label{lem intbas power series}Let $g\in K[[X]][Y]$ be an irreducible
Weierstra\ss \ polynomial of degree $m$ in $Y$. Then there exist monic
polynomials
$1=p_{0},p_{1},\ldots,p_{m-1}\in K[X][Y]$ in $Y$  of degree $i$ and
$e_{i}\in\mathbb{N}$ such that $p_{0},\frac{p_{1}(x,y)}{x^{e_{1}}},\ldots
,\frac{p_{m-1}(x,y)}{x^{e_{m-1}}}$ is an integral basis for $\overline
{K[[X]][Y]/\left\langle g\right\rangle }$ over $K[[x]]$.
The $e_i$ satisfy $e_1\leq \dots \leq e_{m-1}$.
Moreover, if $q(X,Y)\in K[[X]][Y]$ is monic in $Y$ of degree $1\leq i\leq m-1$,
and $e$ is an integer such that $\frac{q(x,y)}{x^e}$ is integral over $K[[x]]$,
then $e\leq e_i$. In particular, the $e_i$ are uniquely determined.
\end{lemma}

\begin{proof}
Each square matrix with entries in $K[[X]]$ of maximal rank has a uniquely
determined upper triangular Hermite normal form $(p_{ij})$, where the diagonal elements 
are of type $p_{ii}=x^{\nu_i}$, and where the $(p_{ij})$, $j>i$, are polynomials in $K[X]$ of degree 
$<\nu_i$ (see \cite{Durvye84}). Hence, if we start from any integral basis for $\overline{K[[x]][y]}$ over $K[[x]]$
and apply unimodular row operations as in Remark \ref{rem:spec-int-basis}, we get an 
integral basis $b_0=1, b_1, \dots b_{m-1}$, where the $b_i$ are of type $\frac{p_{i}(x,y)}{x^{e_{i}}}$, 
with polynomials $p_i$ which are monic in $y$ of degree $i$, and with integers $e_i$. 
By induction, multiplying $b_i$ by $y$ and expressing the result 
as a $K[[x]]$-linear combination of  the $b_k$, we see that $0\leq e_1\leq \dots \leq e_{m-1}$. 
The last statement of the lemma follows similarly.
\end{proof}

\begin{defn}
\label{defn integrality exponent}
If $q(X,Y)\in K[[X]][Y]$ is monic in $Y$ of degree $1\leq i\leq m-1$,
and $e$ is the maximal integer such that $\frac{q(x,y)}{x^e}$ is
integral over $K[[x]]$, then we call $e$ the \emph{integrality exponent
of $q$ with respect to $g$}, written $e(q)=e$. With notation as in the lemma, 
for each $i$, we call $e_i$ the \emph{maximal integrality exponent with respect to $g$
in degree $i$}. 
\end{defn}

\begin{lemma}
\label{lemma loc int bas shape}
Let $g\in K[[X]][Y]$ be an irreducible Weierstra\ss \ polynomial of 
degree $m$ in $Y$. Let $p_i$ be polynomials which are monic in
$Y$ of degree $i$ and such that $e(p_i)$ is the maximal integrality 
exponent with respect to $g$ in degree $i$. Then $p_{0}=1,\frac{p_{1}}{X^{e_{1}}},\ldots
,\frac{p_{m-1}}{X^{e_{m-1}}}$ is an integral basis for $\overline
{K[[X]][Y]/\left\langle g\right\rangle }$ over $K[[x]]$.
\end{lemma}

\begin{proof}
$A' = \langle 1, \frac{p_1}{x^e(p_1)}, \ldots, \frac{p_{n-1}}{x^{e(p_{n-1}}}\rangle$

$A'_j= \langle 1, \ldots, \frac{p_j}{x^{e(p)}},\rangle$

Induction: $q\in K[[x]][y]\ ,\ \deg_y q=j ,\ \frac{q}{x^e}\in\overline{A} \Rightarrow\frac{q}{x^e}\in A'_j$. Let $q=x^e\cdot \widetilde{q}, x\nmid\widetilde{q}\Rightarrow \exists\ u$ unit in $K[[x,y]] \widetilde{q}=u\cdot \overline{q}$, $\overline{q}$ Weierstra{\ss} poly in $y$ of degree $\leq j$.

$\frac{x^e\overline{q}u}{x^e}\in\overline{A}\Rightarrow \frac{\overline{q}}{x^{e-c}}\in\overline{A}$. If $e-c\leq 0$ there is nothing to prove. We may assume $c=0$. If $\deg y\overline{q}<j\Rightarrow \frac{\overline{q}}{x^e}\in A'_k$ (induction). If $\deg y\overline{q}=j$ then $u\in K$ and we may assume that $q$ is monic. This implies $e\leq e(p_j)$. We obtain $\frac{q}{x^e}=\frac{x^{e(p_j)-e}q}{x^{e(p_j)}}= x^{e(p_j)-e}\cdot \frac{p_j}{x^{e(p_j)}}+\frac{r}{x^{e(p_j)}}\ ,\ \deg_y r<j$.

We obtain the claim using induction. It remains to prove $\frac{\overline{q}}{x^e}\in A'_k\Rightarrow \frac{u\overline{q}}{x^e}\in A'_j $. 

We know $\deg_y u=j-k$. This implies $u\in A'_{j-k}$. But $\frac{\overline{q}}{x^e}\in k\Rightarrow y^i\frac{\overline{q}}{x^e}\in A'_{k+i}$ for $i\leq j-k$. This implies $\frac{u\overline{q}}{x^e}\in A'_j$.

\end{proof}

\begin{lemma}
\label{lem split}Let $f\in K[X,Y]$ be a Weierstra\ss \ polynomial in
$K[[X]][Y]$ with respect to $Y$ and
\[
f=f_{1}\cdot\ldots\cdot f_{s}%
\]
be the decomposition into irreducible polynomials, and let $h_{i}=\frac
{f}{f_{i}}$. By the Euclidean algorithm in $Q(K[[X]])[Y]$ for the coprime
$f_{i}$ and $h_{i}$, there are $a_{i},b_{i}\in K[[X]][Y]$, $c\in\mathbb{N}$
with%
\[
a_{i}f_{i}+b_{i}h_{i}=X^{c}\text{.}%
\]
The normalization splits as%
\[
\overline{K[[X]][Y]/\left\langle f\right\rangle }=\overline
{K[[X]][Y]/\left\langle f_{i}\right\rangle }\oplus\overline
{K[[X]][Y]/\left\langle h_{i}\right\rangle }%
\]
and the splitting is given by%
\[
(a\operatorname{mod}f_{i},b\operatorname{mod}h_{i})\mapsto\frac{b_{i}%
h_{i}a+a_{i}f_{i}b}{X^{c}}\text{.}%
\]

\end{lemma}

\begin{proof}
Follows by the Chinese remainder theorem.
\end{proof}

\begin{lemma}
\label{lem complete split}With the notation as in Lemma \ref{lem split}, let
\[
p_{0}^{(i)},\frac{p_{1}^{(i)}}{X^{e_{1}^{(i)}}},\ldots,\frac{p_{d_{i}-1}%
^{(i)}}{X^{e_{d_{i}-1}^{(i)}}}%
\]
be the integral basis obtained from Lemma \ref{lem intbas power series} for
$\overline{K[[X]][Y]/\left\langle f_{i}\right\rangle }$ and define the
$K[[X]]$-modules%
\[
B^{(i)}=\left\langle 1,y,\dots, y^{d_i}, \frac{b_{i}h_{i}}{X^{c}},\frac{b_{i}h_{i}p_{1}^{(i)}%
}{X^{c+e_{1}^{(i)}}},\ldots,\frac{b_{i}h_{i}p_{d_{i}-1}^{(i)}}{X^{c+e_{d_{i}%
-1}^{(i)}}}\right\rangle \text{.}%
\]
Then the normalization is
\[
\overline{K[[X]][Y]/\left\langle f\right\rangle }=\sum_{i=1}^{s}%
B^{(i)}\text{.}%
\]

\end{lemma}

\begin{proof}
Follows from Lemma \ref{lem split}.
\end{proof}

\begin{proposition}
\label{prop semilocal}With the notation as in Lemma \ref{lem complete split},
let $\tilde{b}_{i},\tilde{h}_{i}\in K[x,y]$with%
\[
\tilde{b}_{i}\tilde{h}_{i}\equiv b_{i}h_{i}\operatorname{mod}X^{c+e_{d_{i}%
}^{(i)}-1}%
\]
define the $K[x]_{\left\langle x\right\rangle }$-modules%
\[
A^{(i)}=\left\langle 1,y,\dots, y^{d_i},\frac{\tilde{b}_{i}\tilde{h}_{i}}{X^{c}},\frac{\tilde
{b}_{i}\tilde{h}_{i}p_{1}^{(i)}}{X^{c+e_{1}^{(i)}}},\ldots,\frac{\tilde{b}%
_{i}\tilde{h}_{i}p_{d_{i}-1}^{(i)}}{X^{c+e_{d_{i}-1}^{(i)}}}\right\rangle
\text{.}%
\]
Then%
\[
\widehat{A^{(i)}}=B^{(i)}%
\]
and%
\[
\overline{K[X,Y]_{\left\langle X,Y\right\rangle }/\left\langle f\right\rangle
}=\sum_{i=1}^{s}A^{(i)}\text{.}%
\]

\end{proposition}

\begin{proof}
Follows from Lemma \ref{lem complete split} by faithfully flatness and since
$\widehat{\overline{R}}=\overline{\widehat{R}}$ for any excellent ring $R$.
\end{proof}

To compute a local contribution to the normalization at $\langle x, y\rangle$, we have to take into account the unit in the factorization of $f$.

\begin{proposition}
\label{prop local contribution} Let $f = u \prod_{i=1}^r f_i$ as at the beginning of this section, with $u \in K[X, Y]$ a unit in $K[[X, Y]]$ of degree $d$ in $Y$  and $f_i$, $1 \le i \le r$, irreducible Weierstrass polynomials

With the notation as in Proposition \ref{prop semilocal},
define the $K[x]_{\left\langle x\right\rangle }$-modules%
\[
\tilde A^{(i)}=\left\langle 1,y,\dots, y^{d_i+d-1},\frac{u\tilde{b}_{i}\tilde{h}_{i}}{X^{c}},\frac{u\tilde
{b}_{i}\tilde{h}_{i}p_{1}^{(i)}}{X^{c+e_{1}^{(i)}}},\ldots,\frac{u\tilde{b}%
_{i}\tilde{h}_{i}p_{d_{i}-1}^{(i)}}{X^{c+e_{d_{i}-1}^{(i)}}}\right\rangle
\text{.}%
\]
Then $\sum_{i=1}^{s}\tilde A^{(i)}$ is a minimal local contribution to the normalization at the origin.
\end{proposition}

In Section \ref{sect:loc-contr-via-Hensel}, we will design algorithms to
compute rings $A^{(i)}$ with the properties specified in Proposition
\ref{prop semilocal}. \todo{In particular, such rings exist.}

\begin{example}
\label{exampleTwoBranches} Let
\[
A = K[X,Y]/\left\langle (Y^3+X^8)(Y^6 + Y^3X^7 - 2Y^3X^4 + X^8) + X^{20}\right\rangle
=K[x,y]\text{.}%
\]
In $K[[X]][Y]$, the polynomial $f =  (Y^3+X^8)(Y^6 + Y^3X^7 - 2Y^3X^4 + X^8) + X^{20}$
decomposes into two irreducible factors. Hence, $A$ has a singularity at the
origin with two branches.

With the computational means presented in Section
\ref{sect:loc-contr-via-Hensel}, we compute the integral bases of the rings $A^{(1)}$ and $A^{(2)}$ as
in Proposition \ref{prop local contribution}:
\[
A^{(1)}=\left\langle 1, y, y^2, y^3, y^4, y^5, 
\frac{h_1}{x^8}, \frac{h_1 y}{x^{10}}, \frac{h_1 y^2}{x^{13}} \right\rangle,
\]
where $h_1 = (y^6+y^3x^7-2y^3x^4+x^8)$ and 
\[
A^{(2)}=\left\langle 1, y, y^2, \frac{h_2}{x^4}, \frac{h_2 y}{x^5}, \frac{h_2 y^2}{x^6}, 
\frac{h_2 h_3}{x^9}, \frac{h_2 h_3 y}{x^{10}}, \frac{h_2 h_3 y^2}{x^{12}}\right\rangle,
\]
where $h_2 = (y^3 + x^8)$ and $h_3 = (y^3-x^4)$

Combining the results, we get $\overline {A}=A^{(1)}+A^{(2)}$.  
\end{example}

We finish the section with an example including expansions that do not vanish at the origin.

\begin{example}
\label{exampleTwoBranchesB} Let
\[
 A = K[X,Y]/\left\langle (Y^{3}-X^{7})(Y^{2}-X^{3})+Y^{6}\right\rangle
 =K[x,y]\text{.}%
 \]
In $K[[X]][Y]$, the polynomial $f = (Y^{3}-X)(Y^{2}+X^{3}) + Y^{7}$
can be factorized as $f = u f_1 f_2$ with 
$u = Y^2 + (-X) Y + 1 + \dots$, $f_1 = Y^3 + X Y^2 -X + \dots$ and $f_2 = Y^2 + X^3 + \dots$ where the dots represent in all cases terms with degree greater than $1$ in $X$.

The rings $K[[X]][Y] / \langle f_i \rangle$, $i = 1, 2$, have integral bases $\{1, \frac{y}{x}, \frac{y^2}{x}\}$ and $\{1, \frac{y}{x}\}$. Adding the factors as in Proposition \ref{prop local contribution}, we get 
\[
A^{(1)} = \left\langle 1, y, y^2, uy, u f_2, \frac{u f_2 y}{x}, \frac{u f_2 y^2}{x} \right\rangle 
\]
and 
\[
A^{(2)} = \left\langle 1, y, y^2, uy, u y^2, \frac{u f_1}{x}, \frac{u f_1 y}{x^2} \right\rangle.
\]
\end{example}


% With the computational means presented in Section
% \ref{sect:loc-contr-via-Hensel}, we compute rings $A^{(1)}$ and $A^{(2)}$ as
% in Proposition \ref{prop semilocal}:\todo{adjust}
% \[
% A^{(1)}=\frac{\left\langle x^{7},x^{6}y,x^{4}h_{1},x^{2}h_{1}y,h_{1}%
% y^{2}\right\rangle }{x^{7}},
% \]
% where $h_{1}=y^{2}+x^{3}y+2x^{6}y-x^{3}-x^{6}$, and
% \[
% A^{(2)}=\frac{\left\langle x^{6},x^{5}y,x^{3}y^{2},x^{3}h_{2},h_{2}%
% y\right\rangle }{x^{6}},
% \]
% where $h_{2}=y^{3}$ (note that he polynomials $h_{i}$ are truncations of the
% elements $F_{i}$ specified in Proposition \ref{prop semilocal}).\todo{adjust}
% \todo{what do we wish to express
% with the sentence in brackets?} Combining the results, we get $\overline
% {A}=A^{(1)}+A^{(2)}$. \todo{Better: Note that in this example, $A^{(2)}$
% is contained in $A^{(1)}$. Hence, $\overline{A} = A^{(1)}+A^{(2)}=A^{(1)}$. Then we have to get a different example.
% See Remark 4.21 for more on this phenomenon.}
% \end{example}

\section{Normalization of Plane Curves via Localization and Completion: 
Handling the Branches Using Puiseux Expansions 
and Hensel's Lemma}

\label{sect:loc-contr-via-Hensel}


Let $A$ be the ring 
\[
A=K[x,y]=K[X,Y]/\langle f(X,Y)\rangle
\]
as before. For such a ring, in this section we
show how to compute a local contribution to $\overline{A}$ at each prime ideal
$P\in\Sing(A)$ via Puiseux expansions and Hensel's lemma.
Once the local contribution at each component of the singular locus is
obtained, the global integral basis is computed applying Proposition
\ref{prop local contribution}.


We start with a sketch of the algorithm.

\begin{enumerate}
\item \textbf{Translation of the singularity to the origin.} If the prime
component corresponds to a singular point, apply a translation to the
variables so that the singularity is moved to the origin. If the prime
component corresponds to a groups of conjugated singular points, apply first a
linear transformation so that no point has the same $X$-coordiante and then
apply a translation so that one of the singularity is moved to the origin (an
algebraic field extension is needed in this case).
\end{enumerate}

For the singularity at the origin,

\begin{enumerate}[resume]
\item Determine the maximum integrality exponent $e = e_{n-1}$. (See Defintion \ref{defn integrality exponent}.)

\item \textbf{Factorization by Hensel Lemma.} Applying Hensel Lemma, compute
the factorization $f = h \prod_{i=1}^{r} f_{i}$ as in Section \ref{sect:norm-via-loc-and-compl-the-theory}, developing the factors $h$ and $f_{i}$, $1 \le i \le s$
up to degree $e$ in $X$.

\item \textbf{Local contribution of each factor.} For each $1 \le i \le r$,
compute the local contribution $A^{(i)}$ from Proposition \ref{prop local contribution}
as follows

\begin{enumerate}
\item For each $j = 0, \dots, s-1$, $s = \deg(f_{i})$, compute a polynomial
$p_{j}$ of degree $j$ in $Y$ with maximal valuation $d_j = v_{f_{i}}(p_{j})$ among
all the polynomials of the same degree.

\item Develop the product $F_{i} = \prod_{j \ne i} f_{j}$  up to degree $e$ in $X$.

\item The local contribution $A^{(i)}$ to the integral basis of the
singularity at the origin is
\[
\left\{  1, y, y^{2}, \dots, y^{\deg{h} + \deg{F_{i}} -1}, h F_{i}, h F_{i}
\frac{p_{1}}{x^{d_{1}}}, \dots, h F_{i} \frac{p_{s-1}}{x^{d_{s-1}}}\right\}
.
\]

\end{enumerate}

\item Apply the inverse translation to the elements of the local contribution
to restore the singularity to the original position.

\item If the component corresponds to a groups of singularities, modify the
numerators and denominators of the local contribution to obtain the local
contribution of the component over the original field.
\end{enumerate}

In the following subsections we explain these steps in more detail.

\subsection{Basic Remarks on Puiseux Series}

\label{sect:basic-Puiseux}

We fix our notation and recall a few results in the context of Puiseux series.

\subsubsection*{Puiseux Series}

Let $K\subset L$ be a field extension, with $L$ algebraically closed. Write
$L[[X]]$ for the ring of formal power series in $X$ over $L$ and $L((X)) =
{\text{Q}}(L[[X]])$ for the field of formal Laurent series. The \textit{{field
of Puiseux series}} over $L$ is the field
\[
L\{\{X\}\}=\bigcup_{m=1}^{\infty}L((X^{1/m})).
\]
This field arises naturally in the context of Emmy Noether's finiteness
theorem. In fact, $L\{\{X\}\}$ is the algebraic closure of $L((X))$, and the
integral closure of $L[[X]]$ in $L((X^{1/m}))$ is $L[[X^{1/m}]]$ (see
\cite[Corollary 13.15]{Eis}).

We have a canonical {\emph{valuation map}}
\[
v:L\{\{X\}\}\setminus\{0\}\rightarrow{\mathbb{Q}},\;\gamma\mapsto v(\gamma),
\]
where $v(\gamma)$ is the smallest exponent appearing in a term of $\gamma$. By
convention, $v(0)=\infty$. The corresponding \emph{valuation ring}
$L\{\{X\}\}_{v\geq0}$ consists of all Puiseux series with non--negative
exponents only. Henceforth it will be denoted by ${\mathcal{P}_{X}}$.

If $p\in L\{\{X\}\}[Y]$ is any polynomial in $Y$ with coefficients in
$L\{\{X\}\}$, the {\emph{valuation}} of $p$ at $\gamma\in L\{\{X\}\}$ is
defined to be $v_{\gamma}(p):=v(p(\gamma))$.

\subsubsection*{Conjugate Puiseux Series}

Two Puiseux series in $L\{\{X\}\}$ are called \emph{conjugate} if they are
conjugate as field elements over $K((X))$.

\subsubsection*{Rational Part}

Let $\gamma= a_{1}X^{t_{1}}+a_{2}X^{t_{2}}+\dots+a_{k}X^{t_{k}}+
a_{k+1}X^{t_{k+1}}+ \dots\in{\mathcal{P}_{X}}$, with $0 \le t_{1}<t_{2}<
\dots$. Let $k \geq0$ be such that $a_{i} X^{t_{i}}\in K[X]$ for $1 \leq i
\leq k$ and $a_{k+1} X^{t_{k+1}} \not \in K[X]$. Then we call $a_{1}X^{t_{1}%
}+\dots+a_{k}X^{t_{k}}$ the \emph{rational part} of $\gamma$, and
$a_{k+1}X^{t_{k+1}}$ its \emph{first non--rational term}.

\subsubsection*{Characteristic Exponents}

For $\gamma\in\mathcal{P}_{X}$, let $m\in{\mathbb{N}}$ be minimal with
$\gamma\in L[[X^{1/m}]]$, and write $\gamma=\sum_{i \geq0}b_{i}X^{i/m}$, with
coefficients $b_{i}\in L$. If $m=1$, there are no characteristic exponents. If
$m\geq2$, the {\emph{characteristic exponents}} of $\gamma$ are defined
inductively by
\begin{align*}
e_{1}  &  :=\min\{i \mid b_{i}\neq0\text{ and }m\nmid i\},\\
e_{\nu}  &  :=\min\{i \mid b_{i}\neq0\text{, }\gcd(e_{1},\dots,e_{\nu-1})\nmid
i\}\text{ for }\nu>1\text{.}%
\end{align*}
Then $e_{1} < e_{2} < \dots$; in fact, there are only finitely $e_{\nu}$, and
these are coprime.

\begin{example}
If $\gamma=2X^{1/2}+X^{3/4}+6X^{5/4}-5X^{17/8}$, the common denominator is
$m=8$. Writing $\gamma=2X^{4/8}+X^{6/8}+6X^{10/8}-5X^{17/8}$, we see that the
characteristic exponents are $e_{1}=4$, $e_{2}=6$, and $e_{3}=17$.
\end{example}

\subsubsection*{Puiseux Expansions}

Since $L\{\{X\}\}$ is algebraically closed, the polynomial $f\in
K[X,Y]=K[X][Y]$ has $n=\deg_{Y}(f)$ roots $\gamma_{1},\dots,\gamma_{n}$ in
$L\{\{X\}\}$. These roots are called the \emph{Puiseux expansions} of $f$ (at
$X=0$). Since $f$ is supposed to be monic in $Y$, we have a factorization of
type
\[
f=(Y-\gamma_{1})\cdots(Y-\gamma_{n})\in L\{\{X\}\}[Y].
\]
In particular, each $\gamma_{i}$ is integral over $L[[X]]$ and, thus,
contained in some $L[[X^{1/m}]] \subset{\mathcal{P}_{X}}$. That is, the terms
of $\gamma_{i}$ have non--negative exponents only.

\subsubsection*{Regularity Index and Singular Part}

\noindent If $\gamma= a_{1}X^{t_{1}}+a_{2}X^{t_{2}}+\dots$ is a Puiseux
expansion of $f$, with $0 \le t_{1}<t_{2}<\dots$ and no $a_{i}$ zero, we
define the \emph{regularity index} of $\gamma$ to be the least exponent
$t_{k}$ such that no other Puiseux expansion of $f$ has the same initial part
$a_{1}X^{t_{1}}+\dots+a_{k}X^{t_{k}}$. This initial part is, then, called the
\emph{singular part} of $\gamma$.
%\todo{What precisely do we want? up to $k$ or up to $k-1$?}


\subsubsection*{The Newton-Puiseux Algorithm}

The Puiseux expansions of $f$ can be computed recursively up to any given
order using the Newton-Puiseux algorithm. Essentially, to get a solution
$a_{1}X^{t_{1}}+a_{2}X^{t_{2}}+\dots$ of $f(X, \gamma(X))=0$, with
$t_{1}<t_{2}<\dots$, the algorithm proceeds as follows: Starting from $f^{(0)}
= f$ and $K^{(0)} = K((X))$, we commence the $i$th step of the algorithm by
looking at a polynomial $f^{(i-1)}\in K^{(i-1)} [Y]$. We then choose one face
$\Delta$ of the Newton polygon of $f^{(i-1)}$ such that all the other points
of the polygon lie on or above the line containing the face. Let
$f^{(i-1)}_{\Delta}$ be the sum of terms of $f^{(i-1)}$ involving the
monomials of $f^{(i-1)}$ on $\Delta$. That is, if $-\frac{w_{1}}{w_{2}}$ is
the slope of $\Delta$, then $f^{(i-1)}_{\Delta}$ is the sum of terms of
$f^{(i-1)}$ of lowest $(1,\frac{w_{2}}{w_{1}})$--weighted degree. We write
$d_{i}$ for this degree. Choose an irreducible factor of $f^{(i-1)}_{\Delta}$
over $K^{(i-1)}$ and a root $q_{i}$ of that factor. Note that $q_{i}$ is of
type $q_{i}=c_{i}X^{\frac{w_{2}}{w_{1}}}$, where $c_{i}$ is a root of the
polynomial $f^{(i-1)}_{\Delta}(1,Y)$. Now, let $K^{(i)} = K^{(i-1)}(q_{i})$
and set $f^{(i)} = \frac{1}{X^{d_{i}}}p^{(i-1)}(X, q_{i}\cdot(1+Y))$. Then the
$i$th term of the expansion to be constructed is $a_{i}X^{t_{i}} = q_{1}\cdots
q_{i}$.  It is clear from this construction that
different conjugacy classes of expansions arise from different choices for the
faces and irreducible factors of $f^{(i-1)}_{\Delta}$ over $K^{(i-1)}$, respectively.

\begin{example}
\label{examplePuiseux} The eight Puiseux expansions of the polynomial
\begin{align*}
f  &  =Y^{8}+(-4X^{3}+4X^{5})Y^{7}+(4X^{3}-4X^{5}-10X^{6})Y^{6}+(4X^{5}%
-6X^{6})Y^{5}\\
&  +(6X^{6}-8X^{8})Y^{4}+(8X^{8}-4X^{9})Y^{3}+(4X^{9}+4X^{10})Y^{2}%
+4X^{11}Y+X^{12}\\
&  \in{\mathbb{Q}}[X,Y]
\end{align*}
are conjugate over ${\mathbb{Q}}((X))$; their singular parts are of type
\[
q_{1}+q_{1}q_{2}+q_{1}q_{2}q_{3},
\]
where the $q_{i}$ satisfy
\[%
\begin{tabular}
[c]{ccccc}%
$q_{1}^{2}+X^{3}=0$, &  & $q_{2}^{2}+\frac{1}{2X}q_{1}=0$, &  & \text{and }
$q_{3}^{2}+\frac{1}{16X}q_{1}=0$.
\end{tabular}
\
\]
To see this, note that the Newton polygon of $f^{\left(  0\right)  }=f$ has
only one face $\Delta_{0}$, leading to $f_{\Delta_{0}}^{\left(  0\right)
}=\left(  X^{3}+Y^{2}\right)  ^{4}$ and the extension%
\[
K_{0}=\mathbb{Q}((X))\subset K_{1}=K_{0}[iX^{\frac{3}{2}}].
\]
In the next step, $f^{\left(  1\right)  }$ has only one face $\Delta_{1}$,
yielding
\[
f_{\Delta_{1}}^{\left(  1\right)  }=4\left(  2Y^{2}+\frac{q_{1}}{X}\right)
^{2}%
\]
and%
\[
K_{1}\subset K_{2}=K_{0}[iX^{\frac{3}{2}},(1-i)X^{\frac{1}{4}}].
\]
Finally, also $f^{\left(  2\right)  }$ has only one face $\Delta_{2}$, which
corresponds to
\[
f_{\Delta_{2}}^{\left(  2\right)  }=-2\cdot\left(  8Y^{2}-\frac{q_{1}}%
{X}\right)
\]
and the extension%
\[
K_{2}\subset K_{3}=K_{0}[iX^{\frac{3}{2}},(1-i)X^{\frac{1}{4}},(1+i)X^{\frac
{1}{4}}]=K_{0}[i,X^{\frac{1}{4}}].
\]

\end{example}

\subsubsection*{Puiseux Blocks}

To simplify the presentation of our algorithms, we introduce some special
terminology. We partition the set of all Puiseux expansions of $f$ into
\emph{Puiseux blocks}. A Puiseux block represented by an expansion $\gamma$
with $\gamma\left(  0\right)  =0$ is obtained by collecting all expansions
whose rational part agrees with that of $\gamma$ and whose first non--rational
term is conjugate to that of $\gamma$ over $K((X))$. \todo{that is?} A \emph{Puiseux segment}
is defined as the union of all blocks having the same initial exponent. That
is, we have one Puiseux segment for each face of the Newton polygon of $f$. In
addition, all Puiseux expansions $\gamma$ of $f$ with $\gamma\left(  0\right)
\not =0$ are grouped together to a single Puiseux block of an extra Puiseux
segment. In this way, the Puiseux expansions of $f$ are divided into Puiseux
segments, each segment consists of Puiseux blocks, and each block is the union
of classes of conjugate expansions.

\begin{example}
\label{examplePuiseux2} Suppose that the Puiseux expansions of a polynomial
$f$ are \begin{multicols}{2}
\noindent
\begin{align*}
\gamma_1 &= 1 + X^2 + \dots, \\
\gamma_2 &= -1 + 3X + \dots, \\
\gamma_3 &= a_1 X^{3/2} + 2X^2 + \dots, \\
\gamma_4 &= a_2 X^{3/2} + 2X^2 + \dots, \\
\gamma_5 &= X + 3 X^2 + \dots, \\
\gamma_6 &= X + b_1 X^{5/2} + X^3 + \dots, \\
\gamma_7 &= X + b_2 X^{5/2} + X^3 + \dots, \\
\gamma_8 &= X + b_1 X^{5/2} + X^{4} + \dots, \\
\gamma_9 &= X + b_2 X^{5/2} + X^{4} + \dots,
\end{align*}
\end{multicols}
\noindent where $\{\gamma_{3}, \gamma_{4}\}$, $\{\gamma_{6}, \gamma_{7}\}$ and
$\{\gamma_{8}, \gamma_{9}\}$ are pairs of conjugate Puiseux series. Then
$\{\gamma_{1}, \gamma_{2}\}$ is the segment of expansions $\gamma$ with
$\gamma\left(  0\right)  \not =0$. Another segment is $\{\gamma_{3},
\gamma_{4}\}$ (which consists of one block containing a single class of
conjugate expansions). All the other expansions form a single segment,
consisting of the blocks $\{\gamma_{5}\}$ and $\{\gamma_{6}, \gamma_{7},
\gamma_{8}, \gamma_{9}\}$. The last block contains two classes of conjugate
expansions, namely $\{\gamma_{6}, \gamma_{7}\}$ and $\{\gamma_{8}, \gamma
_{9}\}$.
\end{example}

\subsubsection*{Maximal Integrality Exponents}

\label{sect:max-expo} 

Let $\varGamma = \{\gamma_{1},\dots,\gamma_{n}\}$ be the set of Puiseux
expansions of $f$ at $x = 0$, and let $p\in{\mathcal{P}_{X}}[Y]$. The
{\emph{valuation}} of $p$ at $f$ is defined to be $v_{f}(p)=\min_{1 \leq i
\leq n} v_{\gamma_{i}}(p)$. Note that if $p$ is monic of degree $d$, where
$1\leq d \leq n-1$, and
\[
p=(Y-\eta_{1}(X))\cdots(Y-\eta_{d}(X))\mbox{, } \text{ with all }\eta_{i}
\in{\mathcal{P}_{X}},
\]
is the factorization of $p$ in ${\mathcal{P}_{X}}[Y]$, then 
\[
v_{f}(p)=\min_{1 \leq i \leq n}\sum_{j=1}^{d}v(\gamma_{i}-\eta_{j})\text{.}%
\]

\begin{lemma}
\label{lemma:intA} With notation as above, fix an integer $d$ with $1\leq d
\leq n-1$. If $\mathcal{A}\subset\{1,\dots,n\}$ is a subset of cardinality
$d$, set
\[
\Int({\mathcal{A}})=\min_{i\not \in \mathcal{A}}\left(  \sum_{j\in\mathcal{A}%
}v(\gamma_{i}-\gamma_{j})\right)  .
\]
Choose a subset $\widetilde{\mathcal{A}}\subset\{1,\dots,n\}$ of cardinality $d$ such
that $\Int({\widetilde{\mathcal{A}}})$ is maximal among all $\Int({\mathcal{A}})$ 
as above, and set $\widetilde{p}=\prod_{j\in\widetilde{\mathcal{A}}%
}(Y-\gamma_{j})$. Then $v_{f}(\widetilde{p})=\Int({\widetilde{\mathcal{A}}})$,
and this number is the maximum valuation $v_{f}(p)$, for $p\in {\mathcal{P}_{X}}[Y]$ 
monic in $Y$ of degree $d$. %We write $o(\varGamma,d)=v_{f}(\tilde{p})$.
\end{lemma}

\begin{proof}
It is clear from the definitions that $v_{f}(\widetilde{p})=\Int({\widetilde
{\mathcal{A}}})$. That this number is the maximum valuation $v_{f}(p)$ 
as claimed follows as in the proof of \cite[Theorem 5.1]{vanHoeij94},
where the case $d=n-1$ is treated. 
\end{proof}

\begin{notation} In the situation of the lemma, we write $o(\varGamma,d)
=v_{f}(\tilde{p})$. In case $d = n-1$, we abbreviate
\[
\Int_{i}= \Int(\{1,\dots,i-1,i+1, \dots,n\}) = \sum_{j\neq i}v(\gamma
_{i}-\gamma_{j}).
\]
\end{notation}

\begin{example}
\label{exampleOneBranch} Let $f=(Y^{2}+2X^{3})+Y^{3}\in{\mathbb{Q}}[X,Y]$. The
Puiseux expansions of $f$ are
\begin{align*}
\gamma_{1}  &  =a_{1}X^{3/2}+X^{3}+\dots\;,\\
\gamma_{2}  &  =a_{2}X^{3/2}+X^{3}+\dots\;,\\
\gamma_{3}  &  =-1-2X^{3}+\dots\;,
\end{align*}
where $a_{1}, a_{2}$ are the roots of $X^{2}=-2$. Then $\Int_{1} = 3/2 + 0 =
3/2$, $\Int_{2} = 3/2 + 0 = 3/2$, and $\Int_{3} = 0 + 0 = 0$, so that both $i
= 1$ and $i=2$ maximize the valuation. Taking $i = 1$, we get $\widetilde{p} =
(Y - \gamma_{2})(Y - \gamma_{3})$ and $o(\varGamma,2) = e(\widetilde{p}) 
= \lfloor3/2\rfloor=1$.
%is the maximal valuation $v_{f}(p)$ in degree $d=2$.
\end{example}

\begin{example}
\label{exampleTwoBranches-int} 
Let $f = (Y^3+X^8)(Y^6 + Y^3X^7 - 2Y^3X^4 + X^8) + X^{20}
\in{\mathbb{Q}}[X,Y]$ be the polynomial from Example \ref{exampleTwoBranches}%
. The Puiseux expansions of $f$ at $X = 0$ are 
\begin{multicols}{3}
\noindent
\begin{align*}
\gamma_{1} &= -X^{8/3}+ \dots, \\
\gamma_{2} &= \xi_{1}X^{8/3} + \dots, \\
\gamma_{3} &= \xi_{2}X^{8/3} + \dots, \\
\end{align*}
\end{multicols}
\vspace{-1cm}
\begin{multicols}{2}
\noindent
\begin{align*}
\gamma_{4} &= X^{4/3} + \xi_3 X^{17/6} + \dots, \\
\gamma_{5} &= -\xi_{1}X^{4/3} + \xi_{5} X^{17/6} + \dots, \\
\gamma_{6} &= -\xi_{2}X^{4/3} + \xi_{7} X^{17/6} + \dots, \\
\gamma_{7} &= X^{4/3} + \xi_4 X^{17/6} + \dots, \\
\gamma_{8} &= -\xi_{1}X^{4/3} + \xi_{6} X^{17/6} + \dots, \\
\gamma_{9} &= -\xi_{2}X^{4/3} + \xi_{8} X^{17/6} + \dots
\end{align*}
\end{multicols}
\noindent where $\xi_{1}, \xi_{2}$ are the complex roots of $X^{3} + 1 = 0$;
$\xi_{3}, \xi_{4}$ are the roots of $X^{2} + 1/9 = 0$;  
$\xi_{5}, \xi_{6}$ are the roots of $X^{2} + 1/9 \xi_{1} - 1/9 = 0$ and
$\xi_{7}, \xi_{8}$ are the roots of $X^{2} + 1/9 \xi_{2} - 1/9 = 0$.

Then $\Int_{1} = \Int_{2} = \Int_{3} = 2 \cdot 8/3 + 6 \cdot 4/3 = 40/3$ and
$\Int_{4} = \Int_{5} = \Int_{6} = \Int_{7} = \Int_{8} = \Int_{9} = 3 \cdot 4/3 + 17/6 + 4 \cdot 4/3 = 73 / 6$.
We conclude that $o(\varGamma,8) = \lfloor 40/3 \rfloor = 13$.
% is the maximal valuation $v_{f}(p)$ in degree $d=5$.
\end{example}

% \begin{example}
% \label{exampleTwoBranchesA-int} 
% \todo{TWOBRANCHES modify this example using the new example with no inclusion}
% Let $f = (Y^{3}-X^{7})(Y^{2} - X^{3})+Y^{6}
% \in{\mathbb{Q}}[X,Y]$ be the polynomial from Example \ref{exampleTwoBranchesA}%
% . The Puiseux expansions of $f$ at $X = 0$ are \begin{multicols}{2}
% \noindent
% \begin{align*}
% \gamma_{1}  &= X^{7/3}+ \dots, \\
% \gamma_{2}  &= \xi_{1}X^{7/3}+ \dots, \\
% \gamma_{3}  &= \xi_{2}X^{7/3}+ \dots, \\
% \gamma_{4}  &= X^{3/2}+ \dots, \\
% \gamma_{5}  &= -X^{3/2}+ \dots, \\
% \gamma_{6}  &= -1 + X^3 + \dots,
% \end{align*}
% \end{multicols}
% \noindent where $\xi_{1}, \xi_{2}$ are the complex roots of $X^{3} - 1 = 0$.
% Then $\Int_{1} = \Int_{2} = \Int_{3} = 7/3 + 7/3 + 3/2 + 3/2 + 0 = 23/3$,
% $\Int_{4} = \Int_{5} = 3/2 + 3/2 + 3/2 + 3/2 + 0 = 6$, and $\Int_{6} = 0$. 
% We conclude that $o(\varGamma,5) = 7$.
% % is the maximal valuation $v_{f}(p)$ in degree $d=5$.
% \end{example}

\begin{remark}
\label{rem:prop-max-val}
In general, applying the trace map as in the proof of \cite[Theorem 5.1]{vanHoeij94},
we see from Lemma \ref{lemma:intA} that for each $d$, we may characterize
the number $o(\varGamma,d)$ also as the maximal valuation $v_{f}(p)$, 
for $p\in K[X][Y]$ monic in $Y$ of degree $d$. Note that by construction, 
$$o(\varGamma,1)\leq \dots \leq o(\varGamma,n-1).$$
\end{remark}

The reason for considering the valuations $v_{f}(p)$ is that they are 
directly related to integrality:

\begin{lemma}
\label{lemma:integrality} Let $p\in K[X,Y]=K[X][Y]$ be monic in $Y$ of degree
$d$. Then the integer part $\lfloor v_{f}(p)\rfloor$ of $v_{f}(p)$ is the
integrality exponent of $p$ with respect to $f$, that is, it is the maximum number $e \in{\mathbb{N}}$ such that $p(x,y)/x^{e}$ is integral over
$A=K[x,y]$.
\end{lemma}

\begin{proof}
Note that $p(x,y)/x^{e}$ is integral over $A$ iff $v_{\gamma_{i}}(p(x,y)/x^{e}) \ge0$ 
for each $1 \le i \le n$ (see \cite[Theorem 3.2.6]{Stichtenoth08} and 
\cite[Section 2.4]{vanHoeij94}).  Since $v_{f}(p)$ is defined to be the minimum 
of the $v_{\gamma_{i}}(p)$, the result follows.
\end{proof}

Taking Remark \ref{rem:prop-max-val} into account, we conclude that $o(\Gamma,d)$ is the {\emph{maximal integrality exponent
of $f$ in degree $d$}}, as defined in Definition \ref{defn integrality exponent}. We note 
$$
E(f):=o(\Gamma,d-1),
$$
the {\emph{maximal integrality exponent of $f$}}.



\subsection{Computation of the maximum integrality exponent $e$}
\label{subsection:maxexp}
Denote by $\gamma_{1}(X),\dots,\gamma_{n}(X)$ the Puiseux expansions of $f$ at
$X=0$. As noted in \cite[Theorem 5.1]{vanHoeij94}, if we allow for a more
general $\tilde{p}\in{\mathcal{P}_{X}}[Y]$ which is a polynomial in $Y$ with
coefficients in Puiseux series in $X$, the maximal integrality exponent can be
obtained by choosing $\{\eta_{1}(X),\dots,\eta_{n-1}(X)\}$ to be a subset of
$\{\gamma_{1}(X),\dots,\gamma_{n}(X)\}$. In that paper it is also explained
how to determine which subset to take (see also Section \ref{sect:max-expo}).

The coefficients of $\tilde{p}$ may not lie in the ground field $K$, and
furthermore $\tilde{p}$ may contain fractional exponents. As mentioned before,
by using the trace map, van Hoeij proves that there exists $p\in K[X,Y]$ monic
of $Y$--degree $n-1$ with $e(p)=e(\tilde{p})$.

In \cite{vanHoeij94} these ideas are only used to fix bounds for the algorithm
but not for constructing $p$. In this work, we show that $p$ can be easily
constructed, using Hensel's Lemma to efficiently compute the product
$(Y-\eta_{1}(X))\cdots(Y-\eta_{n-1}(X))$, or more precisely, the product of
the truncated expansions of these factors up to appropriate degrees.

\subsection{Hensel's Lemma\label{sec Hensel}}

In this subsection and the following, we explain how to use Hensel's lemma to
compute the products $(Y - \gamma_{1})\cdots(Y-\gamma_{s})$ up to any
$X$--degree, with $\gamma_{1}, \dots, \gamma_{s}$ conjugate expansions
belonging to a Puiseux segment or block, without computing each individual
expansion. Computing all the expansions separately and then computing the
product is usually much slower.

We recall Hensel's Lemma.

\begin{lemma}
\label{lemma:hensel} Let $f \in K[[X]][Y]$ be a monic polynomial over the
power series ring, and assume that $f(0, Y) = g_{0} h_{0}$ for monic
polynomials $g_{0}, h_{0} \in K[Y]$ such that $\langle g_{0}, h_{0}\rangle=
K[Y]$. Then there exist monic polynomials $g, h \in K[[X]][Y]$ such that

\begin{enumerate}
\item $f = gh$

\item $g(0, Y) = g_{0}$, $h(0, Y) = h_{0}$.
\end{enumerate}

Moreover, for each $m \in{\mathbb{N}}$, there exist unique $g_{m}, h_{m} \in
K[X,Y]$ of $X$--degree $m$ such that

\begin{enumerate}
\item $f \equiv g_{m} h_{m}$ in $(K[[X]] / \langle X^{m+1} \rangle)[Y]$

\item $g_{m} \equiv g_{i}$, $h_{m} \equiv h_{i}$ in $(K[[X]]/ \langle X^{i+1}
\rangle)[Y]$, $i = 0, \dots, m-1$.
\end{enumerate}
\end{lemma}

These last conditions imply that the polynomials $g_{m}$ and $h_{m}$ can be
computed inductively along the $X$--degree, solving for each $m$ a determined
system of linear equations of $n$ equations and $n$ unknowns, where $n$ is the
$Y$--degree of $f$. (For each $i$, $0 \leq i \leq n-1$, we get an equation by
comparing the coefficients of $X^{m}Y^{i}$ in $f$ and in $g_{m}h_{m}$.) For
further reference in the paper, we present this well-known procedure as
Algorithm \ref{alg:Hensel}, omitting the actual computation steps. (This
algorithm has been made available in \textsc{Singular}{} since version 3.1.3
via the command \texttt{factmodd}.)

\begin{algorithm}                      % enter the algorithm environment
\caption{Hensel's lifting}          % give the algorithm a caption
\label{alg:Hensel}
\begin{algorithmic}[1]
\REQUIRE $f \in K[X,Y]$ irreducible polynomial, monic in $Y$; $g_0, h_0 \in K[Y]$ such that $f(0, Y) = g_0 h_0$ and $\langle g_0, h_0 \rangle = K[Y]$; $d \in \N_0$
\ENSURE $g, h \in K[X,Y]$ such that $g(0, Y) = g_0$, $h(0, Y) = h_0$ and $f \equiv gh$ in $(K[[X]]/ \langle X^{d+1} \rangle)[Y]$
\end{algorithmic}
\end{algorithm}


We will usually use Hensel's lifting to separate the component that vanishes
at the origin from the component that vanishes outside. Alternatively, we
could perform this decomposition by means of the Weierstrass Division Theorem.
(See for example, \cite[Theorem 3.2.3]{JP}.) However, the use of Hensel's
Lemma allows for more generality, since we do not need to move the singularity
to the origin. This is particularly useful when the singularity has no
rational coordinates, as we avoid to use algebraic extensions. Also the linear
algebra techniques involved in Hensel's Lemma are usually faster than
computing division of polynomials with remainders.

In the following example, we show how to use the lemma to decompose a
polynomial and compute the integral basis in a simple case.

\begin{example}
Let $f = (Y-X)(Y+X)(Y+2X) + Y^{7}$. There are $3$ Puiseux expansions at $Y =
0$ and $4$ expansions outside $Y = 0$. (The degree of $f$ in $Y$ is $7$, so
there must be a total of $7$ expansions.)

We call $\gamma_{1} = X + \dots, \gamma_{2} = -X + \dots, \gamma_{3} = -2X +
\dots$ the expansions at the origin and $\gamma_{4}, \dots, \gamma_{7}$ the
expansions outside the origin. We want to compute $(Y-\gamma_{4}%
)\cdots(Y-\gamma_{7})$ up to a given degree in $X$ without computing each
expansion separately.

Here $\Int_{i} = 2$ for $i = 1, 2, 3$ and this is maximal. So $e(\tilde g) =
2$ and we need to compute the product up to degree $2$ in $X$.

Since $f(0, Y) = Y^{3} + Y^{7}$, we take $g_{0} = Y^{3}$, $h_{0} = 1 + Y^{4}$,
and apply Hensel's lemma to lift these factors up to degree $2$. We obtain
$g_{2} = Y^{3} + 2XY^{2} -2X^{2}Y$ and $h_{2} = 5X^{2}Y^{2}-2XY^{3}+Y^{4}+1$,
and we conclude that $(Y-\gamma_{4})\cdots(Y-\gamma_{7}) \equiv5X^{2}%
Y^{2}-2XY^{3}+Y^{4}+1$ modulo $X^{3}$.

Taking $i = 1$, to compute $p$, we still have to compute $\gamma_{2}$ and
$\gamma_{3}$ up to degree $2$. We obtain $\bar\gamma_{2} = -X$ and $\bar
\gamma_{3} = -2X$. Combining all this we compute
\[
p = \prod_{i=2}^{7} (Y - \bar\gamma_{i}) = (Y-X)(Y+X)(Y+2X)(5X^{2}Y^{2}%
-2XY^{3}+Y^{4}+1).
\]

\end{example}

\subsection{A local version of Hensel's Lemma}

When we want to lift two factors $g$, $h$ that vanish at $Y = 0$ (for example,
to compute $(Y-\gamma_{1})(Y-\gamma_{2})(Y-\gamma_{3})$ as in Example
\ref{exampleTwoBranches} up to any given order), the condition $\langle g(0,
Y), h(0,Y)\rangle= K[Y]$ is not satisfied.

We explain how to transform the polynomials so that Hensel's lemma can still
be applied.

Let $f$ have the following Puiseux expansions at $0$:
\begin{align*}
\gamma_{1}  &  = a_{1}^{1} X^{t_{1}^{1}} + a_{2}^{1} X^{t_{2}^{1}} + \dots\\
\gamma_{2}  &  = a_{1}^{2} X^{t_{1}^{2}} + a_{2}^{2} X^{t_{2}^{2}} + \dots\\
&  \dots\\
\gamma_{s}  &  = a_{1}^{s} X^{t_{1}^{s}} + a_{2}^{s} X^{t_{2}^{s}} + \dots
\end{align*}
and assume $t = t_{1}^{1} = \min_{1 \leq i \leq s}t_{1}^{i}$. We define $f_{0}
= (Y-\gamma_{1}) \dots(Y-\gamma_{s})$, $f_{0} \in K[[X]][Y]$.

We would like to replace $Y$ by $X^{t}Y$, so that we can factor out $X^{t}$ in
all factors. But this will introduce fractional exponents in $f_{0}$, so we
first write $t = u/v$ and replace $X$ by $X^{v}$ and $Y$ by $X^{u}Y$. We
define
\begin{align*}
\tilde f_{0}(X, Y)  &  = f_{0}(X^{v}, X^{u}Y)\\
&  = (X^{u} Y - (a_{1}^{1} X^{vt_{1}^{1}} + \dots)) \cdots(X^{u} Y -
(a_{1}^{s} X^{vt_{1}^{s}} + \dots)\\
&  = X^{su}(Y - (a_{1}^{1} + a_{2}^{1} X^{\tilde t_{2}^{1}} + \dots))\cdots(Y
- (a_{1}^{s} X^{\tilde t_{1}^{s}} + \dots))
\end{align*}
and
\begin{align*}
F(X, Y)  &  = \tilde f_{0}(X, Y) / X^{su}\\
&  = (Y - (a_{1}^{1} + a_{2}^{1} X^{\tilde t_{2}^{1}} + \dots))\cdots(Y -
(a_{1}^{s} X^{\tilde t_{1}^{s}} + \dots)),
\end{align*}
with $F(X, Y) \in K[[X]][Y]$.

So we can first use Hensel's lemma to compute the factor $f_{0}$ up to the
required degree, and then compute $F$ as defined above.

Now $F$ has factors that do not vanish at the origin. So we can use again
Hensel's lemma to separate the factors that vanish at the origin from the
factors that do not. We get $F=GH$. We obtain the factors $g$ and $h$ by
reversing the transformations, $g(X,Y)=G(X^{1/v},Y/X^{u/v})$, and likewise for
$h$.

We thus arrive to Algorithm \ref{alg:SegmentSplitting}.

\begin{algorithm}                      % enter the algorithm environment
\caption{Segment splitting}          % give the algorithm a caption
\label{alg:SegmentSplitting}
\begin{algorithmic}[1]
\REQUIRE $f \in K[X,Y]$ irreducible polynomial, monic of degree $s$ in $Y$, with no Puiseux expansions vanishing outside the origin; $d \in \N_0$.
\ENSURE $g_1, \dots, g_k \in K[X,Y]$ such that the expansions of $g_i$ correspond to the $i$-th Puiseux segment of $f$, developed up to degree $d$.
\STATE $t_1, \dots, t_k$ the different initial exponents of the Puiseux expansions of $f$ (which are obtained from the Newton polygon of $f$)
\IF{$k = 1$} \RETURN $f$
\ENDIF
\STATE $t = u/v = \min\{t_1, \dots, t_k\}$, with $u, v \in \N$
\STATE $\tilde f(X, Y) = f(X^v, X^uY)$
\STATE $F = \tilde f / X^{su}$
\STATE Compute $G_0, H_0 \in K[Y]$ such that $F(0, Y) = G_0 H_0$, $G_0 = Y^w$,
for some $w \in \N$ and $Y \nmid H_0$
\STATE $(G, H) = $ Hensel($F, G_0, H_0, vd$)
\STATE $g_1 = G(X^{1/v}, Y/X^{u/v})$, $h = G(X^{1/v}, Y/X^{u/v})$
\RETURN $\{g_1\} \cup \SegmentSplitting(h)$.
\end{algorithmic}
\end{algorithm}


See also \cite[Theorem W]{JP} for an alternative approach extending the
Weierstrass Division Theorem.

\begin{example}
We return to Example \ref{exampleTwoBranches}, $f=Y^{6}-(Y^{2}+2X^{3}%
)((Y+2X^{2})^{2}+X^{5})$. We want to compute $(Y-\gamma_{1})(Y-\gamma_{2})$ up
to order $5$. We first use Hensel's lemma to lift the factors $Y^{4}$ and
$1+Y^{2}$ up to degree $8$ (we must lift up to this degree so that no
information from $f$ is lost). We obtain {\tiny
\begin{align*}
f_{0}  &  =48X^{8}Y^{3}+46X^{8}Y^{2}-8X^{7}Y^{3}-16X^{8}Y-8X^{7}Y^{2}%
+32X^{6}Y^{3}+2X^{8}-4X^{6}Y^{2}\\
&  -8X^{5}Y^{3}+8X^{7}+X^{5}Y^{2}+8X^{5}Y+4X^{4}Y^{2}+2X^{3}Y^{2}+4X^{2}%
Y^{3}+Y^{4}\\
f_{1}  &  =-48X^{8}Y+210X^{8}+8X^{7}Y-56X^{7}-32X^{6}Y+4X^{6}+8X^{5}%
Y-X^{5}+12X^{4}-2X^{3}-4X^{2}Y+Y^{2}-1
\end{align*}
} (Note that $f_{1} = (Y-\gamma_{5})(Y-\gamma_{6})$ up to order $8$, so we can
truncate it up to order $5$ to get the product of the expansions outside the origin.)

The smallest $t$ is $t = u/v = 3/2$. We compute $\tilde f_{0} = f_{0}(X^{2},
X^{3}Y) = X^{12}(48X^{13}Y^{3}-8X^{11}Y^{3}+46X^{10}Y^{2}+32X^{9}Y^{3}%
-8X^{8}Y^{2}-8X^{7}Y^{3}-16X^{7}Y-4X^{6}Y^{2}+X^{4}Y^{2}+2X^{4}+4X^{2}%
Y^{2}+4XY^{3}+Y^{4}+8X^{2}+8XY+2Y^{2}) = X^{12}F(X,Y)$.

Now, $F(0, Y) = (Y^{2}+2)Y^{2}$ and we use Hensel's lemma to lift the factors
$Y^{2}+2$ and $Y^{2}$. After lifting and mapping the factors back to the
original $X$ and $Y$, we obtain
\begin{align*}
g  &  = -4X^{6}-8X^{5}Y+2X^{3}+Y^{2}\\
h  &  = X^{5}+4X^{4}+4X^{2}Y+Y^{2}%
\end{align*}
Note that $g = (Y^{2} + 2X^{3})-8X^{5}Y-4X^{6}$ and $h = (Y+2X^{2})^{2}+X^{5}$
are equal in the low degree terms to the factors appearing in $f$.

We can now compute $p = (Y - \bar\gamma_{1})(Y - \bar\gamma_{2})(Y -
\bar\gamma_{4})(Y - \bar\gamma_{5})(Y - \bar\gamma_{6}) = ((Y^{2} +
2X^{3})-8X^{5}Y-4X^{6})Y(8X^{5}Y-X^{5}+12X^{4}-2X^{3}-4X^{2}Y+Y^{2}-1)$.
\end{example}

To separate all the Puiseux segments, we can use this method iteratively. In
each step we separate the segment with smallest initial exponent from the
rest. Now consider blocks inside a segment which have the same initial
exponents but whose initial terms are not conjugate. In this case we can also
use Hensel's lemma to split the blocks after applying the above
transformation, hence we can still proceed in the same way.

To be able to separate all blocks, it remains to consider the separation of
blocks that have the same initial rational term (and therefore the same
initial exponent). Suppose that $f_{1}$ is a factor of $f$ containing some
Puiseux blocks of $f$ such that they all have the same initial terms $\eta=
a_{1} X^{m_{1}} + \dots+a_{k} X^{m_{k}}$, $a_{i} \in K, m_{k} \in{\mathbb{N}%
}_{0}$. (There can be fewer terms than in the rational part of the
expansions.) In this case, we first apply the transformation $Y = Y_{1} +
\eta$, and compute $f_{2}(X,Y_{1}) = f_{1}(X, Y_{1} + \eta)$. Then $f_{2}$
will contain the same expansions as $f_{1}$ but without the initial terms
$\eta$. We can now proceed as before to separate the blocks. After computing
the factors corresponding to each block, we replace $Y_{1}$ by $Y - \eta$, to
get the factor we were looking for.

Algorithm \ref{alg:BlockSplitting} summarizes these ideas.

\begin{algorithm}                      % enter the algorithm environment
\caption{Block splitting}          % give the algorithm a caption
\label{alg:BlockSplitting}
\begin{algorithmic}[1]
\REQUIRE $f \in K[X,Y]$ irreducible polynomial, monic of $Y$--degree $n$; $d \in \N_0$.
\ENSURE $f_0, f_1, \dots, f_r$ such that the expansions of each $f_i$ are the same as the $i$-th Puiseux block of $f$ up to order $d$ in $X$.
\STATE compute $g_0, h_0 \in K[Y]$ such that $g_0h_0 = f(0, Y)$, $g_0 = Y^k$ for some $k \in \N_0$ and $\langle g_0, h_0 \rangle = K[Y]$
\STATE $(f_0, g) = $ Hensel$(f, g_0, h_0, d)$, where $f_0$ is the lifting of $h_0$ and $g$ is the lifting of $g_0$, up to order $d$ in $X$
\STATE $\{g_1, \dots, g_s\} = \SegmentSplitting(g, d)$, the factors corresponding to the different Puiseux segments of $g$
\FORALL{$g_i$, $i = 1, \dots, s$}
\STATE $\eta_i := $ the common rational part of all expansions in $g_i$
\STATE $\tilde g_i = g_i(X, Y + \eta_i)$
\STATE $\{\tilde g_{i,1}, \dots, \tilde g_{i, r_i}\} = $ BlockSplitting($\tilde g_i, d$)
\STATE $g_{i,j}(X,Y) = \tilde g_{i, j}(X, Y - \eta_i)$, $j = 1, \dots, r_i$
\ENDFOR
\STATE $\{f_1, \dots, f_r\} = \cup_{i = 1}^s \{g_{i,1}, \dots, g_{i, r_i}\}$
\RETURN $\{f_0, f_1, \dots, f_r\}$
\end{algorithmic}
\end{algorithm}


The ideas from \cite[Theorem 5.1.20]{JP} can in some cases also be used for
our purpose. However, the cited theorem is not as general as we require, and
the details on how to initiate the algorithm are not given.

Our final goal is to separate all factors corresponding to different conjugacy
classes of expansions. In this case, we do not know of any algorithm to do it
without working in algebraic extensions. We compute the conjugate Puiseux
expansions $\bar{\gamma}_{1},\dots,\bar{\gamma}_{s}$ up to the desired degree
and then compute the product $(Y-\bar{\gamma}_{1})\cdots(Y-\bar{\gamma}_{s})$.
This last step is only needed when a Puiseux block contains more than one
conjugacy class of expansions.

In Algorithm \ref{alg:Splitting} we combine all the contents of this subsection in a general splitting algorithm.

\begin{algorithm}                      % enter the algorithm environment
\caption{Splitting}          % give the algorithm a caption
\label{alg:Splitting}
\begin{algorithmic}[1]
\REQUIRE $f \in K[X,Y]$ irreducible polynomial, monic of $Y$-degree $n$; $e \in \N_0$.
\ENSURE $L = \{f_0, f_1, \dots, f_r\}$ such that the expansions of each $f_i$ are the same as the $i$-th conjugacy class of Puiseux expansions of $f$ up to order $e$ in $X$.
\STATE compute $\{g_0, g_1, \dots, g_s\} = \BlockSplitting(f, e)$
\STATE $L = \{g_0\}$
\FOR{$i = 1, \dots, s$}
\STATE Compute $\varGamma = \{\gamma_1, \dots, \gamma_l\}$, the singular part of the expansions of $g_i$
\STATE $k = $ number of conjugacy classes in $\varGamma$
\IF{$k > 1$}
\FOR{$j = 1, \dots, k$}
\STATE Compute $\varGamma_j = \{\gamma_{j,1}, \dots, \gamma_{j, s_j}\}$, the expansions of the $j$-th conjugacy class of $\varGamma$, up to order $e$ in $X$
\STATE $h_j = (Y- \gamma_{j, 1}) \cdots (Y - \gamma_{j, s_j})$
\ENDFOR
\STATE $L = L \cup \{h_1, \dots, h_k\}$
\ELSE
\STATE $L = L \cup \{g_i\}$
\ENDIF
\ENDFOR
\RETURN $L$.
\end{algorithmic}
\end{algorithm}


\subsection{Local integral basis}

\label{section:lowdegree}

Let $g \in K[[X]][Y]$ be an irreducible Weierstrass polynomial of degree $m$. We show how to compute the integral basis for $\overline{K[[X]][Y]/\langle g \rangle}$ over $K[[X]]$. That is, we compute the polynomials $p_1$, \dots, $p_{m-1}$ described in Lemma \ref{lemma loc int bas shape} and their corresponding integrality exponents.

For each $d$, $0 \leq d \leq m-1$, we look for a polynomial $p_d \in K[X][Y]$ of degree $d$ with maximal valuation
at $g$. 

Let $\varGamma$ be the set of Puiseux expansions of $g$. Since we are assuming $g$ is irreducible, all the expansions of $g$ are conjugate. 

%For any subset $\varDelta$ of
%expansions of $g$ of $s$ elements, and any $c\in{\mathbb{N}}_{0}$, $0\leq
%c<s$, we set $g_{\varDelta}=\prod_{\delta\in\varDelta}(Y-\delta)
%\in{\mathcal{P}_{X}}[Y]$ and define
%\[
%o(\varDelta,c)=\max_{\substack{N\subset\varDelta \\\#N = c}}\left\{
%v_{g_{\varDelta}}\left(  \prod_{\eta\in N}(Y-\eta)\right)  \right\}  .
%\]

For any $d\in{\mathbb{N}}_{0}$, $0 \leq d < m$, we define
\[
o(\varGamma, d)=
\max_{\substack{N\subset\varGamma \\\#N = d}}\left\{v_{g}\left(  \prod_{\eta\in N}(Y-\eta)\right)  \right\}.
\]

Recall that for a given $N \subset\varGamma$, we have the formula
\[
v_{g}\left(  \prod_{\eta\in N}(Y - \eta)\right)  = \min_{\delta
\in\varGamma\setminus N} \left\{  \sum_{\eta\in N} v(\delta- \eta)\right\}  .
\]

To compute $o(\varGamma,d)$, $1 \leq d < m$, we do not apply the above
formulas but we compute a polynomial $p_d \in K[X,Y]$ of $Y$--degree $d$ such
that $v_{g}(p_d)=o(\varGamma,d)$, recursively truncating the expansions of $g$. 

We assume first that there exists $t\in\mathbb{Q}$ such that the conjugated expansions $\gamma_{1},\dots,\gamma_{m}$ of $g$ 
agree in the terms of degree lower than $t$ and have conjugate coefficients
$c_{i}\in\overline{K}$ at the monomial $X^{t}$, that is%
\[
\gamma_{i}=a_{1}X^{d_{1}}+a_{2}X^{d_{2}}+\dots+a_{k}X^{d_{k}}+c_{i}X^{t}+\dots
\]
where $a_{j}\in K$ and $d_{j}\in{\mathbb{N}}$ for $1 \le j \le k$. That is, the inital part $a_{1}X^{d_{1}}+a_{2}X^{d_{2}}+\dots+a_{k}X^{d_{k}}$ is rational.
To compute the numerator of the element of degree $d$ in the integral basis, we truncate $\gamma_{i}$ to
$\bar{\gamma}_{i}$ for $1 \le i \le d$ to degree $d_{k}$ and we set%
\[
p_d=(Y-\bar{\gamma}_{1}) \cdots(Y-\bar{\gamma}_{d}) \in K[X,Y]
\]

\begin{lemma}
\label{lemma:onecharexp} The polynomial $p_d$ defined above have maximal integrality exponents among all monic polynomials of degree $d$ in $Y$.
\end{lemma}

\begin{proof}
Let $\tilde p_d \in \Px[y]$ be an element of degree $d$ in $Y$ of largest valuation $e_d$ at $g$. We know that we can take $\tilde p_d = (Y - \gamma_{i_1}) \dots (Y - \gamma_{i_d})$ for some $1 \le i_1 \le \dots \le i_d \le m$. Let $i'$ be an index not appearing in $\{i_1, \dots, i_d\}$. We have by construction%
\[
v_{g}(\tilde p_d) = v_{\gamma_{i'}}(\tilde p_d) = \sum_{j=1}^{d}v(\gamma_{i'}%
-\gamma_{j}) =\sum_{j=1}^{d}v(\gamma_{i'}-\bar{\gamma}_{j}) = v_{\gamma_{i'}%
}(p_d).
\]

Since $\gamma_{1}, \dots, \gamma_{s}$ are conjugate and $p_d \in K[X,Y]$,
$v_{\gamma_{j}}(p_d) = v_{\gamma_{i'}}(p)$ for $1 \leq j \leq m$. 

Recall that $v_{g}(p_d) = \min_{1 \leq j \leq s} v_{\gamma_{j}}(p_d)$. So
$v_{g}(p_d) = v_{g}(\tilde p_d)$, as wanted.
\end{proof}

In the general case, the truncation has to be done
iteratively. We describe a recursive process to obtain $p_d$, the numerator of the integral basis of degree $d$ in $Y$.

Let $\tilde p_d \in \Px[y]$ be as in Lemma \ref{lemma:onecharexp}.

%starting with $g_{0}= \prod_{j=1}^{s} (y-\bar{\gamma}_{j})\in K[x,y]$.

The singular parts%
\[
\gamma_{j}^{\operatorname{sing}}=a_{1}^{j}X^{t_{1}}+\dots+a_{k}^{j}X^{t_{k}%
}\text{, }t_{1}<...<t_{k}%
\]
of the expansions $\gamma_{j}$, $1 \leq j \leq m$, are pairwise different and
algebraically conjugate over $K((X))$.

Since $\gamma_{1}^{\operatorname{sing}}\neq\gamma_{j}^{\operatorname{sing}}$,
if we truncate the expansions $\gamma_{j}$, $1 \leq j \leq m$, to degree
$t_{k-1}$:
\[
\gamma^{(1)}_{j}=a_{1}^{j}X^{t_{1}}+\dots+a_{k-1}^{j}X^{t_{k-1}}
\]
and define
\[
\tilde q=%
%TCIMACRO{\tprod \nolimits_{j=2}^{s}}%
%BeginExpansion
{\textstyle\prod\nolimits_{j=1}^{d}}
%EndExpansion
(Y-\gamma^{(1)}_{i_j}),
\]
we have%
\[
v_{\gamma_{i'}}(\tilde q)=v_{\gamma_{i'}}(\tilde{p_d}).
\]
for any $i' \not\in \{i_1, \dots, i_d\}$. That is, the valuation at $\gamma_{i'}$ does not decrease.

For simplicity, we explain first how to compute recursively the element of degree $m-1$, assuming $\tilde p_{m-1} = (Y-\bar{\gamma}_{2}) \cdots(Y-\bar{\gamma}_{m})$.

For the recursion, we define $g_{0} = \prod_{j = 1}^{m} (Y-\gamma_{j})$ and
$\overline{g_{0}} = \prod_{j = 1}^{m} (Y-\gamma^{(1)}_{j})$. Since $t_{k}$ was
the smallest integer for which all the truncated expansions were different,
the expansions $\gamma^{(1)}_{j}$, $1 \le j \le m$, can now be grouped into
sets of identical expansions, each set having the same number of elements.
Denote by $\eta_{1},\dots,\eta_{r}$ the mutually distinct expansions, and set
$g_{1}=(Y-\eta_{1})\cdots(Y-\eta_{r}) \in K[X,Y]$. By construction
$\overline{g_{0}} = g_{1}^{u_{1}}$, with $u_{1}=s/r \in{\mathbb{N}}$.

We start the $i$-th step by applying the whole procedure inductively to
$g_{i-1}$, computing $\overline{g_{i-1}}$, $g_{i}$ and $u_{i}$ such that
$\overline{g_{i-1}}=g_{i}^{u_{i}}$ and $\overline{g_{i-1}}$ comes from
truncating the expansions of $g_{i-1}$. In each step the degree $r_{i}$ of
$g_{i}$ is smaller or equal than the degree $r_{i-1}$ of $g_{i-1}$, and it
will be equal to $1$ after a finite number $w$ of steps (bounded by the degree
$t_{k}$ of the expansions in $g_{0}$). For that value $w$, $r_{w}=1$ and all
the expansions in $g_{w}$ are equal. The desired polynomial is
\[
p_{s-1}=g_{1}^{u_{1}-1}g_{2}^{u_{2}-1}\cdots g_{w}^{u_{w}-1}\in K[X,Y].
\]
We obtain Algorithm \ref{alg:Truncate}.

\begin{algorithm}                      % enter the algorithm environment
\caption{Truncated Factor}          % give the algorithm a caption
\label{alg:Truncate}
\begin{algorithmic}[1]
\REQUIRE $\Delta = \{\gamma_i = a_1^{(i)} X^{t_1} + \dots + a_k^{(i)} X^{t_k}\}_{1 \leq i \leq m}$, a conjugacy class of Puiseux series of finite length.
\ENSURE $q \in K[X, Y]$ of degree $m-1$ in $Y$ such that $v_{\gamma_1}(q) = v_{\gamma_1}(\tilde q)$, with $\tilde q = (Y - \gamma_2)\cdots (Y-\gamma_m)$.
\STATE Set $\eta_1, \dots, \eta_r$ the different expansions in the set $\{\overline{\gamma_1}^{t_{k-1}}, \dots,  \overline{\gamma_s}^{t_{k-1}}\}$
\STATE $p = (Y- \eta_1)\cdots(Y-\eta_r)$
\STATE $u = s / r$
\IF{$r > 1$}
\STATE $p' = \TruncatedFactor(\{\eta_1, \dots, \eta_r\})$
\RETURN $q = p^{u-1}p'$.
\ELSE
\RETURN $q = p^{u-1}$.
\ENDIF
\end{algorithmic}
\end{algorithm}


\begin{lemma}
\label{lemma:truncatedFactor} With notation as above, let $p_{m-1} =
\text{TruncatedFactor}(\gamma_{1}, \dots, \gamma_{m})$. Then $p_{m-1}$ has maximal valuation at $g$ 
over all monic polynomials of degree $m-1$ in $Y$.
\end{lemma}

\begin{proof}
As in the proof of Lemma \ref{lemma:onecharexp}, it is enough to show that
$v_{\gamma_{1}}(p_{m-1}) = v_{\gamma_{1}}(\tilde p_{m-1})$. Let $\overline{\gamma_{2}}, \dots, \overline
{\gamma_{m}}$ be the Puiseux expansions of $p_{m-1}$, correspoding to truncations of
the expansions $\gamma_{2}, \dots, \gamma_{m}$ of $g$. By construction,
$v(\gamma_{1} - \overline{\gamma_{i}}) = v(\gamma_{1} - \gamma_{i})$ for $i =
2, \dots, m$. Hence $v_g(p_{m-1}) = v_{\gamma_1}(p_{m-1}) = v_{\gamma_{1}}(\tilde p_{m-1}) = v_g(\tilde p_{m-1})$ as wanted.
\end{proof}

\begin{example}
\label{exa:conj-classes} Returning to Example \ref{examplePuiseux}, the
singular parts of the Puiseux expansions are
\begin{align*}
\gamma_{1}^{\operatorname{sing}}  &  =iX^{3/2}+(-1/2i-1/2)X^{7/4}+1/4iX^{2}\\
\gamma_{2}^{\operatorname{sing}}  &  =iX^{3/2}+(-1/2i-1/2)X^{7/4}-1/4iX^{2}\\
\gamma_{3}^{\operatorname{sing}}  &  =iX^{3/2}+(1/2i+1/2)X^{7/4}+1/4iX^{2}\\
\gamma_{4}^{\operatorname{sing}}  &  =iX^{3/2}+(1/2i+1/2)X^{7/4}-1/4iX^{2}\\
\gamma_{5}^{\operatorname{sing}}  &  =-iX^{3/2}+(1/2i-1/2)X^{7/4}+1/4iX^{2}\\
\gamma_{6}^{\operatorname{sing}}  &  =-iX^{3/2}+(1/2i-1/2)X^{7/4}-1/4iX^{2}\\
\gamma_{7}^{\operatorname{sing}}  &  =-iX^{3/2}+(-1/2i+1/2)X^{7/4}+1/4iX^{2}\\
\gamma_{8}^{\operatorname{sing}}  &  =-iX^{3/2}+(-1/2i+1/2)X^{7/4}-1/4iX^{2}%
\end{align*}
with $i^{2}=-1$.

Truncating $\gamma_{i}^{\operatorname{sing}}$ to degree $7/4$ we obtain
\begin{align*}
\overline{\gamma_{1}}^{7/4}  &  =\overline{\gamma_{2}}^{7/4}=iX^{3/2}%
+(-1/2i-1/2)X^{7/4}\\
\overline{\gamma_{3}}^{7/4}  &  =\overline{\gamma_{4}}^{7/4}=iX^{3/2}%
+(1/2i+1/2)X^{7/4}\\
\overline{\gamma_{5}}^{7/4}  &  =\overline{\gamma_{6}}^{7/4}=-iX^{3/2}%
+(1/2i-1/2)X^{7/4}\\
\overline{\gamma_{7}}^{7/4}  &  =\overline{\gamma_{8}}^{7/4}=-iX^{3/2}%
+(-1/2i+1/2)X^{7/4}%
\end{align*}
hence $u_{1}=2$ and%
\begin{align*}
g_{1}  &  =(Y-\overline{\gamma_{1}}^{7/4})(Y-\overline{\gamma_{3}}%
^{7/4})(Y-\overline{\gamma_{5}}^{7/4})(Y-\overline{\gamma_{7}}^{7/4})\\
&  =Y^{4}+2X^{3}Y^{2}+2X^{5}Y+X^{6}+1/4X^{7}%
\end{align*}
Applying the whole procedure inductively to $g_{1}$ we obtain $g_{2}%
=Y^{2}+X^{3}$, $u_{2} = 2$ and $g_{3} = Y$, $u_{3} = 2$. Combining the
factors, we get
\[
g=g_{1}^{u_{1}-1}g_{2}^{u_{2}-1}g_{3}^{u_{3}-1}=\left(  Y^{4}+2X^{3}%
Y^{2}+2X^{5}Y+X^{6}+\frac{1}{4}X^{7}\right)  (Y^{2}+X^{3})Y.
\]

\end{example}

For computing the elements of any degree $d$, $1 \le d \le m-1$, we can easily extend the above construction. We getAlgorithm \ref{alg:TruncateGeneral}.

\begin{lemma}
\label{lemma int bas shape}
Let $g \in K[[X]][Y]$ be a Weierstrass polynomial of degree $m$ in $Y$ . 
Let $\varGamma = \{\gamma_{1}, \dots, \gamma_{m}\}$ be the expansions
of $g$ at the origin, which correspond all to the same conjugacy class. Then, for
any $c$, $1 \le c < m$, the output $p_{c} = \mathtt{TruncatedFactorGeneral}(\varGamma,
c)$ is a polynomial with maximal valuation at the origin in the ring
$K[[X]][Y]/\langle g \rangle$ among all polynomials in $K[X,Y]$ monic of degree
$c$ in $Y$.
\end{lemma}

\begin{proof}
Let $\bar\gamma_{1}, \dots, \bar\gamma_{s}$ be the singular part of the
expansions in $\varGamma$. Let $t_{k}$ be the degree in $X$ of these
expansions. Let $g_{0}, \overline{g_{0}}, \dots, g_{w-1}, \overline{g_{w-1}},
g_{w}$ be defined as before.

We have noted in Section \ref{subsection:maxexp} that a polynomial $p_{d}$ satisfying the requirements of the
lemma can be chosen so that all the Puiseux expansions of $p_{d}$ at the
origin are truncations of the expansions in $\varGamma$. This implies that we
can take $p_{d}$ to be a product $p_{d} = g_{1}^{d_{1}} \dots g_{w}^{d_{w}}$
of the polynomials $g_{i}$ with appropriate exponents. To find the exponents,
we note that for all $i$ the polynomials $g_{i+1}^{u_{i+1}}$ and $g_{i}$ have
the same degree, but the valuation of $g_{i}$ at $g$ is larger than the
valuation of $g_{i+1}^{u_{i+1}}$ (since the expansions are developed up to a
larger degree). Hence, to construct $p_{d}$, we must first take $d_{1}$ as
large as possible. Then maximize $d_{2}$ and so on iteratively. This is done
by Algorithm \ref{alg:TruncateGeneral}.
\end{proof}

We can now compute $o(\varGamma, c)$ by the formula
\[
o(\varGamma, c) = \sum_{\eta\in N} v(\gamma- \eta),
\]
where $N = \{\eta_{1}, \dots\eta_{d}\}$ are the expansions appearing in
$p_{d}$ and $\gamma\in\varGamma$. (For any expansion $\gamma\in\varGamma$ the
result of the sum is the same, because conjugating the above expression does
not modify $N$.)

\begin{algorithm}                      % enter the algorithm environment
\caption{Truncated Factor General}          % give the algorithm a caption
\label{alg:TruncateGeneral}
\begin{algorithmic}[1]
\REQUIRE $\Delta = \{\gamma_i = a_1^{(i)} X^{t_1} + \dots + a_k^{(i)} X^{t_k}\}_{1 \leq i \leq s}$, a conjugacy class of Puiseux series of finite length; $c \in \N$, $c < s$.
\ENSURE $p \in K[X, Y]$ of $Y$--degree $c$ such that $v_{f_\Delta}(p) = v_{f_\Delta}(\tilde p)$, with $\tilde p$ the element in $\Px[Y]$ of degree $c$ with maximal valuation at $f_\Delta$.
\STATE Set $\eta_1, \dots, \eta_r$ the different expansions in the set $\{\overline{\gamma_1}^{t_{k-1}}, \dots,  \overline{\gamma_s}^{t_{k-1}}\}$
\STATE $u = \lfloor c / r \rfloor$, $c' = c - ur$
\STATE $g_1 = \TruncatedFactorGeneral(\{\eta_1, \dots, \eta_r\}, c')$
\IF{$u > 0$}
\STATE $g = (Y- \overline{\gamma_1}^{t_{k-1}})\cdots(Y-\overline{\gamma_d}^{t_{k-1}})$
\RETURN $p = g^u g_1$.
\ELSE
\RETURN $p = g_1$.
\ENDIF
\end{algorithmic}
\end{algorithm}


\begin{example}
We carry on Example \ref{exa:conj-classes}, computing all the numerators of
the elements of the integral basis. We have obtained that the element of the
integral basis of degree $m-1 = 7$ is the product $p_{7} = g_{1}
g_{2} g_{3}$, where $g_{1}$, $g_{2}$ and $g_{3}$ have degrees $4$, $2$ and $1$
respectively. To obtain the numerators of the elements of the integral basis
of smaller degree $d$, $1 \le d \le 6$, following Algorithm
\ref{alg:TruncateGeneral}, we have to first take the largest possible power of
$g_{1}$ so that the total degree is smaller than or equal to $d$, then choose
the power of $g_{2}$ in the same way and finally the power of $g_{3}$. We get
the following elements $p_{6} = g_{1} g_{2}$, $p_{5} = g_{1} g_{3}$, $p_{4} =
g_{1}$, $p_{3} = g_{2} g_{3}$, $p_{2} = g_{2}$ and $p_{1} = g_{3}$.

The denominators are powers of $x$. To obtain the exponents, we compute
$o(\varGamma, d)$ for $1 \le d \le 7$ by looking at the expansions
corresponding to each $g_{i}$, $i =1, 2, 3$, given in Example
\ref{exa:conj-classes}. Setting $N_{g_{i}}$ the expansions appearing in
$g_{i}$, $i = 1, 2, 3$, we have $\sum_{\eta\in N_{g_{1}}} v(\gamma- \eta) =
27/4$, $\sum_{\eta\in N_{g_{2}}} v(\gamma- \eta) = 13/4$ and $\sum_{\eta\in
N_{g_{3}}} v(\gamma- \eta) = 3/2$ for any $\gamma\in\varGamma$. Hence
$o(\varGamma, 1) = 3/2$, $o(\varGamma, 2) = 13/4$, $o(\varGamma, 3) = 13/4 +
3/2 = 19/4$, $o(\varGamma, 4) = 27/4$, $o(\varGamma, 5) = 27/4 + 3/2 = 33/4$,
$o(\varGamma, 6) = 27/4 + 13/4 = 10$ and $o(\varGamma, 7) = 27/4 + 13/4 + 3/2
= 23/2$. The exponents in the denominators are the integer part of these
valuations and the integral basis is
\[
\left\{  \frac{g_{3}}{x}, \frac{g_{2}}{x^{3}}, \frac{g_{2} g_{3}}{x^{4}},
\frac{g_{1}}{x^{6}}, \frac{g_{1} g_{3}}{x^{8}}, \frac{g_{1} g_{2}}{x^{10}},
\frac{g_{1} g_{2} g_{3}}{x^{11}}\right\}.
\]
\end{example}

\vspace{0.5cm}

We have shown how to compute the local integral basis when $g \in K[[X]][Y]$ is an irreducible Weierstrass polynomial.
For the general case when $g$ is not irreducible, we add up all the contributions of the branches following Proposition \ref{prop semilocal}.
We get Algorithm \ref{alg:loclocbasis}.

\begin{algorithm}                      % enter the algorithm environment
\caption{Local integral basis}          % give the algorithm a caption
\label{alg:loclocbasis}
\begin{algorithmic}[1]
\REQUIRE $L = \{\{\varGamma_1, f_1\}, \dots, \{\varGamma_r, f_r\}$\}, where $\varGamma_i = \{\gamma_{i,1}, \dots, \gamma_{i,s_i}\}$ is the set of singular parts of the $i$-th conjugacy class of expansions that vanish at the origin of a polynomial $f \in K[X,Y]$ monic in $Y$ and $f_i$ is the corresponding factor developed up to $X$--degree $E(f)$.
\ENSURE $\{(p_0, e_0), \dots, (p_{m}, e_{m})\}$ such that $\{\frac{p_0}{x^{e_0}}, \dots, \frac{p_{m-1}}{x^{e_{m-1}}}\}$ is the local integral basis at the origin and $(p_m, e_m) = (f, \infty)$.
\STATE $m = \deg_Y(f_1 \cdots f_r)$
\FOR{$i = 1, \dots, r$}
\STATE $h = \prod_{j \neq i} f_j$
\FOR{$j = 0, \dots, m-s_i-1$}
\STATE $p_j = Y^j$, $e_j =$ the integrality exponent of $p_j$
\ENDFOR
\FOR{$c = 0, \dots, s_i-1$}
\STATE $q_c = \TruncatedFactorGeneral(\varGamma_i, c)$
\STATE $p_{m-s+c} = h \cdot q_c$, $e_{m-s+c} =$ the integrality exponent of $p_{m-s+c}$
\ENDFOR
\STATE $A^{(i)} = \left\langle \frac{p_{0}}{x^{e_{0}}}, \frac{p_{1}}{x^{e_{1}}},
\dots, \frac{p_{m-1}}{x^{e_{m-1}}}\right\rangle _{K[x]}$
\ENDFOR
\STATE From $A^{(1)} + \dots + A^{(r)}$, compute the integral basis $\{b_0, \dots, b_{m-1}\}$, as explained in Section \ref{sect:Int-basis-via-norm}
\RETURN $\{(p_0, e_0), \dots, (p_{m}, e_{m})\}$, where $p_i$ the denominator of $b_i$ and $e_i$ the exponent of the numerator for $0 \leq i \leq m-1$ and $(p_m, e_m) = (f, \infty)$.
\end{algorithmic}
\end{algorithm}

\begin{example}
\label{exampleTwoBranches-locloc} 
Let $f = (Y^3+X^8)(Y^6 + Y^3X^7 - 2Y^3X^4 + X^8) + X^{20}
\in{\mathbb{Q}}[X,Y]$, from Example \ref{exampleTwoBranches}, and $A = K[X,Y]
/ \langle f \rangle$. 

%The expansions of $f$ are $\gamma_{1,2,3} = \xi_{1,2,3}X^{7/3} +
%\dots$ and $\gamma_{4,5} = \pm X^{3/2} + \dots$, where $\xi_{i}$, $i=1,2,3$,
%are the roots of $X^{3} - 1 = 0$.

As we have seen in Example \ref{exampleTwoBranches-int}, the maximal
integrality exponent is $e = 13$.

We have two conjugacy classes, $\varGamma_{1} = \{\gamma_{i}: i=1, 2, 3\}$ and 
$\varGamma_{2} = \{\gamma_{i}: i = 4, \dots, 9\}$, so we compute
two local rings $A^{(1)}$ and $A^{(2)}$.

For $A^{(1)}$, the development of $(Y - \gamma_{4}) \dots (Y-\gamma_{9})$ up to order
$13$ in $X$ is $h_1 = Y^6 + Y^3X^7 - 2Y^3X^4 + X^8$ (we omit the details
of this computation which can be done using Hensel's lemma). Since $n - s_{1}
- 1 = 9 - 3 - 1 = 5$, the first $5$ numerators of the integral basis are the powers of $y$. 
For $c = 0, 1, 2$, the truncated factors corresponding to
$\varGamma_{1}$ are $1, Y, Y^{2}$. Therefore, the numerators of the generators
of $A^{(1)}$ are $1, y, y^2, y^3, y^4, h_1, h_1 y, h_{1} y^{2}$. Computing
the integrality exponents, we get the integral basis 
{\small
\[
A^{(1)} = \left\langle 1,
y, y^2, y^3, y^4, y^5, \frac{h_1}{x^8}, \frac{h_{1} y}{x^{10}}, \frac{h_{1} y^{2}}{x^{13}}\right\rangle_{K[x]}.
\]
}

Similarly, for $A^{(2)}$, the development of $(Y - \gamma_{1})(Y - \gamma_{2})(Y
- \gamma_{3})$ up to degree $13$ is $h_2 = Y^3 + x^8$. Now $n - s_{2} - 1 = 9 - 6 - 1 = 2$ and the
first numerators of the integral basis are $1, y, y^2$. For $c = 0, \dots, 5$, the truncated factors corresponding to
$\varGamma_{2}$ are $1, Y, Y^2, Y^3 - X^4, (Y^3 - X^4)Y, (Y^3 - X^4)Y^2$. Computing the integrality exponents, we get 
{\small
\[
A^{(2)} = \left\langle 1, y, y^2, 
\frac{h_{2}}{x^{4}}, \frac{h_{2} y}{x^{5}}, \frac{h_{2} y^2}{x^{6}}, 
\frac{h_{2}(y^3-x^4)}{x^{9}}, \frac{h_{2}(y^3-x^4)y}{x^{10}}, \frac{h_{2}(y^3-x^4)y^2}{x^{10}}
\right\rangle_{K[x]}.
\]
}

The normalization of the local ring is $A^{(1)} + A^{(2)}$.
\end{example}

% \begin{example}
% \label{exampleTwoBranchesA-locloc} 
% \todo{TWOBRANCHES modify this example using the new example with no inclusion}
% Let $f = (Y^{3}-X^{7})(Y^{2} - X^{3})+Y^{6}
% \in{\mathbb{Q}}[X,Y]$, from Example \ref{exampleTwoBranchesA}, and $A = K[X,Y]
% / \langle f \rangle$. Ignoring the expansions of $f$ that do not vanish at the
% origin, the expansions of $f$ are $\gamma_{1,2,3} = \xi_{1,2,3}X^{7/3} +
% \dots$ and $\gamma_{4,5} = \pm X^{3/2} + \dots$, where $\xi_{i}$, $i=1,2,3$,
% are the roots of $X^{3} - 1 = 0$.
% 
% As we have seen in Example \ref{exampleTwoBranchesA-int}, the maximal
% integrality exponent is $e = 7$.
% 
% We have two conjugacy classes, $\varGamma_{1} = \{\gamma_{1}, \gamma_{2},
% \gamma_{3}\}$ and $\varGamma_{2} = \{\gamma_{4}, \gamma_{5}\}$, so we compute
% two local rings.
% 
% For $i = 1$, the development of $(Y - \gamma_{4})(Y-\gamma_{5})$ up to order
% $7$, is $h_{1} = Y^{2} + X^{3}Y+ 2X^{6}Y -X^{3} - X^{6}$ (we ommit the details
% of this computation which can be done using Hensel's lemma). Since $n - s_{1}
% - 1 = 5 - 3 - 1 = 1$, the first elements of the integral basis are $b_{0} =
% 1$, $b_{1} = y / x$. For $c = 0, 1, 2$, the truncated factors corresponding to
% $\varGamma_{1}$ are $1, Y, Y^{2}$. Therefore, the numerators of the generators
% of $A^{(1)}$ are $1, y, h_{1}, h_{1} \cdot y, h_{1} \cdot y^{2}$. Computing
% the integrality exponents, we get the integral basis $A^{(1)} = \langle1,
% \frac{y}{x}, \frac{h_{1}}{x^{3}}, \frac{h_{1}\cdot y}{x^{5}}, \frac{h_{1}
% \cdot y^{2}}{x^{7}}\rangle_{K[x]}$.
% 
% Similarly, for $i = 2$, the development of $(Y - \gamma_{1})(Y - \gamma_{2})(Y
% - \gamma_{3})$ is $h_{2} = Y^{3}$. Now $n - s_{2} - 1 = 5 - 2 - 1 = 2$ and the
% first elements of the integral basis are $b_{0} = 1$, $b_{1} = y/x$, $b_{2} =
% y^{2}/x^{3}$. For $c = 0, 1$, the truncated factors corresponding to
% $\varGamma_{2}$ are $1, Y$. Therefore, $A^{(2)} = \langle1, \frac{y}{x},
% \frac{y^{2}}{x^{3}}, \frac{h_{2}}{x^{4}}, \frac{h_{2}\cdot y}{x^{6}}%
% \rangle_{K[x]}$.
% 
% The normalization of the local ring is $A^{(1)} + A^{(2)}$. In this case,
% $A^{(2)} \subseteq A^{(1)}$ and hence the local integral basis of $A$ is
% $\left\langle 1, \frac{y}{x}, \frac{h_{1}}{x^{3}}, \frac{h_{1}y}{x^{5}},
% \frac{h_{1}y^{2}}{x^{7}}\right\rangle _{K[x]}$.
% \end{example}

%\begin{remark}
%For simple examples, the inclusion of local contributions (such as $A^{(2)}
%\subseteq A^{(1)}$ in the last example) usually holds. However, for more
%complicated examples, this might not happen. For example, if $f = (y^{2}%
%-x^{3})(y^{2}-x^{5})(y^{2}-x^{7}) + y^{7}$, the local contribution
%corresponding to the second and third factors are not included one in the
%other, and are both needed for computing the local integral basis. (We omit
%the details of the output, which is too large to display.)
%\end{remark}

We finish this section with a special algorithm for the case of simple singularities.

\begin{lemma}
Let $f\in K[[x]][y]$ be an irreducible Weierstra\ss \ polynomial with respect
to $y$ and $\deg_{y}f=n$. Let $y(x)$ be a Puiseux expansion, and
$y(x)=\sum_{i\geq m}a_{i}x^{\frac{i}{n}}$, $a_{m}\neq0$, $m>n$ and
$\gcd(m,n)<n$. Let $k_{0}=n$, $k_{1}=m$, $k_{2},\ldots,k_{g}$ be the
characteristic exponents and let $\varepsilon$ be a primitive root of unity.
The following holds (cf. \cite[Lemma 5.2.18(1)]{JP} and the proof thereof):

\begin{enumerate}
\item $f=%
%TCIMACRO{\dprod \limits_{i=1}^{n}}%
%BeginExpansion
{\displaystyle\prod\limits_{i=1}^{n}}
%EndExpansion
(y-y(\varepsilon^{i}x))$

\item For $j=1,\ldots,g$ denote by $N_{j}$ the set of all $i\in\{1,\ldots,n\}$
such that%
\[
\frac{k_{0}}{\gcd(k_{0},\ldots,k_{j-1})}\mid i\text{ and }\frac{k_{0}}%
{\gcd(k_{0},\ldots,k_{j})}\not \mid i\text{.}%
\]
Then%
\[
\operatorname{ord}_{x}(y(x)-y(\varepsilon^{i}x))=\frac{k_{j}}{n}%
\]
for all $i\in N_{j}$. In particular, if $g=1$ then%
\[
\operatorname{ord}_{x}(y(x)-y(\varepsilon^{i}x))=\frac{k_{1}}{n}%
\]
if $i$ is not a multiple of $n$.

\item We have%
\begin{align*}
\operatorname{ord}_{x}\frac{\partial f}{\partial y}(x,y(x)) &  =\sum_{j=1}%
^{g}\left(  \gcd(k_{0},\ldots,k_{j-1})-\gcd(k_{0},\ldots,k_{j})\right)
\frac{k_{j}}{n}\\
&  =\operatorname*{Int}\nolimits_{\{1,\ldots,\hat{\imath},\ldots,n\}}%
\end{align*}
for all $i$.
\end{enumerate}
\end{lemma}

\begin{proposition}
\label{prop integral}
With notation as above
\begin{enumerate}
\item For $e=\left\lfloor \operatorname{ord}_{x}\frac{\partial f}{\partial
y}(x,y(x))\right\rfloor $ the element $\frac{\frac{\partial f}{\partial y}%
}{x^{e}}$ is integral over $K[[x]]$ and $e$ is maximal.

\item Let%
\[
e=\left\lfloor \operatorname{ord}_{x}\frac{\partial^{n-1}f}{\partial y^{n-1}%
}(x,y(x))\right\rfloor
\]
Then $\frac{\frac{\partial^{n-1}f}{\partial y^{n-1}}}{x^{e}}$ is integral over
$K[[x]]$, $e=\left\lfloor \frac{k_{1}}{n}\right\rfloor $ and $e$ is maximal.

\item If $g=1$ then
\[
1,\frac{\frac{\partial^{n-1}f}{\partial y^{n-1}}}{x^{e_{1}}},\ldots
,\frac{\frac{\partial f}{\partial y}}{x^{e_{n-1}}}%
\]
with
\[
e_{i}=\left\lfloor \operatorname{ord}_{x}\frac{\partial^{n-i}f}{\partial
y^{n-i}}(x,y(x))\right\rfloor
\]
form an integral basis of $\overline{K[[x,y]]/\left(  f\right)  }$ over
$K[[x]]$.
\end{enumerate}
\end{proposition}

\begin{remark}
If $f=y^{4}-2x^{3}y^{2}-4x^{11}y+x^{6}-x^{19}$ then $y(x)=x^{\frac{6}{4}%
}+x^{\frac{19}{4}}$ is a Puiseux expansion, $g=2$ and $(3)$ in Proposition
\ref{prop integral} does not hold.
\end{remark}

We now prove Proposition \ref{prop integral}.

\begin{proof}
Choose $\Omega\subseteq\{1,\ldots,n\}$ with $\left\vert \Omega\right\vert =d$
and $\operatorname*{Int}_{\Omega}$ maximal. Then%
\[
\bar{p}:=%
%TCIMACRO{\dprod \limits_{j\in\Omega}}%
%BeginExpansion
{\displaystyle\prod\limits_{j\in\Omega}}
%EndExpansion
\left(  y-y(\varepsilon^{j}x)\right)
\]
is a polynomial of degree $d$ with respect to $y$ and $\operatorname{ord}%
_{x}\bar{p}(x,y(x))$ is maximal. Let%
\[
e=\left\lfloor \operatorname{ord}_{x}\bar{p}(x,y(x))\right\rfloor \text{.}%
\]

By Lemma \ref{lemma int bas shape}, for some approximation $p$ of $\bar{p}$, we
can choose $\frac{p(x,y)}{x^{e}}$ as the degree $d$ element in the integral basis.

We obtain $(1)$ for $d=n-1$ and $(2)$ for $d=1$, and $\operatorname*{Int}%
_{\Omega}$ is independent of $\Omega$. The same holds true for $(3)$ in case
$g=1$.

To see $(4)$ we compute $\left\lfloor \operatorname{ord}_{x}\frac{\partial
^{2}f}{\partial y^{2}}(x,y(x))\right\rfloor =4$. However,
\[
\bar{p}=\left(  y-y(-x)\right)  \left(  y-y(ix)\right)
\]
gives $\left\lfloor \operatorname{ord}_{x}\bar{p}(x,y(x))\right\rfloor =6$.
\end{proof}

\subsection{Computation of the local contribution to the integral basis}

For computing the local contribution to the integral basis at the origin, we
have to multiply the numerators of the local integral basis at the origin by
the factor corresponding to the expansions that do not vanish at the origin.
As noted in the sketch of the algorithm at the beginning of this section, for
the terms of degree smaller than the degree of that factor, the integrality
exponent will be $0$.

Combining the results of the previous sections we obtain Algorithm
\ref{alg:HenselBasis}.

%Using the results of the previous sections we now give the algorithm for
%computing a local contribution to the integral basis at the origin. We can
%apply any of the two local integral basis algorithms from the last section. We
%first give a short summary.


%Let $f\in K[X,Y]$ be a polynomial of degree $n$ in $Y$ with an isolated
%singularity at the origin. We first use the Newton-Puiseux method to compute
%just the singular parts of the Puiseux expansions $\gamma_{1},...,\gamma_{s}$,
%as described in Section \ref{sect:basic-Puiseux}.


%As explained in Section \ref{sect:max-expo}, we use the singular parts of the
%Puiseux expansions to find the expansion $\gamma_{j}$ which when discarded
%leads to the maximal integrality exponent $e$.


%We apply Algorithm \ref{alg:BlockSplitting} (which in turns uses Algorithm
%\ref{alg:SegmentSplitting} to split into Puiseux segments) to approximate up
%to order $e$ a splitting of $f$ into Puiseux blocks. Here we use Hensel
%lifting (Algorithm \ref{alg:Hensel}), in particular to obtain the factor not
%vanishing at the origin.


%In order to discard $\gamma_{j}$ from the approximation and still have a
%result $p$ over the ground field, we truncate the expansions of the splitting
%factors using Algorithm \ref{alg:Truncate}. We obtain the highest degree
%element of the integral basis as $\frac{p}{x^{e}}$.


%Now we use all the factors to build the remaining elements of the integral
%basis. For each degree $d$, $0 \leq d \leq n-2$, we determine how many
%expansions of each Pusieux block we have to keep, as explained in
%\ref{section:lowdegree} and we modify the corresponding factors using
%Algorithm \ref{alg:TruncateGeneral}. The numerator of the integral element is
%the product of all the modified factors, and the denominator is $x^{e(d)}$
%where $e(d)$ is the integrality exponent of the numerator.


%These considerations lead us to Algorithm \ref{alg:HenselBasis}.


\begin{algorithm}                      % enter the algorithm environment
\caption{Local contribution to the integral basis}          % give the algorithm a caption
\label{alg:HenselBasis}
\begin{algorithmic}[1]
\REQUIRE $f \in K[X,Y]$ irreducible polynomial, monic of $Y$--degree $n$, with only one singularity, located at the origin.
\ENSURE $b_0, \dots, b_{n-1}$, an integral basis of $K[X, Y] / \langle f \rangle$.
\STATE Compute $\varGamma = \{\bar\gamma_1, \dots, \bar\gamma_s\}$, the singular part of the Puiseux expansions $\{\gamma_1, \dots, \gamma_s\}$ of $f$ that vanish at $Y = 0$
\STATE Compute $e = E(f)$ as indicated in Section \ref{subsection:maxexp}
\STATE $\{h, f_1, \dots, f_r\} = \Splitting(f, e)$, where $f_i$, $i = 1, \dots, r$ are the factors corresponding to the conjugacy classes of expansions of $f$ that vanish at the origin and $h$ is the factor of the expansions that vanish outside, both developed up to degree $e$
\STATE From $\varGamma$ and $\{f_1, \dots, f_r\}$, set $L = \{L_1, \dots, L_s\}$, where $L_i = \{(\varGamma_{(i,1)}, f_{\varGamma_{(i,1)}}),
\dots, (\varGamma_{(i, u_i)}, f_{\varGamma_{(i, u_i)}})\}$,
the singular parts of the conjugacy classes of the $i$-th Puiseux block of $f$ and the corresponding factors
\STATE $m = n - \deg(h)$
\STATE $\{(p_0, o_0), \dots, (p_m, o_m)\} = \LocalIntegralBasis(L)$
\FOR{$i = 0, \dots, \deg(h)-1$}
\STATE $b_i = y^i$
\ENDFOR
\FOR{$i = 0, \dots, m-1$}
\STATE $b_{\deg(h) + i} = h \cdot p_i / x^{\lfloor o(i) \rfloor}$
\ENDFOR
\RETURN $\{b_0, \dots, b_{n-1}\}$.
\end{algorithmic}
\end{algorithm}


In order to obtain a (global) integral basis of $\overline{A}$ over $K[x]$ we
can now use Proposition \ref{prop:local-to-global}.

\begin{remark}
In the presence of conjugated singularities, to get a better perfomance, our
local algorithm can handle groups of conjugate singularities simultaneously,
in a similar way as in \cite[Section 4]{vanHoeij94}. If $I \subset k[X,Y]$ is
an associated prime of the singular locus, corresponding to a group of
conjugate singularities, we apply a linear coordinate change if necessary, so
that no two of these singularities have the same $X$-coordinate. Then we can
find polynomials $q_{1}, q_{2} \in K[X]$ such that $I = \langle q_{1}(X),
Y-q_{2}(X)\rangle$. We take $\alpha$ a root of $q_{1}(X)$ and traslate the
singularity $(\alpha, q_{2}(\alpha))$ to the origin. We compute the local
contribution to integral basis at the origin and apply the inverse traslation
to the output. The common denominator of the resulting generators will be a
power of $x - \alpha$. We replace $(x-\alpha)$ by $q_{1}(x)$ in the
denominators and we eliminate $\alpha$ from the numerators by considering
$\alpha$ as a new variable and reducing each numerator by the numerators of
smaller degree (written all with the same common denominator), using an
elimination ordering $\alpha\gg y \gg x$. Since an integral basis over the
original ring always exists, the elimination process is garanteed to
eliminiate $\alpha$ from the numerators.
\end{remark}

\begin{example}
Let $A = k[X, Y]$ and $f(X, Y) = Y^{3} - (X^{2}-2)^{2}$. The singular locus
contains only one primary component $\langle X^{2} - 2, Y^{2} \rangle$, with
radical $\langle X^{2} - 2, Y\rangle$. It consists of the two conjugated
points $(-\sqrt{2}, 0)$ and $(\sqrt{2}, 0)$. We take $\alpha= \sqrt{2}$ and
compute the local contribution at $(\alpha, 0)$ translating that point to the
origin. After after the inverse translation, we get the integral basis of the
local contribution
\[
\left\{  1, y, \frac{y^{2}}{x-\alpha}\right\}  .
\]


The local contribution to the integral basis at the conjugated singularity is
$\left\{  1, y, \frac{y^{2}}{x+\alpha}\right\}  $. Hence the global integral
basis is $\left\{  1, y, \frac{y^{2}}{x^{2}-2}\right\}  $. (In this simple
case, we did not need to eliminate $\alpha$ from the denominator.)
\end{example}

\begin{example}
Let $A = k[X, Y]$ and $f(X, Y) = (Y-X)^{3} - (X^{2}-2)^{2}$. Now the radical
of singular locus is the prime ideal $\langle X^{2} - 2, Y-X \rangle$. It
consists of the two conjugated points $(-\sqrt{2}, -\sqrt{2})$ and $(\sqrt{2},
\sqrt{2})$. We take $\alpha= \sqrt{2}$ and compute the local contribution at
$(\alpha, \alpha)$. We get the integral basis of the local contribution
\[
\left\{  1, y, \frac{y^{2} - 2\alpha y + 2}{x-\alpha}\right\}  .
\]


To eliminate $\alpha$ from the last numerator, we write all the fractions with
the same denominator $\left\{  \frac{x-\alpha}{x-\alpha}, \frac{y(x-\alpha
)}{x-\alpha}, \frac{y^{2} - 2\alpha y + 2}{x-\alpha}\right\}  $, and we can
now reduce the last one to get $\left\{  1, y, \frac{y^{2} - 2x y +
2}{x-\alpha}\right\}  $. Hence the global integral basis is $\left\{  1, y,
\frac{y^{2} - 2x y + 2}{x^{2}-2}\right\}  $.
\end{example}

\subsection{Appendix: Direct approach}

\label{section:directApproach}

\todo{Is this what is implemented? I am a bit puzzled with respect to the
following text: On the first 24 pages we talk about how fast our new approach is,
and now we say that it is slow. What is the truth?}

The local approach of Algorithm \ref{alg:loclocbasis} can be computationally
slow, since summing up the local results and computing the integral basis from
that requires the computation of Groebner bases of possibly complicated ideals.

We present a direct approach that computes the integral basis at the origin
handling all the different conjugacy classes together, without computing
Groebner bases, and which is therefore usually faster.

However, since the details are very technical, we only give a sketch of the
algorithms for this approach.

The main result for constructing $p$ is given in the following theorem, which
generalizes the results in \cite{vanHoeij94}.

\begin{theorem}
\label{thm:truncation} Let $f \in K[X,Y]$ and $\tilde p \in{\mathcal{P}_{X}%
}[Y]$ of $Y$--degree $d$ with maximal valuation at $f$. Then there exists $p
\in K[X,Y]$ of $Y$--degree $d$ such that $v_{f}(\tilde p) = v_{f}(p)$ and such
that the Puiseux expansions of $p$ are all truncations of expansions of $f$.
\end{theorem}

\begin{proof}
In \cite{vanHoeij94} it is proved that there exists $q \in K[X,Y]$ of
$Y$--degree $d$ such that $v_{f}(\tilde p) = v_{f}(q)$. To construct $p$ we
truncate the expansions appearing in $q$, removing all the terms that do not
coincide with the initial parts of Puiseux expansions of $f$. By doing this,
the valuation does not decrease, $v_{f}(p) = v_{f}(q) = v_{f}(\tilde p)$, and
$p \in K[X,Y]$.
\end{proof}

For the algorithm, instead of starting with $\tilde p$ and then building $p$
from it in such a way that the valuation at $f$ does not decrease, we will
directly build a polynomial $p$ of maximal valuation among all polynomials
coming from truncating expansions of $f$.

What is more important, by Proposition \ref{prop semilocal}\todo{adjust}, the choice and
truncation of a given number of expansions in a conjugacy class can be done
independently of the choice and truncation of expansions in other classes.

\begin{lemma}
\label{lemma:indep} Let $f \in K[X,Y]$ and $v$ the maximal valuation at $f$
among all polynomials in $K[X,Y]$ of $Y$--degree $d$. Let $\varGamma_{1},
\dots, \varGamma_{r}$ be the conjugacy classes of expansions of $f$. There
exist $q_{1}, \dots, q_{r} \in K[X,Y]$ such that $p = q_{1} \cdots q_{r}$ has
$Y$--degree $d$, $v_{f}(p) = v$ and $q_{i}$ has maximal valuation at $f_{i} =
\prod_{\gamma\in\Gamma_{i}} (Y - \gamma)$, $1 \leq i \leq r$, among all the
polynomials of the same $Y$--degree as $q_{i}$.
\end{lemma}

\begin{proof}
This is a direct corollary of Proposition \ref{prop semilocal}\todo{adjust}.
\end{proof}

\begin{comment}
\begin{proof}
For a given $d \in{\mathbb{N}}$, let $\tilde p \in \Px[Y]$ and $p
\in K[X,Y]$ be as in Theorem \ref{thm:truncation}. That is $v_{f}(p) =
v_{f}(\tilde p)$ and $p$ is obtained from $\tilde p$ by truncating its Puiseux
expansions. If different choices of expansions in $\varGamma$ give the maximum
valuation, for each possibility we define $w = (c_{1}, \dots, c_{r})$, $c_{i}$
the number of expansions in $\varGamma_{i}$ and take $\tilde p$ with largest
$w$ under the lexicographical ordering.
Let $N$ be the expansions appearing in $\tilde p$ and for a given $j$ ($1 \leq
j \leq r$) let $N_{j} = N \cap\varGamma_{j}$. Let $\tilde p_{j} = \prod
_{\eta\in N_{j}} (Y - \eta)$ and $p_{j}$ the minimum truncation of the
expansions in $\tilde p_{j}$ to get an element in $K[X,Y]$.
We first show that $p = p_{1} \dots p_{r}$ (that is, the truncation in each
class can be done independently from the truncation in other classes). If
$N_{1} \neq\emptyset$, take $\eta_{1} \in N_{1}$. We want to prove that the
truncation of $\eta_{1}$ to build $p$ (which we note $t(\eta_{1})$) is the
same as the truncation of $\eta_{1}$ to build $p_{1}$ (which we note
$t_{1}(\eta_{1})$).
If not, this means that there exists $\eta_{2}$ in some $N_{i}$, $i > 1$, such
that $t(\eta_{1})$ and $t(\eta_{2})$ are conjugate elements in the extension
$K[X] \hookrightarrow \Px$ and such that no expansion in $N_{1}$
is truncated to $t(\eta_{2})$. (Note that since $p$ is over the ground field,
all the conjugates of $t(\eta_{1})$ must also be expansions of $p$.)
Let $G = G(K[X] \hookrightarrow \Px)$ be the Galois group of the
extension, and let $g \in G$ be such that $g(t(\eta_{1})) = t(\eta_{2})$. Then
$g(\eta_{1}) \in\Gamma_{1}$ but is not in $N_{1}$. But then, to build $\tilde
p$, we could have taken $g(\eta_{1})$ instead of $\eta_{2}$ and $v_{f}(\tilde
p)$ would remain equal (if for $t(\eta_{1})$ the best continuation is
$\eta_{1}$, then for $g(t(\eta_{1}))$ the best continuation must be
$g(\eta_{1})$. This contradicts the fact that, because of the lexicographical
ordering used, we were taking the largest possible number of expansions in
$\Gamma_{1}$.
Therefore, all the conjugates of $t(\eta_{1})$ come from truncating expansions
in $N_{1}$. Now, proceeding inductively, we prove that the truncation of any
expansion can be done inside each conjugacy class $\Gamma_{i}$, independently
of the expansions chosen in other conjugacy classes.
\end{proof}
\end{comment}


We can therefore compute the polynomials of each degree restricting to
products of polynomials $q_{i}$ which only depend on the number of expansions
chosen in each conjugacy class, and therefore we only have to decide optimally
how many expansions to choose in each conjugacy class. We explain this in more detail.

Let $n_{i}$, $1 \leq i \leq r$, be the number of expansions in the $i$-th
conjugacy class. We define $p_{i}(c) = \TruncatedFactorGeneral(\varGamma_{i},
c)$ for $0 \leq c < n_{i}$ and $p_{i}(n_{i}) = f_{i}$ developed up to degree
$e$ in $X$ (which can be done by Algorithm \ref{alg:Splitting}). We call
$N_{i}(c)$ the Puiseux expansions appearing in $p_{i}(c)$.

Next, we consider the set of tuples $T_{d}=\{(c_{1},\dots,c_{r}),c_{i}%
\in{\mathbb{N}}_{0},0\leq c_{i}\leq n_{i},c_{1}+\dots+c_{r}=d\}$. For
$w=(c_{1},\dots,c_{r})\in T_{d}$, the polynomial of maximal valuation at $f$
containing $c_{i}$ expansions in the $i$-th conjugacy class is $p_{w}%
=p_{1}(c_{1})\cdots p_{r}(c_{r})$. The valuation of $p_{w}$ at $f$ can be
computed by the formula
\[
v_{f}(p_{w})=\min_{1 \leq i \leq r}\left\{  o(\varGamma_{i},c_{i})+\sum_{j\neq
i}v_{\gamma_{(i,1)}}(p_{j}(c_{j}))\right\}  .
\]


We look for the vector $w = (c_{1}, \dots, c_{r})$ for which $v_{f}(p_{w})$ is
maximal, and for such $w$ we set $p_{d} := p_{w}$. The numerator of the
element of degree $d$ in the integral basis is $p_{d}$ and the denominator is
$x^{\lfloor v_{f}(p_{d}) \rfloor}$.

\begin{algorithm}                      % enter the algorithm environment
\caption{Integral element}          % give the algorithm a caption
\label{alg:integralElementBasic}
\begin{algorithmic}[1]
\REQUIRE $(\varGamma_1, f_1), \dots, (\varGamma_r, f_r)$,
the singular parts of the conjugacy classes of expansions that vanish at the origin of a polynomial $f \in K[X,Y]$ monic in $Y$ and their corresponding factors developed up to $X$--degree $E(f)$;
$d \in \N_0$, $0 \leq d \leq m = \deg_Y(f_1 \cdots f_r)$.
\ENSURE $(p, o)$, with $p \in K[X,Y]$ of $Y$--degree $d$ of
maximal valuation at $f$; $o \in \Q_{\geq 0}$, the valuation of $p$ at $f$.
\STATE $m_i = \#\varGamma_i$ for $i = 1, \dots, r$
\STATE $T = \{(c_1, \dots, c_r)\}_{c_i \in \N_0 \mbox{, } 0 \leq c_i \leq m_i \mbox{, } c_1 + \dots + c_r = d}$
\FORALL {$w = (c_1, \dots, c_r) \in T$}
\FOR {$i = 1, \dots, r$}
\IF{$0 \leq c_i < m_i$}
\STATE $p_i(c_i) = \TruncatedFactorGeneral(\varGamma_{i}, c_i)$
\ELSE
\STATE $p_{i}(c_{i}) = f_i$
\ENDIF
\ENDFOR
\STATE $p_{w} = p_{1}(c_{1})\cdots p_{r}(c_{r})$
\STATE $v_{g}(p_{w})=\min_{1 \leq i \leq r}\left\{  o(\varGamma_{i},c_{i})+\sum_{j\neq
i}v_{\gamma_{i}}(p_{j}(c_{j}))\right\}$, where $\gamma_{i}$ is any expansion of $\varGamma_i$.
\ENDFOR
\STATE $p = p_w$ for $w$ such that $v_{g}(p_{w})$ is maximal
\RETURN $(p, v_{g}(p))$.
\end{algorithmic}
\end{algorithm}


To apply the algorithm as described above, we must run over all the elements
of $T_{d}$ and compute the corresponding valuations. This can still be slow
when $T_{d}$ is large.

We explain how to find the optimal $(c_{1}, \dots, c_{r}) \in T_{d}$ in an
efficient way. Instead of considering tuples of $r$ elements, we will always
consider tuples of $2$ elements and proceed iteratively.

For each Puiseux block $\varPi_{i}$, $1 \leq i \leq a$, we define
\[
L_{i} = \{(\varGamma_{(i,1)}, f_{\varGamma_{(i,1)}}), \dots, (\varGamma_{(i,
r_{i})}, f_{\varGamma_{(i, r_{i})}})\},
\]
where $\varGamma_{(i, j)}$ are the singular parts of the $j$-th conjugacy
classes of the $i$-th block and $f_{\varGamma_{(i, j)}}$ is the corresponding
factor of $f$ developed up to $X$--degree in $e$.

For a list $L$ of this kind, we define $f_{L} = \prod_{(\Gamma, f_{\varGamma})
\in L} f_{\varGamma}$ and we show how to compute $p_{L}(c)$, the polynomial in
$K[X,Y]$ of $Y$--degree $c$ of maximal valuation at $f_{L}$, $0 \leq c \leq
m$, where $m$ is the degree of $f_{L}$. For a Puiseux series $\gamma$, the
notation $\gamma\in L$ will mean that there exists $(\Gamma, f_{\Gamma}) \in
L$ such that $\gamma\in\Gamma$.

We group the lists in new lists $\varLambda_{1}, \dots, \varLambda_{u}$ such
that all the expansions in the same list $\varLambda_{i}$ have the same
initial term (or conjugate initial terms). We order them in increasing order
by the initial exponent. (The order among groups with the same initial
exponent is not important.) Since $v(\gamma_{i})$ is the same for any
$\gamma_{i} \in\varLambda_{i}$, we define $v(i) = v(\gamma_{i})$. The key
property is that if $1 \leq i < j \leq u$, then $v(\gamma_{i} - \gamma_{j}) =
v(i)$ for any $\gamma_{i} \in\varLambda_{i}$ and $\gamma_{j} \in
\varLambda_{j}$.

Let $m_{i}$ be the number of expansions in $\varLambda_{i}$, $1 \leq i \leq
u$. We define $\varTheta_{i} = \varLambda_{i} + \dots+ \varLambda_{u}$ and we
want to compute inductively $p_{\varTheta_{i}}(c)$, for $0 \leq c \leq m_{i} +
\dots+ m_{u}$.

We start by computing $p_{\varTheta_{u}}(c) = p_{\varLambda_{u}}(c)$ for $0
\leq c \leq m_{u}$. For any $1 \leq i \leq u$ and $1 \leq c \leq m_{u}$ we can
compute $p_{\varLambda_{i}}(c)$ using Algorithm \ref{alg:integralElementBasic}%
, or applying this new algorithm recursively as we will see below.

Now, proceeding inductively, once we have computed $p_{\varTheta{i+1}}(c)$ for
all $0 \leq c \leq m_{i+1} + \dots+ m_{u}$, we want to compute
$p_{\varTheta_{i}}(c)$ for all $0 \leq c \leq m_{i} + \dots+ m_{u}$.

The property mentioned above implies that $v(\gamma_{i} - \gamma_{i+1}) =
v(i)$ for any $\gamma_{i} \in\varLambda_{i}$ and $\gamma_{i+1} \in
\varTheta_{i+1}$.

Hence for any set $N_{1}$ of $c_{1}$ expansions of $\varLambda_{i}$ and any
set $N_{2}$ of $c_{2}$ expansions of $\varTheta_{i+1}$, if $q_{1} =
\prod_{\eta\in N_{1}}(Y - \eta)$, $q_{2} = \prod_{\eta\in N_{2}}(Y - \eta)$
and $q = q_{1}q_{2}$, then
\[
v_{\gamma_{i}}(q_{2}) = c_{2} v(i)
\]


Since $v_{f_{\varLambda_{i}}}(q_{1})$ is the minimum of $v_{\gamma_{i}}%
(q_{1})$ for $\gamma_{i} \in\varLambda_{i}$, we obtain that
\[
\min_{\gamma\in\varLambda_{i}} v_{\gamma}(q) = v_{f_{\varLambda_{i}}}(q_{1}) +
c_{2} v(i)
\]


Similarly,
\[
\min_{\gamma\in\varTheta_{i+1}} v_{\gamma}(q) = c_{1} v(i) +
v_{f_{\varTheta_{i+1}}}(q_{2})
\]


We conclude that
\[
v_{f_{\varTheta_{i}}}(q) = \min\{v_{f_{\varLambda_{i}}}(q_{1}) + c_{2} v(i),
c_{1} v(i) + v_{f_{\varTheta_{i+1}}}(q_{2})\}
\]


This allows us to compute inductively
\[
o(\varTheta_{i}, c) = \max_{c_{1} + c_{2} = c} v_{f_{\varTheta_{i}}%
}(p_{\varLambda_{i}}(c_{1})p_{\varTheta_{i+1}}(c_{2}))
\]
and define $p_{\varTheta_{i}}(c)$ as the polynomial for which the maximum is obtained.

The numerator of the element of degree $d$ in the integral basis is
\[
p_{d} = p_{\varTheta_{1}}(d)
\]
and the denominator is $x^{\lfloor v_{f}(p_{d}) \rfloor}$.

For computing the best polynomials in each block, we can use this strategy
recursively. We summarize the method in Algorithm \ref{alg:directApproach}.

\begin{algorithm}                      % enter the algorithm environment
\caption{Local integral basis, combinatorial approach}          % give the algorithm a caption
\label{alg:directApproach}
\begin{algorithmic}[1]
\REQUIRE $L = \{L_1, \dots, L_r\}$, where $L_i = \{(\varGamma_{(i,1)}, f_{\varGamma_{(i,1)}}),
\dots, (\varGamma_{(i, u_i)}, f_{\varGamma_{(i, u_i)}})\}$, $1 \leq i \leq r$, are
the singular parts of the conjugacy classes of some Puiseux blocks of $f$ and their corresponding factors developed
up to $X$--degree $e$.
\ENSURE $\{(p_0, o_0), \dots, (p_{m}, o_{m})\}$ such that $p_i \in K[X,Y]$ ($0 \leq i \leq m$) has $Y$--degree $i$ and
maximal valuation at $g = f_{\varGamma_{(1,1)}}\cdots f_{\varGamma_{(r,u_r)}}$; and $o_i \in \Q_{\geq 0}$, $o_i = v_g(p_i) $, where $m = \deg_Y(g)$.
\IF{$r = 1$}
\RETURN $\{\IntegralElement(L_1, c)\}_{c= 0, \dots, m}$
\ELSE
\STATE $f_{L_i} = \prod_{j = 1}^{u_i} f_{\varGamma_{(i, j)}}$, for $i = 1, \dots, r$.
\STATE Group the lists $L_1, \dots, L_r$ in lists $\varLambda_1, \dots, \varLambda_u$
such that all the expansions in the same
list $\varLambda_{i}$ have the same or conjugate initial terms
(without considering the common rational part of all expansions, if any),
ordered by the initial exponent in increasing order
\STATE $f_{\varLambda_i} = \prod_{L_j \in \varLambda_i} f_{L_j}$ and $m_i = \deg(f_{\varLambda_i})$, for $i = 1, \dots, u$
\STATE $\varTheta_u = \varLambda_u$, $f_{\varTheta_u} = f_{\varLambda_u}$
\STATE $\{(p_{\varTheta_u}(c), o(\varTheta_u, c))\}_{c = 0, \dots, m_u} = \LocalIntegralBasis(\varTheta_u)$
\FOR{$i = u-1, \dots, 1$}
\STATE $\varTheta_i = \varLambda_i \cup \varTheta_{i+1}$, $f_{\varTheta_i} = f_{\varLambda_i} f_{\varTheta_{i+1}}$
\STATE $\{(p_{\varLambda_i}(c), o(\varLambda_i, c))\}_{c = 0, \dots, m_i} = \LocalIntegralBasis(\varLambda_i)$
\FOR{$0 \leq c \leq m_i + \dots + m_u$}
\STATE $C = \{(c_1, c_2) \in \Z_{\leq 0}^2 \mid c_1 + c_2 = c\mbox{, }0 \leq c_1 \leq m_i\mbox{, }0 \leq c_2 \leq m_{i+1} + \dots + m_u\}$
\STATE $o(\varTheta_{i}, c) = \max_{(c_1, c_2) \in C} v_{f_{\varTheta_{i}}%
}(p_{\varLambda_{i}}(c_{1})p_{\varTheta_{i+1}}(c_{2}))$
\STATE $p_{\varTheta_i}(c) = $ the polynomial for which the maximum is obtained
\ENDFOR
\ENDFOR
\RETURN $\{(p_{\varTheta_1}(c), o(\varTheta_1, c))\}_{c = 0, \dots, m}$.
\ENDIF
\end{algorithmic}
\end{algorithm}


We apply this algorithm to Example \ref{exampleTwoBranches}, to compare with
Algorithm \ref{alg:loclocbasis}.

\begin{example}
\todo{TWOBRANCHES modify this example using the new example with no inclusion}
Let $f = (Y^{3}-X^{7})(Y^{2} - X^{3})+Y^{6} \in{\mathbb{Q}}[X,Y]$. The input
for Algorithm \ref{alg:directApproach} is $L = \{L_{1}, L_{2}\} =
\{\{(\varGamma_{1}, f_{1})\}, \{(\varGamma_{2}, f_{2})\}\}$, where
$\varGamma_{1}$ and $\varGamma_{2}$ are the same as in Example
\ref{exampleTwoBranches-locloc} and $f_{1} = h_{2}$, $f_{2} = h_{1}$.

We have $\Lambda_{1} = L_{1}$, $\Lambda_{2} = L_{2}$ and $u = 2$. Also
$f_{\Lambda_{1}} = f_{1}$, $f_{\Lambda_{2}} = f_{2}$, $m_{1} = 3$ and $m_{2} =
2$.

Hence $\varTheta_{2} = \Lambda_{2}$, $f_{\varTheta_{2}} = f_{\Lambda_{2}}$ and
the local integral basis corresponding to $\varTheta_{2}$ is $\{(1, 0), (y,
1), (f_{2}, \infty)\}$.

For $i = 1$, $\varTheta_{1} = L$ and $f_{\varTheta_{1}} = f_{1}f_{2}$. The
local integral basis corresponding to $\varLambda_{1}$ is $\{(1, 0), (y, 2),
(y^{2}, 4), (f_{1}, \infty)\}$.

Now we have to choose the best combinations for each $c = 0, \dots, m$. For
example, for $c = 3$, we try the pairs $(c_{1}, c_{2}) \in\{(1,2), (2,1), (3,
0)\}$ and find that the best choice is $(c_{1}, c_{2}) = (1,2)$, for which
$v_{f_{\vartheta_{1}}}(p_{\Lambda_{1}(2)}p_{\varTheta_{2}}(0)) =
v_{f_{\vartheta_{1}}}(Y f_{2}) = 16/3$.

Applying the formulas for all $c$, $0 \leq c \leq5$, we get the output
\[
\{(1, 0), (y, 1), (f_{2}, 3), (y f_{2}, 16/3), (y^{2} f_{2}, 23/3), (f_{1}
f_{2}, \infty)\}.
\]


The first five elements define the integral basis $\left\langle 1, \frac{y}%
{x}, \frac{f_{2}}{x^{3}}, \frac{f_{2}y}{x^{5}}, \frac{f_{2}y^{2}}{x^{7}%
}\right\rangle _{K[x]}$.
\end{example}

%\section{The nodal locus}
%In this section, we explain how to compute the integral basis of the ideal localized at the nodal locus, without decomposing it into its singular points.


\section{Timings}

\label{sec timings}

We present some timings which compare the implementation of our algorithm
\todo{Which algorithm is meant?
We also have the appendix.} in \textsc{Singular}{} with that of van Hoeij's
algorithm in \textsc{Maple}. We compute integral bases for $A=\mathbb{Q}%
[X,Y]/\langle f\rangle$ with polynomials $f$ as specified. All timings are in
seconds, taken on an AMD Opteron $6174$ machine with $48$ cores, $2.2$GHz, and
$128$GB of RAM running a Linux operating system. We do not make use of
parallel computation so far; this is subject to ongoing implementation work.
In the cases where $f$ has only one singular point, this point is given as
part of the input to both algorithms. That is, no computation or decomposition
of the singular locus is done. In the cases where the singular locus has more
than one point, the timings are taken for decomposing the singular locus,
computing the local contributions, and combining these.
%In some tables, to quantify the influence of the primary decomposition, we give in addition to this (column
%\emph{global}) also the timings for the local algorithm applied to a
%singularity at $(X, Y)  = (0,0)$ in the column
%\emph{origin}.
We remark that for obtaining the integral bases, singularities at infinity of
the curve $\{f=0\}$ do not matter.

\todo{We should have more columns for the timings.
1) Use the global normalization algorithm. 2) Use the local
normalization algorithm. 2) Magma.}

\subsection{\texorpdfstring{$A_{k}$}{Ak}-singularity}

The plane curves with defining equation $f(X,Y)=Y^{2}+X^{k+1}+Y^{d}$, $k\geq
1$, $d\geq3$ have exactly one singularity at the origin, which is of type
$A_{k}$.

%\begin{table}[h]
%\caption{Timings for $A_{k}$ singularities}%
%\label{table Ak}%
%

\[%
\begin{tabular}
[c]{ll|ll|}\hline
\multicolumn{1}{|r}{$k$} & \multicolumn{1}{|r|}{$d$} &
\multicolumn{1}{|l|}{\textsc{Singular}{}} &
\multicolumn{1}{|l|}{\textsc{Maple}}\\\hline
\multicolumn{1}{|r}{$5$} & \multicolumn{1}{|r|}{$10$} &
\multicolumn{1}{|r}{$0$} & \multicolumn{1}{|r|}{$0$}\\
\multicolumn{1}{|r}{$5$} & \multicolumn{1}{|r|}{$100$} &
\multicolumn{1}{|r}{$0$} & \multicolumn{1}{|r|}{$2.4$}\\
\multicolumn{1}{|r}{$5$} & \multicolumn{1}{|r|}{$500$} &
\multicolumn{1}{|r}{$14$} & \multicolumn{1}{|r|}{$262$}\\
\multicolumn{1}{|r}{$50$} & \multicolumn{1}{|r|}{$60$} &
\multicolumn{1}{|r}{$0$} & \multicolumn{1}{|r|}{$2.6$}\\
\multicolumn{1}{|r}{$50$} & \multicolumn{1}{|r|}{$100$} &
\multicolumn{1}{|r}{$0$} & \multicolumn{1}{|r|}{$7$}\\
\multicolumn{1}{|r}{$50$} & \multicolumn{1}{|r|}{$500$} &
\multicolumn{1}{|r}{$16$} & \multicolumn{1}{|r|}{$385$}\\
\multicolumn{1}{|r}{$90$} & \multicolumn{1}{|r|}{$100$} &
\multicolumn{1}{|r}{$0$} & \multicolumn{1}{|r|}{$12$}\\
\multicolumn{1}{|r}{$90$} & \multicolumn{1}{|r|}{$500$} &
\multicolumn{1}{|r}{$13$} & \multicolumn{1}{|r|}{$509$}\\
\multicolumn{1}{|r}{$400$} & \multicolumn{1}{|r|}{$500$} &
\multicolumn{1}{|r}{$16$} & \multicolumn{1}{|r|}{$1689$}\\\hline
\end{tabular}
\ \
\]
%\end{table}


\subsection{\texorpdfstring{$D_{k}$}{Dk}-singularity}

The plane curves with defining equation $f(X,Y)=X(X^{k-1}+Y^{2})+Y^{d}$,
$k\geq2$, $d\geq3$ have exactly one $D_{k}$-singularity at the origin.

%\begin{table}[h]
%\caption{Timings for $D_{k}$ singularities}%
%\label{table Dk}%
\[%
\begin{tabular}
[c]{|c|c|l|l|}\hline
$k$ & \multicolumn{1}{|c|}{$d$} & \multicolumn{1}{|l|}{\textsc{Singular}} &
\textsc{Maple}\\\hline
\multicolumn{1}{|r|}{$5$} & \multicolumn{1}{|r|}{$10$} &
\multicolumn{1}{|r|}{$0$} & \multicolumn{1}{|r|}{$0$}\\
\multicolumn{1}{|r|}{$5$} & \multicolumn{1}{|r|}{$100$} &
\multicolumn{1}{|r|}{$2$} & \multicolumn{1}{|r|}{$2.6$}\\
\multicolumn{1}{|r|}{$5$} & \multicolumn{1}{|r|}{$500$} &
\multicolumn{1}{|r|}{$51$} & \multicolumn{1}{|r|}{$206$}\\
\multicolumn{1}{|r|}{$50$} & \multicolumn{1}{|r|}{$60$} &
\multicolumn{1}{|r|}{$1$} & \multicolumn{1}{|r|}{$14$}\\
\multicolumn{1}{|r|}{$50$} & \multicolumn{1}{|r|}{$100$} &
\multicolumn{1}{|r|}{$2$} & \multicolumn{1}{|r|}{$45$}\\
\multicolumn{1}{|r|}{$50$} & \multicolumn{1}{|r|}{$500$} &
\multicolumn{1}{|r|}{$49$} & \multicolumn{1}{|r|}{$2114$}\\
\multicolumn{1}{|r|}{$90$} & \multicolumn{1}{|r|}{$100$} &
\multicolumn{1}{|r|}{$2$} & \multicolumn{1}{|r|}{$142$}\\
\multicolumn{1}{|r|}{$90$} & \multicolumn{1}{|r|}{$500$} &
\multicolumn{1}{|r|}{$50$} & \multicolumn{1}{|r|}{$5918$}\\
\multicolumn{1}{|r|}{$400$} & \multicolumn{1}{|r|}{$500$} &
\multicolumn{1}{|r|}{$50$} & \multicolumn{1}{|r|}{$>6000$}\\\hline
\end{tabular}
\
\]
%\end{table}


\subsection{Ordinary multiple points}

We consider random curves of degree $d$ with an ordinary $k$-fold point at the
origin. The defining polynomials were generated by the function
\texttt{polyDK} from the \textsc{Singular}{} library
\texttt{integralbasis.lib} (using the random seed $1231$).

%\begin{table}[h]
%\caption{Timings for curves of degree $d$ with an ordinary $k$-fold point}%
%\label{table ordinary k fold degree d}%
%

\[%
\begin{tabular}
[c]{|c|c|l|l|}\hline
$k$ & \multicolumn{1}{|c|}{$d$} & \textsc{Singular} & \textsc{Maple}\\\hline
\multicolumn{1}{|l|}{$5$} & \multicolumn{1}{|l|}{$10$} & $0$ & $0$\\
\multicolumn{1}{|l|}{$15$} & \multicolumn{1}{|l|}{$20$} & $0$ & $3$\\
\multicolumn{1}{|l|}{$15$} & \multicolumn{1}{|l|}{$30$} & $1$ & $1095$\\
\multicolumn{1}{|l|}{$20$} & \multicolumn{1}{|l|}{$25$} & $0$ & $13$\\
\multicolumn{1}{|l|}{$20$} & \multicolumn{1}{|l|}{$30$} & $1$ & $352$\\\hline
\end{tabular}
\ \
\]


\subsection{Curves with many \texorpdfstring{$A_{k}$}{Ak} singularities}

The plane curves with defining equations
\[
f=\left(  X^{k+1}+Y^{k+1}+Z^{k+1}\right)  ^{2}-4\left(  X^{k+1}Y^{k+1}%
+Y^{k+1}Z^{k+1}+Z^{k+1}X^{k+1}\right)
\]
were given in \cite{Hirano1992} and have $3\left(  k+1\right)  $ singularities
of type $A_{k}$ if $n$ is even. To ensure that all singularities of the curves
are in the affine chart $\{Z\neq0\}$, we substitute $Z=2X-Y+1$.

%\begin{table}[h]
%\caption{Timings for curves with $3\left(  n+1\right)  $ singularities of type
%$A_{n}$}%
%\label{table Hirano}%
\[%
\begin{tabular}
[c]{|r|l|l|}\hline
$k$ & \textsc{Singular} & \textsc{Maple}\\\hline
$6$ & \multicolumn{1}{|r|}{$2$} & \multicolumn{1}{|r|}{$11$}\\
$8$ & \multicolumn{1}{|r|}{$18$} & \multicolumn{1}{|r|}{$109$}\\
$10$ & \multicolumn{1}{|r|}{$240$} & \multicolumn{1}{|r|}{$4756$}\\\hline
\end{tabular}
\]
%\end{table}
The plane curves with defining equations
\[
f_{5,n}=X^{2n}+Y^{2n}+Z^{2n}+2(X^{n}Z^{n}-X^{n}Y^{n}+Y^{n}Z^{n})
\]
were given in \cite{Cogolludo1999} and have $3n$ singularities of type
$A_{n-1}$ if $n$ is odd. We now substitute $Z=X-2Y+1$.

%\begin{table}[h]
%\caption{Timings for curves with $3n$ singularities of type $A_{n-1}$.}%
%\label{table COGOLLUDO}%
\[%
\begin{tabular}
[c]{|r|l|l|}\hline
$n$ & \textsc{Singular} & \textsc{Maple}\\\hline
$5$ & \multicolumn{1}{|r|}{$1$} & \multicolumn{1}{|r|}{$3$}\\
$7$ & \multicolumn{1}{|r|}{$2$} & \multicolumn{1}{|r|}{$37$}\\
$9$ & \multicolumn{1}{|r|}{$27$} & \multicolumn{1}{|r|}{$478$}\\
$11$ & \multicolumn{1}{|r|}{$53$} & \multicolumn{1}{|r|}{$>6000$}\\\hline
\end{tabular}
\
\]
%\end{table}


\subsection{More general singularities\label{sec some more general examples}}

We now consider some examples of curves which have singularities of a type
other than $ADE$ or ordinary multiple points:

\begin{enumerate}
{\scriptsize \setlength{\leftskip}{-2em} }

\item {\scriptsize $f=-X^{15}+21X^{14}-8X^{13}Y+6X^{13}+16X^{12}%
Y-20X^{11}Y^{2}+X^{12}-8X^{11}Y+36X^{10}Y^{2}-24X^{9}Y^{3}-4X^{9}Y^{2}%
+16X^{8}Y^{3}-26X^{7}Y^{4}+6X^{6}Y^{4}-8X^{5}Y^{5}-4X^{3}Y^{6}+Y^{8}$: one
singularity at the origin with multiplicity $m=8$ and delta invariant
$\delta=42$, a node, and a set of $6$ conjugate nodes. [Pfister]} \todo{Pfister refers to?}

\item {\scriptsize $f=(Y^{4}+2X^{3}Y^{2}+X^{6}+X^{5}Y)^{3}+X^{11}Y^{11}$: one
singularity at the origin with $m=12$ and $\delta=133$. [Pfister]} \todo{Pfister refers to?}

\item {\scriptsize $f=(Y^{5}+Y^{4}X^{7}+2X^{8})(Y^{3}+7X^{4})(Y^{7}%
+2X^{12})(Y^{11}+2X^{18})+Y^{30}$: one singularity at the origin with $m=26$
and $\delta=523$.}

\item {\scriptsize $f=(Y^{15}+2X^{38})(Y^{19}+7X^{52})+Y^{36}$: one
singularity at the origin with $m=34$ and $\delta=1440$.}

\item {\scriptsize $f=(Y^{15}+2X^{38})(Y^{19}+7X^{52})+Y^{100}$: higher
degree, but same type of singularity.}

\item {\scriptsize \label{exam:Hoeij1} $f=Y^{40}+XY^{13}+X^{4}Y^{5}%
+X^{5}+2X^{4}+X^{3}$: one double point with $\delta=2$ and one triple point
with $\delta=19$ (see \cite[Section 6.1]{vanHoeij94}).}

\item {\scriptsize $f=Y^{200}+XY^{13}+X^{4}Y^{5}+X^{5}+2X^{4}+X^{3}$: higher
degree, but same type of singularity.}

\item {\scriptsize $f=(Y^{35}+Y^{34}X^{7}+2X^{38})(Y^{33}+7X^{44}%
)(Y^{37}+2X^{52})+Y^{110}$: one singularity at the origin with $m=105$ and
$\delta=6528$.}
\end{enumerate}

Although some of the examples have only one singularity at the origin, we
apply the local and the global algorithm in all cases. That is, in the columns
labelled \emph{Origin}, we compute the timings for the local contribution to
the integral basis at the origin, which does not involve the decomposition of
the singular locus. In the columns labelled \emph{Global}, we decompose the
singular locus, compute the local contributions, and combine them.

\todo{How are the Maple timings under origin taken?
In the Singular* column, I do not understand the two bad timings in the middle.}
\[%
\begin{tabular}
[c]{c|c|c|c|c|c|c}
& \multicolumn{2}{|c|}{Origin} & \multicolumn{3}{|c|}{Global} & \\\hline
\multicolumn{1}{|c|}{No.} & \textsc{Singular} & \textsc{Maple} &
\textsc{Singular}{} & \textsc{Singular}{}$^{*}$ & \textsc{Maple} &
\multicolumn{1}{|l|}{$Y$--degree}\\\hline
\multicolumn{1}{|c|}{$1$} & \multicolumn{1}{|r|}{$0$} &
\multicolumn{1}{|r|}{$0$} & \multicolumn{1}{|r|}{$0$} &
\multicolumn{1}{|r|}{$5$} & \multicolumn{1}{|r|}{$1$} &
\multicolumn{1}{|r|}{$8$}\\
\multicolumn{1}{|c|}{$2$} & \multicolumn{1}{|r|}{$36$} &
\multicolumn{1}{|r|}{$2$} & \multicolumn{1}{|r|}{$37$} &
\multicolumn{1}{|r|}{$37$} & \multicolumn{1}{|r|}{$2$} &
\multicolumn{1}{|r|}{$12$}\\
\multicolumn{1}{|c|}{$3$} & \multicolumn{1}{|r|}{$2$} &
\multicolumn{1}{|r|}{$6$} & \multicolumn{1}{|r|}{$>6000$} &
\multicolumn{1}{|r|}{$41$} & \multicolumn{1}{|r|}{$16$} &
\multicolumn{1}{|r|}{$30$}\\
\multicolumn{1}{|c|}{$4$} & \multicolumn{1}{|r|}{$1$} &
\multicolumn{1}{|r|}{$10$} & \multicolumn{1}{|r|}{$1$} &
\multicolumn{1}{|r|}{$>6000$} & \multicolumn{1}{|r|}{$12$} &
\multicolumn{1}{|r|}{$36$}\\
\multicolumn{1}{|c|}{$5$} & \multicolumn{1}{|r|}{$0$} &
\multicolumn{1}{|r|}{$47$} & \multicolumn{1}{|r|}{$1$} &
\multicolumn{1}{|r|}{$>6000$} & \multicolumn{1}{|r|}{$115$} &
\multicolumn{1}{|r|}{$100$}\\
\multicolumn{1}{|c|}{$6$} & \multicolumn{1}{|r|}{$1$} &
\multicolumn{1}{|r|}{$0$} & \multicolumn{1}{|r|}{$1$} &
\multicolumn{1}{|r|}{$1$} & \multicolumn{1}{|r|}{$1$} &
\multicolumn{1}{|r|}{$40$}\\
\multicolumn{1}{|c|}{$7$} & \multicolumn{1}{|r|}{$9$} &
\multicolumn{1}{|r|}{$12$} & \multicolumn{1}{|r|}{$35$} &
\multicolumn{1}{|r|}{$10$} & \multicolumn{1}{|r|}{$50$} &
\multicolumn{1}{|r|}{$200$}\\
\multicolumn{1}{|c|}{$8$} & \multicolumn{1}{|r|}{$154$} &
\multicolumn{1}{|r|}{$5708$} & \multicolumn{1}{|r|}{$>6000$} &
\multicolumn{1}{|r|}{$>6000$} & \multicolumn{1}{|r|}{$>6000$} &
\multicolumn{1}{|r|}{$110$}\\\hline
\end{tabular}
\
\]
%\end{table}


In the column \textsc{Singular}{}$^{*}$, we use modular techiques for
computing the decomposition of the singular locus. \newpage In this table, the
computations in Singular that did not finish are all due to the computation of
the
\todo{This can easily be avoided. First insert the origin into the equations of the
singular locus to check whether the origin is a singular point. Then there are
two possibilities. 1) Compute a global and a local Groebner basis
and compare dimensions. See Lecture 9 in the book Decker/Lossen.
2) Saturate in the maximal ideal of the origin. That might be slower than 1),
but if there are more singularities, it will lead to an easier primary
decomposition problem. Santiago, please make tests.} decomposition of the
singular locus (although we know that these examples have only one singularity
at the origin).

\vspace{0.5cm}

We note that in most cases, our proposed algorithm is much faster than the
algorithm implemented in \textsc{Maple}{}. Note, however, that in the last
table, there is one example in which \textsc{Singular}{} is significantly
slower than \textsc{Maple}{}. In this example, the algorithm runs into an
algebraic field extension of high degree. At current state, the handling of
such extensions in \textsc{Singular}{} is far from being optimal.
%We expect this issue to be resolved in the near future, which will improve
%our timings in these cases.


\bibliographystyle{plain}
\bibliography{mybib}



\end{document}


%\begin{remark}
%We always can find an integral basis of type
%\[
%1,\frac{p_{1}}{x^{e_{1}}},\dots,\frac{p_{n-1}}{x^{e_{n-1}}}.
%\]
%\todo{fix this: if only singularities with $x = 0$ occur. Move to a later section.}
%\end{remark}

\begin{remark}
Let $L=\overline{K}$ and let $A=L[x,y]=L[X,Y]/\left\langle f\right\rangle $ be
denote the coordinate ring of $C$ with $f\in K[X][Y]$ monic in $Y$ of degree
$n$. Let $a_{1},...,a_{s}\in L$ denote the $x$-coordinates of the
singularities of $C$. Then there are $p_{i}\in K[X][Y]$ with $\deg_{Y}%
(p_{i})=i$ for all $i$ and $e_{i,j}\in\mathbb{N}_{0}$ with $e_{i,j}\geq
e_{i-1,j}$, such that, as an $L[X]$-module%
\[
\bar{A}=_{L[X]}\left\langle \frac{\bar{p}_{0}}{\bar{q}_{0}},\ldots,\frac
{\bar{p}_{n-1}}{\bar{q}_{n-1}}\right\rangle \subseteq
Q(A)=L(X)[Y]/\left\langle f\right\rangle
\]
with%
\[
q_{i}=%
%TCIMACRO{\dprod \limits_{j=1}^{s}}%
%BeginExpansion
{\displaystyle\prod\limits_{j=1}^{s}}
%EndExpansion
(X-a_{j})^{e_{i,j}}\in K[X]\text{.}%
\]


If $a_{1}=0$ and $P=\left\langle x,y\right\rangle $ then, as an
$L[X]_{\left\langle X\right\rangle }$-module,%
\[
\overline{A_{P}}=\left\langle \frac{\bar{p}_{0}}{x^{e_{0,1}}},\ldots
,\frac{\bar{p}_{n-1}}{x^{e_{n-1,1}}}\right\rangle
\]
and%
\[
\overline{\widehat{A_{P}}}=\left\langle \frac{\bar{p}_{0}}{x^{e_{0,1}}}%
,\ldots,\frac{\bar{p}_{n-1}}{x^{e_{n-1,1}}}\right\rangle \subseteq
Q(A_{P})\otimes_{A_{P}}\widehat{A_{P}}%
\]
as an $L[[X]]$-module.
\end{remark}


\begin{proof}
By van Hoeij's algorithm the claim for $\bar{A}$ is clear. The second claim
follows since $\_\otimes_{A}A_{P}$ is right exact, $\overline{A_{P}}%
=\overline{A}_{P}$, and $x-a$ is a unit in $\overline{A_{P}}$ if $a\neq0$. The
third claim follows since $\_\otimes_{A_{P}}\widehat{A_{P}}$ is right exact
and $\widehat{\overline{A_{P}}}=\overline{\widehat{A_{P}}}$, since the
semilocal ring $\overline{A_{P}}$ is excellent.
\end{proof}